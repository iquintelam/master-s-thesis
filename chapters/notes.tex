the SAFT-$\gamma$ Mie is straightforward to implement.

For a more thorough explanation of this models, readers are referred to specific reviews. 

%http://aip.scitation.org/doi/pdf/10.1063/1.4818908

Typically, in comparison to its all-atom counterpart, the coarse-grained force field smoothens out the energy landscape, and thereby helps to avoid local energy minima “traps,” see Figure 3. Coarse-graining also affects thermodynamic properties of a modeled system, particularly the balance between enthalpy and entropy. Reduction of the degrees of freedom affects the entropy of the simulation system, which is compensated by reduced enthalpic terms. In turn, a coarse-grained model may accurately reproduce free energy differences but contributing enthalpy and entropy values may be inaccurate. Such limitations are typical for the majority of coarse-grained models \cite{kmiecik2016}. 
Coarse-Grained Protein Models and Their Applications


the pair potential is based on molecule specific
parameters, such as the ranges of the repulsive and attractive interactions, the size of
the bead and the energy parameter which characterises the strength of the attractive interaction.

At short distances, the potential is
invariably repulsive and tends to infinity, whereas at larger separations, the potential is
attractive and tends to zero so that the energy remains finite. The total potential energy
is usually assumed as a sum of both, the repulsive and the attractive, contributions. Some
of the commonly used pair potentials are illustrated in Figure 2.1 and briefly described
below.

sing the versatile exponents has been shown to provide
a significant improvement of the vapour pressure and the second-derivative thermodynamic
properties of real fluids, such as speed of sound, heat capacity, and compressibility \cite{avendano2011,lafitte2013,lafitte2006}.

In conventional statistical mechanics, it is common to replace time
averages with ensemble averages. This hypothesis, currently referred to as the ergodic
hypothesis, states that the time and the ensemble averages of two systems with the same tate variables, e.g., N, V , and T are identical in infinite time and thermodynamic limits.

By contrast, commonly used
bottom-up approaches often make use of temperature dependent parameters that have to
be re-derived at various state points. The parameters obtained with our SAFT-gamma top-down
methodology can be used as a direct input in molecular simulation. Unlike intermolecular
interactions can also be obtained by using appropriate experimental data for the mixtures,
but often combining rules are employed for some of the parameters as described in the
next section.

The parameterisation of a force field is a non-trivial and a cumbersome task. Some of
the main requirements on a force field include the accuracy, transferability, and robustness. Unfortunately, there is no generic force field that can reproduce all properties with a
unique parameter set.



mostly the intermolecular interactions and the torsional terms. In this work, our main focus lies on the
dispersion interactions. The dispersion interactions are normally described by the intermolecular potentials presented in Section 2.1.2.

The LJ potential function forms the basis for a large number of widely
used molecular force fields, such as OPLS [73], TraPPE [74], NERD [75] force fields, and
the TIP [76] and SPC [77] models of water, to mention but a few. The popular coarsegrained MARTINI force field [78] is also based on the LJ representation.


Review

which are (i) the

choice of a suitable model Hamiltonian, (ii) the choice of a

sampling protocol that allows generating a representative

ensemble of configurations, and (iii) the choice of an estimator

for the free energy difference

%Recent work \cite{mobley2014,mobley2017} made available a big database of hydration free energy of small molecules using the GAFF force field. A comparision of polar and nonpolar contributions to these hydratation free energy inidcated the significance of each the terms  \cite{izairi2017}. Garrido \textit{et al.} \citep{garrido,garrido2011} calculated the free energy of solvation of large alkanes in 1-octanol and water with three different force fields (TraPPE, Gromos, OPLS-AA/TraPPE) and the solvation free energy of propane and benzene in non aqueous solvents like n-hexadecane, n-hexane, ethyl benzene, acetone  with the force fields TraPPE-UA and TraPPE-AA. Roy \textit{et al.} \citep{roy2017} addressed the choice of the Lennard Jones parameters for predicting solvation free energy in 1-octanol. Gon\c{c}alves and Stassen \citep{goncalves} calculated the free energy of solvation in the solvents tetrachloride, chloroform and benzene with GROMOS force field. Using the GAFF and the polarazible AMOEBA force fields, Mohamed \textit{et al.} \citep{mohamed2016} evaluated the solvation free energy of small molecules in toluene, chloroform and acetonitrile and obtained a mean unsigned error of 1.22 $kcal/mol$ for the AMOEBA and 0.66 $kcal/mol$ for the GAFF. 



benchmark
Specifically, in some
situations, free energy calculations appear to be capable
of achieving RMS errors in the 1-2 kcal/mol range with
current force fields, even in prospective applications.

The most immediate application of these techniques is to guide synthesis for lead optimization, but applications to scaffold
hopping and in other areas also appear possible.

At the same time, it is clear that not all situations are
so favorable, so it is worth asking what level of accuracy
is actually needed

The present review focuses on a class of methods in
which free energy differences are computed with simulations that sample Boltzmann distributions of molecular configurations. These samples are usually generated by molecular dynamics (MD) simulations [92], with
the system effectively coupled to a heat bath at constant temperature, but Monte Carlo methods may also
be used [32, 120, 121]. 

In either case, the energy of a
given configuration is provided by a potential function,
or force field, which estimates the potential energy of
a system of solute and solvent molecules as a function
of the coordinates of all of its atoms.

In all cases, however, the calculations yield the free energy
difference between two states of a molecular system, and
they do so by computing the reversible work for changing
the initial state to the final one. 