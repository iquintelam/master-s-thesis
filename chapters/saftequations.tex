
\chapter{Detailing of the SAFT-VR Mie Equation of State } \label{restodaseq}
The term $a_{1}$  in Eq. \ref{eqn:aM} is the first-order mean-attractive energy of the mixture \cite{lafitte2013}, and is given by

\begin{equation}
a_{1} = \sum_{i=1}^{n} \sum_{j=1}^{n} x_{s,i} x_{s,j} a_{1,ij},
\end{equation}	
where $a_{1,ij}$ is equal to

\begin{equation}
\begin{aligned}
a_{1,ij} {}=& \mathcal{C}_{ij} \lbrace x_{0,ij}^{\lambda _{a,ij}} [a_{1,ij}^{s}(\rho _{s}; \lambda _{a,ij}) + B_{ij}(\rho _{s}; \lambda _{a,ij})] \\
& - x_{0,ij}^{\lambda _{r,ij}} [a_{1,ij}^{s}(\rho _{s}; \lambda _{r,ij}) + B_{ij}(\rho _{s}; \lambda _{r,ij})] \rbrace ,
\end{aligned}
\label{a1ijap}
\end{equation}
with $\mathcal{C}_{ij}$ equals to
\begin{equation}
\mathcal{C}_{ij} = \frac{\lambda_{r,ij}}{\lambda_{r,ij} - \lambda_{a,ij}} \left(\frac{\lambda_{r,ij}}{\lambda_{a,ij}} \right)^{\left( \frac{\lambda_{a,ij}}{\lambda_{a,ij} - \lambda_{a,ij}} \right)}.
\end{equation}

Also in Eq. \ref{a1ijap}, $B_{ij}(\rho _{s}; \lambda _{ij})$ is equal to

\begin{equation}
B_{ij}(\rho _{s}; \lambda _{ij}) =  2 \pi \rho _{s} d_{ij}^{3} \epsilon _{ij} \left[\dfrac{1 - \zeta _{x}/2}{(1-\zeta _{x})^3} I_{\lambda , ij} - \dfrac{9 \zeta _{x} (1 + \zeta _{x})}{2(1-\zeta _{x})^3} J_{\lambda , ij} \right],
\end{equation}

\begin{equation}
\zeta _{x} = \frac{\pi \rho _{s}}{6} \sum_{i=1}^{n} \sum_{j=1}^{n} x_{s,i} x_{s,j} d_{ij}^{3} .
\end{equation}

Here, $I_{\lambda , ij}$ and $J_{\lambda , ij}$ are given by

\begin{equation}
I_{\lambda , ij} = \dfrac{ (x_{0,ij})^{3 - \lambda _{ij}} - 1}{\lambda _{ij} -3},
\end{equation}

\begin{equation}
J_{\lambda , ij} = \dfrac{ (x_{0,ij})^{4 - \lambda _{ij}}(\lambda _{ij} -3) - (x_{0,ij})^{3 - \lambda _{ij}}(\lambda _{ij} -4) -1}{(\lambda _{ij} -3)(\lambda _{ij} -4)}.
\end{equation}

The perturbation terms $a_{1,ij}^{s}$ are obtained with the following equation:

\begin{equation}
a_{1,ij}^{S} (\rho _{s} ; \lambda _{ij}) = -2 \rho _{s} \left(\dfrac{\pi \epsilon _{ij} d_{ij}^{3}}{\lambda _{ij} -3}\right) \dfrac{1 - \zeta _{x}^{eff}(\lambda _{ij})/2}{[1- \zeta _{x}^{eff}(\lambda _{ij})]^3},
\end{equation}
where $\zeta _{x}^{eff}(\lambda _{ij})$ is the effective packing fraction  obtained within a van der Waals one-fluid approximation \cite{lafitte2013}. It is equal to

\begin{equation}
\zeta _{x}^{eff}(\lambda _{ij}) = c_{1}(\lambda_{ij}) \zeta_{x} + c_{2}(\lambda_{ij}) \zeta_{x}^{2} + c_{3}(\lambda_{ij}) \zeta_{x}^3 + c_{4}(\lambda_{ij}) \zeta_{x}^4 . 
\end{equation}

Here, the coefficients $c_{1}$, $c_{2}$, $c_{3}$ and $c_{4}$ are 

\begin{equation}
\begin{aligned}
\left[ \begin{array}{c} c_{1} \\ c_{2} \\ c_{3} \\ c_{4} \end{array} \right] = 
\left[ \begin{array}{c} 0.81096 \quad 1.7888 \quad -37.578 \quad 92.284 \\ 1.0205 \quad -19.341 \quad 151.26 \quad -463.50 \\ -1.9057 \quad 22.845 \quad -228.14 \quad 973.92 \\ 1.0885 \quad  -6.1962 \quad 106.98 \quad -677.64 \end{array} \right] \cdot \left[ \begin{array}{c} 1 \\ 1/\lambda_{ij} \\ 1/\lambda_{ij}^{2} \\ 1/\lambda_{ij}^{3} \end{array} \right].
\end{aligned}
\end{equation}

The term $a_{2}$ in Eq. \ref{eqn:aM} has a similar formulation to $a_{1}$. It is given by

\begin{equation}
a_{2} = \sum_{i=1}^{n} \sum_{j=1}^{n} x_{s,i} x_{s,j} a_{2,ij},
\end{equation}	
where $a_{2,ij}$ is equal to

\begin{equation}
\begin{aligned}
a_{2,ij} {}=&  \frac{1}{2} K^{HS} (1+ \chi_{ij}) \epsilon_{ij} \mathcal{C}_{ij}^2 \lbrace x_{0,ij}^{2\lambda _{a,ij}} [a_{1,ij}^{s}(\rho _{s}; 2\lambda _{a,ij}) + B_{ij}(\rho _{s}; 2\lambda _{a,ij})] \\
& - 2x_{0,ij}^{2\lambda _{a,ij} + 2\lambda _{r,ij}} [a_{1,ij}^{s}(\rho _{s}; \lambda _{a,ij} + \lambda _{r,ij}) + B_{ij}(\rho _{s}; \lambda _{a,ij} + \lambda _{r,ij})] \\
& +  x_{0,ij}^{2\lambda _{r,ij}} [a_{1,ij}^{s}(\rho _{s}; 2\lambda _{r,ij}) + B_{ij}(\rho _{s}; 2\lambda _{r,ij})] \rbrace.
\end{aligned}
\label{eqn:a2ijap}
\end{equation}

In Eq. \ref{eqn:a2ijap}, $K^{HS}$ is the isothermal compressibility of the mixture of hard spheres. It is equal to

\begin{equation}
K^{HS} = \dfrac{(1-\zeta_{x})^{4}}{1+4 \zeta_{x}+4 \zeta_{x}^{2}+4 \zeta_{x}^{3}+\zeta_{x}^{4}},
\end{equation}
and $\chi_{ij}$ is given by

\begin{equation}
\chi_{ij} = f_{i}(\alpha_{ij}) \bar{\zeta}_{x} + f_{2}(\alpha_{ij}) \bar{\zeta}_{x}^{5} + f_{3}(\alpha_{ij}) \bar{\zeta}_{x}^{8},
\end{equation}
where $\bar{\zeta}_{x}$ is equal to
\begin{equation}
\bar{\zeta}_{x} = \frac{\pi \rho _{s}}{6} \sum_{j=1}^{n} x_{s,i} x_{s,j} \sigma_{ij}^{3},
\end{equation}

and $\alpha_{ij}$ is equal to

\begin{equation}
\alpha_{ij} = \mathcal{C}_{ij} \left(\frac{1}{\lambda_{a,ij}-3} - \frac{1}{\lambda_{r,ij}-3} \right).
\end{equation}

Finally, $a_{3}$  in Eq. \ref{eqn:aM} is equal to

\begin{equation}
a_{3} = \sum_{i=1}^{n} \sum_{j=1}^{n} x_{s,i} x_{s,j} a_{3,ij},
\end{equation}	
where $a_{3,ij}$ is equal to

\begin{equation}
a_{3,ij} = - \epsilon _{ij}^{3} f_{4}(\alpha_{ij}) \bar{\zeta}_{x} \exp[f_{5}(\alpha_{ij}) \bar{\zeta}_{x}+ f_{6}(\alpha_{ij}) \bar{\zeta}_{x}^{2}].
\end{equation}

The functions $f_{k}(k=1,...,6)$ are obtained with
\begin{equation}
f_{k}(\alpha_{ij}) \dfrac{\sum_{n=0}^{n=3} \phi_{k,n} \alpha_{ij}^{n}}{1+ \sum_{n=4}^{n=6} \phi_{k,n} \alpha_{ij}^{n-3}},
\end{equation}
where $\phi_{k,n}$ are coefficients defined in the original paper by \citeonline{lafitte2013}.

The density dependent coefficients of Eq. \ref{eqn:ghs} are given by the following equations:

\begin{equation}
k_{0} = - \ln(1-{\zeta}_{x}) + \frac{42{\zeta}_{x} -39{\zeta}_{x}^{2}+ 9{\zeta}_{x}^{3}-2{\zeta}_{x}^{4}}{6(1-\zeta_{x})^{3}},
\end{equation}

\begin{equation}
k_{1} = \frac{{\zeta}_{x}^{4} +6{\zeta}_{x}^{2}- 12{\zeta}_{x}}{2(1-\zeta_{x})^{3}},
\end{equation}

\begin{equation}
k_{2} = \frac{-3{\zeta}_{x}^{2}}{8(1-\zeta_{x})^{2}},
\end{equation}

\begin{equation}
k_{3} = \frac{-{\zeta}_{x}^{4}+3{\zeta}_{x}^{2}+3{\zeta}_{x}}{6(1-\zeta_{x})^{3}}.
\end{equation}

In Eq. \ref{eqn:gmie}, the term $g_{1,ij}(\sigma_{ij})$ is the first-order contribution to the radial distribution function. It has the following form:

\begin{equation}
\begin{aligned}
g_{1,ij}(\sigma_{ij}) {}=& \dfrac{1}{2 \pi \epsilon_{ij} d _{ij}^{3}} \left[ 3 \frac{\partial a_{1,ij}}{\partial \rho _{s}}   - \mathcal{C}_{ij} \lambda_{a,ij} x_{0,ij}^{\lambda_{a,ij}} \dfrac{a_{1,ij}^{s}(\rho_{s};\lambda_{a,ij})+B_{ij}(\rho_{s};\lambda_{a,ij})}{\rho _{s}} \right. \\
& \left.  + \mathcal{C}_{ij} \lambda_{r,ij} x_{0,ij}^{\lambda_{r,ij}} \dfrac{a_{1,ij}^{s}(\rho_{s};\lambda_{r,ij})+B_{ij}(\rho_{s};\lambda_{r,ij})}{\rho _{s}} \right].
\end{aligned}
\label{eq:g1saft}
\end{equation} 

Also in Eq. \ref{eqn:gmie}, the second-order contribution to the radial distribution function ($g_{2,ij}(\sigma_{ij})$) is equal to

\begin{equation}
g_{2,ij}(\sigma_{ij}) = (1 + \gamma_{c,ij}) g_{2,ij}^{MCA}(\sigma_{ij}), 
\end{equation}
where $g_{2,ij}^{MCA}(\sigma_{ij})$ is equal to

\begin{equation}
\begin{aligned}
g_{2,ij}(\sigma_{ij}) {}=& \dfrac{1}{2 \pi \epsilon_{ij} d _{ij}^{3}} \left[3 \frac{\partial \frac{a_{2}}{1+\gamma_{ij}}}{\partial \rho _{s}} - \epsilon_{ij} K^{HS} \mathcal{C}_{ij}^{2} \lambda_{r,ij} x_{0,ij}^{2\lambda_{r,ij}} \dfrac{a_{1,ij}^{s}(\rho_{s};2\lambda_{r,ij})+B_{ij}(\rho_{s};2\lambda_{r,ij})}{\rho _{s}} \right. \\
& \left. + \epsilon_{ij} K^{HS} \mathcal{C}_{ij}^{2} (\lambda_{r,ij}+\lambda_{a,ij}) (x_{0,ij})^{\lambda_{r,ij}+\lambda_{a,ij}} \dfrac{a_{1,ij}^{s}(\rho_{s};\lambda_{r,ij}+\lambda_{a,ij})+B_{ij}(\rho_{s};\lambda_{r,ij}+\lambda_{a,ij})}{\rho _{s}} \right. \\
& \left. - \epsilon_{ij} K^{HS} \mathcal{C}_{ij}^{2} \lambda_{a,ij} x_{0,ij}^{2\lambda_{a,ij}} \dfrac{a_{1,ij}^{s}(\rho_{s};2\lambda_{a,ij})+B_{ij}(\rho_{s};2\lambda_{a,ij})}{\rho _{s}}\right],
\end{aligned}
\label{eq:g2saft}
\end{equation} 
and $\gamma_{c,ij}$ is a correction factor obtained from the equation bellow:

\begin{equation}
\gamma_{c,ij} = \phi_{7,0} \bar{\zeta}_{x} \theta_{ij} \exp(\phi_{7,3} \bar{\zeta}_{x} + \phi_{7,4} \bar{\zeta}_{x}^2) \lbrace 1-\tanh [\phi_{7,1}(\phi_{7,2} - \alpha _{ij})] \rbrace,
\end{equation}
where $\theta _{ij} = \exp (\beta \epsilon_{ij}) -1$.
