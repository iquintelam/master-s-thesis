% Chapter Template

\chapter{SAFT-$\gamma$ Mie Force Field} % Main chapter title

\label{ChapterX} % Change X to a consecutive number; for referencing this chapter elsewhere, use \ref{ChapterX}

%----------------------------------------------------------------------------------------
%	SECTION 1
%----------------------------------------------------------------------------------------

\section{SAFT-VR Mie}

The SAFT-VR Mie equation of state \cite{lafitte2013} is the basis for the SAFT-$\gamma$ Mie coarse grained force field \cite{avendano2011}. This EoS was initially developed to describe chain molecule formed from fused Mie segments using the Mie attractive and repulsive potential. The Mie potential is a type of generalized Lennard-Jones potential that can be used to describe explicitly repulsive interactions of different hardness/softness and attractive interactions of different ranges, and is given by:
\begin{equation}
U_{Mie}(r) = \epsilon\frac{\lambda_r}{\lambda_r - \lambda_a} \left(\frac{\lambda_r}{\lambda_a} \right)^{\left( \frac{\lambda_a}{\lambda_r - \lambda_a} \right)}
\left[ \left(\frac{\sigma}{r} \right)^{\lambda_r} - \left(\frac{\sigma}{r} \right)^{\lambda_a} \right]
\label{eqn:miepotential}
\end{equation}
where $\epsilon$ is the potential well depth, $\sigma$ is the segment diameter, r is the distance between the spherical segments, $\lambda_r$ is the repulsive exponent and $\lambda_s$ is the attractive exponent. This equation uses the \citeonline{bh1976} high perturbation expansion of the Helmholtz free energy up to third order and an improved expression for the  radial distribution function (RDF) of Mie monomers at contact to obtain a equation capable to give an accurate theoretical description of the vapor-liquid equilibria and second derivative properties \cite{lafitte2013}. For a non-associating fluid, the Helmholtz free energy is:
\begin{equation}
\frac{A}{N\kappa_{b}T} = a = a^{IDEAL} + a^{MONO} + a^{CHAIN}
\label{eqn:miehelm}
\end{equation}

%	SUBSECTION 1
%-----------------------------------
\subsection{Ideal Contribution}

The ideal contribution for a mixture is given by:
\begin{equation}
a^{IDEAL} = \sum_{i=1}^{N_{c}} x_{i}\ln{(\rho_{i}{\Lambda_{i}}^3)} -1
\label{eqn:aideal}
\end{equation}
where $x_{i}=N_{i}/N$ is the molar fraction of component i, $\rho_{i}=N_{i}/V$ is the number density, $N_{i}$ is the number of molecules of each component and $\Lambda_{i}^3$ is de Broglie wavelength. 
%-----------------------------------
%	SUBSECTION 2
%-----------------------------------

\subsection{Monomer Contribution}
The monomer contribution describes the interactions between Mie segments and can be expressed for a mixture as:
\begin{equation}
a^{MONO} = \left(\sum_{i=1}^{N_{c}} x_{i}m_{s,i} \right)a^{M}
\label{eqn:amonomer}
\end{equation}
\par
In the equation above, $m_{s,i}$ is the number of spherical segments making up the molecule i and $a^{M}$  is the monomer dimensionless Helmholtz free energy and it is expressed as a third order perturbation expansion in the inverse temperature \cite{bh1976}:
\begin{equation}
a^{M} = a^{HS}+\beta{a_{1}}+\beta{a_{2}}^2+\beta{a_{3}}^3 
\label{eqn:aM}
\end{equation}
where $\beta=\kappa_{b}T$ and $a^{HS}$ is the hard-sphere dimensionless Helmholtz free energy for a mixture :
\begin{equation}
a^{HS} = \frac{6}{\pi\rho_{s}}\left[\left(\frac{\zeta^3_2}{\zeta^2_3}-\zeta_0 \right)\ln(1-\zeta_3)+\frac{3\zeta_{1}\zeta_{2}}{1-\zeta_3}+ \frac{\zeta^3_2}{\zeta_{3}(1-\zeta_3)^2}\right]
\label{eqn:hs}
\end{equation}
where $\rho_{s}=\rho\sum_{i}^{N_c} x_{i}m{s,i}$ is the total number density of spherical segments and $\zeta_l$ are the moments of the number density:
\begin{equation}
\zeta_l = \frac{\pi\rho_s}{6}\left(\sum_{i=1}^{N_c} x_{s,i}d^l_{ii} \right), l = 0,1,2,3
\label{eqn:zetal}
\end{equation}
where $x_{s,i}$ is the mole fraction of the segments and is related through the mole fraction of component i ($x_i$) by:
\begin{equation}
x_{s,i} = \frac{m_{s,i}x_i}{\sum_{k=1}^{N_c} m_{s,k}x_{k} }
\label{eqn:xsi}
\end{equation}
\par
The effective hard-sphere diameter $d_{ii}$ for the segments is:
\begin{equation}
d_{ii} =\int_{0}^{\sigma_{ii}} ( 1 - \exp(-\beta U^{Mie}_{ii}(r)) ) dr
\label{eqn:diameter}
\end{equation}
\par
The integral in Eq. \eqref{eqn:diameter} is normally obtained by means of Gauss-Legendre with a 5-point quadrature \cite{papa2014}. The detailing of the terms of Eq. \eqref{eqn:amonomer} can be found in \citeonline{lafitte2013}.
\subsection{Chain Contribution}
The chain formation of $m_{s}$ tangentially bonded Mie segments contribution is based on the first-order pertubation theory (TPT1)  \cite{papa2014} and can be expressed as:
\begin{equation}
a^{CHAIN} =-\sum_{i=1}^{N_{c}} x_{i}(m_{s,i} - 1)\ln(g_{ii}^{Mie}(\sigma_{ii}))
\label{eqn:achain}
\end{equation}
\par
The $g_{ij}^{Mie}(\sigma_{ij})$ term correspond to the value of the radial distribution function (RDF) of the hypothetical Mie system evaluated at the effective diameter and can be obtained with the perturbation expansion:
\begin{equation}
g_{ij}^{Mie}(\sigma_{ij}) =g_{d,ij}^{HS}(\sigma_{ij})\exp[\beta\epsilon g_{1,ij}(\sigma_{ij})/g_{d,ij}^{HS}(\sigma_{ij}) + (\beta\epsilon)^{2} g_{2,ij}(\sigma_{ij})/g_{d,ij}^{HS}(\sigma_{ij})]
\label{eqn:achain}
\end{equation}
\par
The terms in the equations above are explicitly exposed in the original article \cite{lafitte2013}. 
\subsection{Ring Contribution}
There are two forms for the Helmholtz free energy for rings formed from $m_{s}$ tangentially bonded segments in the literature. The first one  \cite{lafitte2012} considered that the difference between a chain and a ring molecule is that the latter one has one more bond that is connecting the first segment to the last. With this assumption, the Eq. \eqref{eqn:achain} becomes:
\begin{equation}
a^{RING} =-\sum_{i=1}^{N_{c}} x_{i}m_{s,i}\ln(g_{ii}^{Mie}(\sigma_{ii}))
\label{eqn:aringlafitte}
\end{equation}
\par
According to \citeonline{lafitte2012}, Eq. \eqref{eqn:aringlafitte} needs an additional parametrization with molecular simulation data so the EoS can  be used in molecular simulations, but this procedure is not the necessary for ring molecules. Recently \citeonline{muller2017} tried to correct this inconsistency developing the ring free energy based on the work of \citeonline{muller1993} whom obtained rigorous expressions for molecular geometries of rings of $m_s=3$ for hard fluids. The final expression for the dimensionless Helmholtz free energy is:
\begin{equation}
a^{RING} =-\sum_{i=1}^{N_{c}} x_{i}(m_{s,i}-1+\chi_{i}\eta_{i})\ln(g_{ii}^{Mie}(\sigma_{ii}))
\label{eqn:aringmuller}
\end{equation}
where $\eta_{i}=m_{s,i}\rho_{i}\sigma_{ii}^{3}/6$ is the packing fraction and $\chi_{i}$ is a parameter which depends on $m_{s,i}$ and the geometry of the ring of each component i. For a value of $\chi=0$ Eq. \eqref{eqn:aringmuller} is equal to Eq. \eqref{eqn:achain} and $\chi=1.3827$ corresponds to a hard sphere system of triangles. \citeonline{muller2017} also calculated values of $\zeta$ for the Saft-VR Mie EoS for the values of $m_{s}=3,m_{s}=4,m_{s}=5,m_{s}=7$ with pseudo-experimental data from molecular dynamics (MD) for a defined pure fluid. The values of $\chi$ estimated can be seen in the figure below:
\begin{figure}[th]
\centering
\includegraphics[scale=0.8]{Figures/mullergeo.jpg}
\caption{Values for parameter $\chi$ according to the ring geometry \cite{muller2017}}
\label{fig:ringqsi}
\end{figure}
\subsection{Combining rules for the intermolecular potential parameters}
\citeonline{lafitte2013} also suggested mixing rules for the potential parameters based on Lorentz-Berthelot combining rules \cite{rowlinson}:
\begin{equation}
\sigma_{ij} =\frac{\sigma{ii}+\sigma{jj}}{2}
\label{eqn:sigmamix}
\end{equation}
\begin{equation}
\lambda_{k,ij} -3 =\sqrt{(\lambda_{k,ii}-3)(\lambda{k,jj}-3)} , k=r,a
\label{eqn:lambdamix}
\end{equation}
\begin{equation}
\epsilon_{ij} =(1-k_{ij})\frac{\sqrt{\sigma_{ii}^{3}\sigma_{jj}^{3}}}{\sigma_{ij}^{3}}\sqrt{\epsilon_{ii}\epsilon_{jj}}
\label{eqn:epsmix}
\end{equation}
\par
The $k_{ij}$ is a binary interaction parameter to correct the deviations of the Lorentz-Berthelot rule for chemically distinct compounds. This parameter can also be fitted to experimental data or pseudo experimental data.
%----------------------------------------------------------------------------------------
%	SECTION 2
%----------------------------------------------------------------------------------------

\section{Parameter Estimation for the SAFT-$\gamma$ Mie Force Field}

The SAFT-$\gamma$ Mie Force Field is force field that uses a  coarse graining top down methodology in your parameterization so the intermolecular parameters are obtained from macroscopic experimental data like fluid-phase equilibrium or superficial tension data. The idea is that the force field's  parameters for the molecular simulation can  be estimated with the equation of state can be used on molecular simulations since both the Saft-VR Mie EoS and the force field use the Mie Potential.
