
\chapter{Fundamentals of the Computational Methods} % Main chapter title

\label{Chapter3} % Change X to a consecutive number; for referencing this chapter elsewhere, use \ref{ChapterX}


\section{SAFT-$\gamma$ Mie Force Field}

\subsection{SAFT-VR Mie EoS}

The SAFT-VR Mie equation of state \cite{lafitte2013} is the basis for the SAFT-$\gamma$ Mie coarse grained force field \cite{avendano2011}. This EoS was initially developed to describe chain molecule formed from fused Mie segments using the Mie attractive and repulsive potential. The Mie potential is a type of generalized Lennard-Jones potential that can be used to explicitly describe repulsive interactions of different hardness/softness and attractive interactions of different ranges, and is given by:
\begin{equation}
U_{Mie}(r) = \epsilon\frac{\lambda_r}{\lambda_r - \lambda_a} \left(\frac{\lambda_r}{\lambda_a} \right)^{\left( \frac{\lambda_a}{\lambda_r - \lambda_a} \right)}
\left[ \left(\frac{\sigma}{r} \right)^{\lambda_r} - \left(\frac{\sigma}{r} \right)^{\lambda_a} \right]
\label{eqn:miepotential}
\end{equation}
where $\epsilon$ is the potential well depth, $\sigma$ is the segment diameter, r is the distance between spherical segments, $\lambda_r$ is the repulsive exponent and $\lambda_a$ is the attractive exponent. This equation uses the \citeonline{bh1976} high perturbation expansion of the Helmholtz free energy up to third order and an improved expression for the  radial distribution function (RDF) of Mie monomers at contact to obtain an equation able to give an accurate theoretical description of the vapor-liquid equilibrium and second derivative properties \cite{lafitte2013}. For a non-associating fluid, the Helmholtz free energy is:
\begin{equation}
\frac{A}{N\kappa_{b}T} = a = a^{IDEAL} + a^{MONO} + a^{CHAIN}
\label{eqn:miehelm}
\end{equation}

\subsubsection{Ideal Contribution}

The ideal contribution for a mixture is given by:
\begin{equation}
a^{IDEAL} = \sum_{i=1}^{N_{c}} x_{i}\ln{(\rho_{i}{\Lambda_{i}}^3)} -1
\label{eqn:aideal}
\end{equation}
where $x_{i}=N_{i}/N$ is the molar fraction of component i, $\rho_{i}=N_{i}/V$ is the number density, $N_{i}$ is the number of molecules of each component and $\Lambda_{i}^3$ is the de Broglie wavelength. 

\subsubsection{Monomer Contribution}

The monomer contribution describes interactions between Mie segments and can be expressed, for a mixture, as:
\begin{equation}
a^{MONO} = \left(\sum_{i=1}^{N_{c}} x_{i}m_{s,i} \right)a^{M}
\label{eqn:amonomer}
\end{equation}

In the equation above, $m_{s,i}$ is the number of spherical segments making up the molecule i and $a^{M}$  is the monomer dimensionless Helmholtz free energy and it is expressed as a third order perturbation expansion in the inverse temperature \cite{bh1976}:
\begin{equation}
a^{M} = a^{HS}+\beta{a_{1}}+\beta{a_{2}}^2+\beta{a_{3}}^3 
\label{eqn:aM}
\end{equation}
where $\beta=\kappa_{b}T$ and $a^{HS}$ is the hard-sphere dimensionless Helmholtz free energy for a mixture :
\begin{equation}
a^{HS} = \frac{6}{\pi\rho_{s}}\left[\left(\frac{\zeta^3_2}{\zeta^2_3}-\zeta_0 \right)\ln(1-\zeta_3)+\frac{3\zeta_{1}\zeta_{2}}{1-\zeta_3}+ \frac{\zeta^3_2}{\zeta_{3}(1-\zeta_3)^2}\right]
\label{eqn:hs}
\end{equation}

The variable $\rho_{s}=\rho\sum_{i}^{N_c} x_{i}m{s,i}$ is the total number density of spherical segments and $\zeta_l$ are the moments of the number density:
\begin{equation}
\zeta_l = \frac{\pi\rho_s}{6}\left(\sum_{i=1}^{N_c} x_{s,i}d^l_{ii} \right), l = 0,1,2,3
\label{eqn:zetal}
\end{equation}
where $x_{s,i}$ is the mole fraction of segments and is related through the mole fraction of component i ($x_i$) by:
\begin{equation}
x_{s,i} = \frac{m_{s,i}x_i}{\sum_{k=1}^{N_c} m_{s,k}x_{k} }
\label{eqn:xsi}
\end{equation}


The effective hard-sphere diameter $d_{ii}$ for the segments is:
\begin{equation}
d_{ii} =\int_{0}^{\sigma_{ii}} ( 1 - \exp(-\beta U^{Mie}_{ii}(r)) ) dr
\label{eqn:diameter}
\end{equation}


The integral in Eq. \eqref{eqn:diameter} is normally obtained by means of Gauss-Legendre with a 5-point quadrature \cite{papa2014}. The detailing of terms of Eq. \eqref{eqn:amonomer} can be found in \citeonline{lafitte2013}.

\subsubsection{Chain Contribution}
The chain formation of $m_{s}$ tangentially bonded Mie segments contribution is based on the first-order perturbation theory (TPT1)  \cite{papa2014} and can be expressed as:
\begin{equation}
a^{CHAIN} =-\sum_{i=1}^{N_{c}} x_{i}(m_{s,i} - 1)\ln(g_{ii}^{Mie}(\sigma_{ii}))
\label{eqn:achain}
\end{equation}


The $g_{ij}^{Mie}(\sigma_{ij})$ term correspond to the radial distribution function (RDF) of the hypothetical Mie system evaluated at the effective diameter and can be obtained with the perturbation expansion:
\begin{equation}
\begin{aligned}
g_{ij}^{Mie}(\sigma_{ij}) =g_{d,ij}^{HS}(\sigma_{ij})\exp[\beta\epsilon g_{1,ij}(\sigma_{ij})/g_{d,ij}^{HS}(\sigma_{ij}) + (\beta\epsilon)^{2} g_{2,ij}(\sigma_{ij})/g_{d,ij}^{HS}(\sigma_{ij})]
\end{aligned}
\label{eqn:gmie}
\end{equation}


The other terms in the equations above are explicitly exposed in the original article \cite{lafitte2013}. 

\subsubsection{Ring Contribution}
There are two forms for the Helmholtz free energy for rings formed from $m_{s}$ tangentially bonded segments in the literature. The first one  \cite{lafitte2012} considered that the difference between a chain and a ring molecule is that the latter has one more bond that is connecting the first segment to the last. With this assumption, Eq. \eqref{eqn:achain} can be adapted to rings by:
\begin{equation}
a^{RING} =-\sum_{i=1}^{N_{c}} x_{i}m_{s,i}\ln(g_{ii}^{Mie}(\sigma_{ii}))
\label{eqn:aringlafitte}
\end{equation}

According to \citeonline{lafitte2012}, Eq. \eqref{eqn:aringlafitte} needs an additional parameterization with molecular simulation data so the EoS can  be used in molecular simulations, but this procedure is not necessary for chain molecules. Recently, \citeonline{muller2017} tried to correct this inconsistency by means of developing the ring free energy based on the work of \citeonline{muller1993}, who obtained rigorous expressions for hard fluids with molecular geometries of rings of $m_s=3$. The final expression developed for the ring dimensionless Helmholtz free energy is:
\begin{equation}
a^{RING} =-\sum_{i=1}^{N_{c}} x_{i}(m_{s,i}-1+\chi_{i}\eta_{i})\ln(g_{ii}^{Mie}(\sigma_{ii}))
\label{eqn:aringmuller}
\end{equation}

$\eta_{i}=m_{s,i}\rho_{i}\sigma_{ii}^{3}/6$ is the packing fraction and $\chi_{i}$ is a parameter which depends on $m_{s,i}$ and on the geometry of the ring of each component i. For a value of $\chi=0$, Eq. \eqref{eqn:aringmuller} is equal to Eq. \eqref{eqn:achain}. Meanwhile, the equation corresponds to a hard sphere system of triangles when $\chi=1.3827$. \citeonline{muller2017} also calculated values of $\zeta$ for $m_{s}=3,m_{s}=4,m_{s}=5,m_{s}=7$ with pseudo-experimental data from molecular dynamics (MD) for a defined pure fluid. The values of $\chi$ for each geometry estimated can be seen in \figref{ringqsi}.
\begin{figure}[th]
	\centering
	\includegraphics[scale=0.8]{Figures/mullergeo.jpg}
	\caption{Values for parameter $\chi$ according to the ring geometry \cite{muller2017}}
	\label{ringqsi}
\end{figure}

\subsubsection{Combining rules for the intermolecular potential parameters}
\citeonline{lafitte2013} also suggested mixing rules for the potential parameters based on Lorentz-Berthelot combining rules \cite{rowlinson}:
\begin{equation}
\sigma_{ij} =\frac{\sigma{ii}+\sigma{jj}}{2}
\label{eqn:sigmamix}
\end{equation}
\begin{equation}
\lambda_{k,ij} -3 =\sqrt{(\lambda_{k,ii}-3)(\lambda_{k,jj}-3)} , k=r,a
\label{eqn:lambdamix}
\end{equation}
\begin{equation}
\epsilon_{ij} =(1-k_{ij})\frac{\sqrt{\sigma_{ii}^{3}\sigma_{jj}^{3}}}{\sigma_{ij}^{3}}\sqrt{\epsilon_{ii}\epsilon_{jj}}
\label{eqn:epsmix}
\end{equation}

The $k_{ij}$ is a binary interaction parameter to correct the deviations of the Lorentz-Berthelot rule for chemically distinct compounds. This parameter can be fitted to experimental data or pseudo experimental data.


\subsection{Parameter Estimation for the SAFT-$\gamma$ Mie Force Field}\label{parsaft}

The SAFT-$\gamma$ Mie Force Field uses a coarse graining top down methodology in its parameterization. This methodology aims to obtain the intermolecular parameters from macroscopic experimental data such as fluid-phase equilibrium or interfacial tension data. The idea is that the force field  parameters estimated with the SAFT-VR Mie EoS can be used on molecular simulations since both the equation of state and the force field use the same explicit intermolecular potential model (Mie potential). This correspondence between models has already been seen for a variety of fluids in which this force field was parameterized. This success in the representation of the properties of real fluids can be imputed to the degrees of freedom of Mie Potential \cite{herdes2015}. Furthermore, this flexibility provides the exploration of a very large parameter space without using an iterative simulation scheme \cite{avendano2011}. 

Each substance has initially five parameters to be estimated ($m_s$,$\sigma$,$\epsilon$,$\lambda_{r}$ and $\lambda_{a}$) according to Eq. \eqref{eqn:miepotential}. The number of segments are usually fixed in an integer value since each segment represents one pseudo atom. The attractive parameter is generally  fixed due to its  high correlation with the repulsive parameter. Usually, the parameter is fixed in the London value of 6, which is a good representation of the dispersion scale of most simple fluids that don't have strong polar interactions \cite{ramrattan2015,herdes2015}. There are two strategies to obtain the parameters: one is by fitting the Saft-Vr Mie EoS to experimental data as vapor pressure and liquid density and the other one is using correspondent state parametrization. The first, generally, minimizes the following unweighted least-squares objective function:

\begin{equation}
\begin{aligned}
\min\limits_{\sigma,\epsilon,\lambda_{r}} F_{obj}(\sigma,\epsilon,\lambda_{r})= \sum_{i=1}^{N_{p}} \left(\frac{P_{v}^{SAFT}(T_{i},\sigma,\epsilon,\lambda_{r})-P_{v}^{exp}(T_{i})}{P_{v}^{exp}(T_{i})} \right)^2 +\\
\sum_{i=1}^{N_{p}} \left(\frac{\rho_{l}^{SAFT}(T_{i},\sigma,\epsilon,\lambda_{r})-\rho_{l}^{exp}(T_{i})}{\rho_{l}^{exp}(T_{i})} \right)^2
\end{aligned}
\label{eqn:fobj}
\end{equation}
where $N_{p}$ is the number of experimental points, $P_{v}$ is the vapor pressure and $\rho_{l}$ is the saturated liquid density. Another properties that can be used in the estimation are superficial tension and speed of sound. The multiple parameters of the model make it necessary the use of a wide range of experimental data since multiple solutions may be found for the fit. Therefore, one needs to be careful in deciding the level of coarse graining (i.e. the parameter $m_{s}$) and the subsequent parameter space that will not result in some physical inconsistencies such as a premature freezing fluid.

\citeonline{lafitte2012} suggested that two corrections factors ($c_{\sigma}$ and $c_{\epsilon}$) should be estimated with simulation data when using Eq. \eqref{eqn:aringlafitte} for the ring contribution. They are related to the EoS parameters by scaled parameters:

\begin{equation}
\sigma^{scaled} = c_{\sigma}\sigma^{SAFT}
\label{eqn:csigma}
\end{equation}
\begin{equation}
\epsilon^{scaled} = c_{\epsilon}\epsilon^{SAFT}
\label{eqn:ceps}
\end{equation}

According to \citeonline{lafitte2012}, these corrections are necessary because the approximations employed in the EoS theory generate discrepancies between molecular simulations and the EoS for ring molecules modeled with Eq. \eqref{eqn:aringlafitte}. The objective function for this estimation is given by:

\begin{equation}
\begin{split}
\min\limits_{c_{\sigma},c_{\epsilon}} F_{obj}(c_{\sigma},c_{\epsilon})= \sum_{i=1}^{N_{p}} \left(\frac{P_{v}^{sim}(T_{i},\sigma^{SAFT},\epsilon^{SAFT})-P_{v}^{SAFT}(T_{i},\sigma^{scaled},\epsilon^{scaled})}{P_{v}^{sim}(T_{i},\sigma^{SAFT},\epsilon^{SAFT})} \right)^2 + \\
\sum_{i=1}^{N_{p}} \left(\frac{\rho_{liq}^{sim}(T_{i},\sigma^{SAFT},\epsilon^{SAFT})-\rho_{liq}^{SAFT}(T_{i},\sigma^{scaled},\epsilon^{scaled})}{\rho_{liq}^{sim}(T_{i},\sigma^{SAFT},\epsilon^{SAFT})} \right)^2
\end{split}
\label{eqn:fobjla}
\end{equation}

The repulsive parameter is maintained in the value found on the minimization of Eq. \eqref{eqn:fobj}. The refined values for $\sigma$ and $\epsilon$ are:

\begin{equation}
\sigma^{sim} = \sigma^{SAFT}/c_{\sigma}
\label{eqn:simsigma}
\end{equation}

\begin{equation}
\epsilon^{sim} = \epsilon^{SAFT}/c_{\epsilon}
\label{eqn:simeps}
\end{equation}

It is interesting to point out that this new parameterization is not necessary when using Eq. \eqref{eqn:aringmuller} as the ring contribution. The other method to obtain the force field parameters is the correspondent state parametrization \cite{mejia2014}. This method considers that the unweighted volume average of the attractive contribution to the Mie intermolecular potential,$a_{1}$, is a mean field approximation:

\begin{equation}
a_{1} = 2\pi\rho\sigma^{3}\epsilon\alpha
\label{eqn:a1corres}
\end{equation}

The van der Waals constant, $\alpha$, considering $ \lambda_{a} = 6$ is related by the Mie exponents by:

\begin{equation}
\alpha = \frac{1}{\epsilon\sigma^{3	}} \int_{\sigma}^{\infty} \phi(r)r^{2}dr = \frac{\lambda_{r}}{3(\lambda_{r}-3)}\left(\frac{\lambda_r}{6}\right)^{6/(\lambda_{r} - 6)}  
\label{eqn:alpha}
\end{equation}

The parameterization in this method starts by using the experimental acentric factor, $\omega$, for each molecule with a fixed value of $ m_{s}$ to obtain the value of the repulsive exponent with the following Padé series:

\begin{equation}
\lambda_{r} = \frac{\sum_{i=0} a_{i}\omega^{i}}{1+\sum_{i=1} b_{i}\omega^{i}}   
\label{eqn:lambdaco}
\end{equation}

$a_{i}$ and $b_{i}$ are dependent parameters of the number of segments and a table with their values is presented in the original paper \cite{mejia2014}. The van der Waals constant can be found substituting $\lambda_{r}$ into Eq. \eqref{eqn:alpha}. The reduced critical potential $T_{c}^{*}$ is related to $\alpha$ by a Padé series: 

\begin{equation}
T_{c}^{*} = \frac{\sum_{i=0} c_{i}\alpha^{i}}{1+\sum_{i=1} d_{i}\alpha^{i}}   
\label{eqn:tc}
\end{equation}

The values of $c_{i}$ and $d_{i}$ are also available in the original paper. The reduced temperature of the equation above is used in conjunction with the experimental critical temperature, $ T_{c}$, to find the energy parameter with the relation below:

\begin{equation}
T_{c}^{*} = \frac{\kappa_{b}T_{c}}{\epsilon}   
\label{eqn:epscorre}
\end{equation}

The diameter parameter, however, is not obtained with the critical properties, but with the reduced liquid density,$\rho_{T_{r}=0.7}$, at the reduced temperature ,$T_{r}$, of $0.7$. This density is also obtained with a Padé series using parameters by \citeonline{mejia2014}:

\begin{equation}
\rho_{T_{r}=0.7}^{*} = \frac{\sum_{i=0} j_{i}\alpha^{i}}{1+\sum_{i=1} k_{i}\alpha^{i}} 
\label{eqn:denscorre}
\end{equation}

The relation among the equation above, $\sigma$ and the experimental density is given by:

\begin{equation}
\rho_{T_{r}=0.7}^{*} = \rho_{T_{r}=0.7}\sigma^{3}N_{av}   
\label{eqn:sigmacorre}
\end{equation}
where $N_{av}$ is The Avogadro number. This correspondent state method has the advantage of only requiring critical data, which is available for a great range of fluids, and liquid density data. The parameters found with this strategy are available at an online database \cite{ervik2016}.     

The binary interaction parameter $k_{ij}$ of Eq. \eqref{eqn:epsmix} is necessary to adjust the mixture behavior of chemically distinct components. Normally, it is estimated minimizing the difference between experimental binary vapor liquid equilibrium or interfacial tension data and the SAFT-VR Mie EoS output data \cite{muller2017,lobanova2016}. The objective function is similar to: 

\begin{equation}
\begin{aligned}
\min\limits_{k_{ij}} F_{obj}(k_{ij})= \sum_{k=1}^{N_{p}} \left(\frac{P_{v}^{SAFT}(T_{k},x,k_{ij})-P_{v}^{exp}(T_{k},x)}{P_{v}^{exp}(T_{k},x)} \right)^2 +\\
\sum_{k=1}^{N_{p}} \left(\frac{\rho_{l}^{SAFT}(T_{k},x,k_{ij})-\rho_{l}^{exp}(T_{i})}{\rho_{l}^{exp}(T_{i})} \right)^2
\end{aligned}
\label{eqn:fobjmix}
\end{equation}

However, \citeonline{ervik20162} used molecular simulation results to fit the parameter to the superficial tension data. The strategy followed by them was to do simulations in three values of $k_{ij}$ first and, after, they refined the parameter until a value in good agreement with the experimental data was found. 

\section{Expanded Ensemble Method}\label{ee}

Instead of doing various simulations in different values of $\lambda$, expanded ensemble simulations \cite{lyubartsev} were developed to allow a non-Boltzmann sampling scheme of different states in only one simulation. The statistical expanded ensemble, $Z^{EE}$ can be defined as a sum of sub ensembles $Z_{i}$ in different values of $\lambda$:

\begin{equation}
Z^{EE} = \sum_{i=1}^{N} Z_{i}(\lambda_{i}) exp(\eta_{i})
\label{eqn:ee}
\end{equation}   
where N is the number of alchemical states and $\eta_{i}$ is the arbitrary weight of the sub ensemble $Z_{i}$ at each state. In solvation energy calculations with molecular dynamics, $\lambda$ corresponds to the coupling parameter of the soft-core potential (Eq. \ref{eq:softcore}) and the expanded ensemble is sampled by performing an arbitrary number of MD  steps followed by a $\lambda$ transition. \citeonline{chodera2011} proved that the sampling of the expanded ensemble is similar to the Gibbs sampling method \cite{geman1984,liu2002}. Following the Gibbs method, the sampling of the configuration space x for one state $\lambda_{k}$ during the MD steps is done by using the conditional distribution:

\begin{equation}
\pi(x|\lambda_{k}) = \dfrac{\exp[-\beta u(x,\lambda_{k})]}{\int dx \exp [- \beta u(x,\lambda_{k})]}
\label{eqn:rhoee1}
\end{equation} 

Meanwhile, the state transition in the MD simulation uses the following conditional distribution:

\begin{equation}
\pi(\lambda_{k}|x) = \dfrac{\exp[-\beta u(x,\lambda_{k}) + \eta_{k}]}{ \sum_{k=1}^{K} \exp [- \beta u(x,\lambda_{k})+ \eta_{k}]}
\label{eqn:rhoee2}
\end{equation} 
where $u(x,\lambda_{k}) = U(x,\lambda_{k}) + PV(x,\lambda_{k})$ is the reduced potential function for the NPT ensemble. There are a variety of acceptance schemes to do the expanded sampling using Eq. \eqref{eqn:rhoee2}, but \citeonline{chodera2011} suggested that the independence sampling \cite{liu2002} is the best strategy to increase the number of uncorrelated configurations. The implementation suggested by them updates the state index from $i$ to $j$ by first generating a uniform random number $R$ on the interval $[0,1)$ and then selecting the smallest new value of $j$ that satisfies  the relation below:

\begin{equation}
R < \sum_{i=1}^{j} \pi(\lambda_{i}|x) 
\label{eqn:relee2}
\end{equation} 

The sampling strategy above depends on the weights in order to assure an adequate sampling of the states. If there isn't a sufficient number of states sampled, the expanded ensemble becomes deficient in obtaining input data to estimate free energy differences with the methods exposed in Chapter 2. The weights can be calculated following the flat-histogram approach \cite{bernd1992,bernd1993,dayal2004}. This strategy aims to obtain adequate sampling by assuring that all the states have an equal number of samples, i.e. the ratio of the probability of sampling state ($\pi_{i}$) to the probability of sampling state $j$ ($\pi_{j}$) is equal to one. Given that $\pi_{i}$ has the following equation:

\begin{equation}
\pi_{i} = \dfrac{Z_{i}(\lambda_{i}) exp(\eta_{i})}{Z^{EE}} 
\label{eqn:wei1}
\end{equation} 
and using Eqs. \ref{eq:dif} and \ref {eq:partiso}, the following relation can be obtained for $\pi_{i}/\pi_{j}=1$:

\begin{equation}
(\eta_{i} - \eta_{j}) = \beta(G_i-G_j)
\label{eqn:weight}
\end{equation}

Eq. \eqref{eqn:weight} is solved iteratively with trial simulations. For the first simulation, the values of $\eta$ are chosen or set to zero and the histogram of the states visited is obtained. With this histogram, it is possible to estimate the free energy differences and, since the weights are related to the free energies by Eq. \eqref{eqn:weight}, the next values of $\eta$ can be calculated. This iteration goes on until a uniform distribution is secured. The weights found are then used in a longer simulation to obtain the final solvation free energy differences.

The choice of the $\lambda$ set correspondent to overlapping alchemical states are crucial to acquire accurate energy differences. In this work, the method chosen to obtain the optimal stage of the $\lambda$ domain is the one developed by \citeonline{escobedo2007} with basis in the study of  \citeonline{trebst2004}. The method optimizes $\lambda$ through the minimization of the number of round trips per CPU time between the lowest $\lambda$ ($=0$) and highest $\lambda$ ($=1$). This is done by maximizing the steady-state stream $\phi$ of the simulation, which "walks" among the values of $\lambda$. This stream is estimated from Fick's diffusion type of law:

\begin{equation}
\phi = D(\Lambda) \Pi (\Lambda) \dfrac{dx(\Lambda)}{d \Lambda}
\label{eqn:stream}
\end{equation}

In the equation above, $\Lambda$ is the actual continue value of the coupling parameter. This continue function of $\lambda s$ may be obtained by interpolating them linearly. $D(\Lambda)$ is the diffusivity at  state $\Lambda$ and $x(\Lambda)$ is the fraction of times that the trial simulation at state $\Lambda_{i}$ has most recently visited the state $\lambda=1$ as opposed to state $\lambda=0$. The derivative ${dx(\Lambda)}/{d \Lambda}$ can be approximated with the central finite differences method. Finally, $\Pi (\Lambda)$ is the probability of visiting $\Lambda$:

\begin{equation}
\Pi (\Lambda) = \dfrac{C^{'} \bar{\Pi} (\lambda)}{\Lambda_{i+1} - \Lambda_{i}}
\label{eqn:plambda}
\end{equation}

The $C^{'} $ term in the equation above represents a constant and $\bar{\Pi} (\lambda)$ is the arithmetic average of visiting the $\Lambda$ states:

\begin{equation}
\bar{\Pi} (\lambda) = \dfrac{\pi_{i+1} - \pi_{i}}{2}
\label{eqn:barplambda}
\end{equation}

The $\phi$ is maximum when the probability $\Pi^{'}(\Lambda_{i})$ is proportional to $1/\sqrt{D(\Lambda)}$. With that information, it is possible to estimate the diffusivity using one trial simulation with the following equation:

\begin{equation}
D(\Lambda) = \dfrac{\Lambda_{i+1} - \Lambda_{i}}{\bar{\Pi} (\lambda) {dx(\Lambda)}/{d \Lambda}}
\label{eqn:diff}
\end{equation}

Hence, we can calculate $\bar{\Pi} $ and, consequently, the cumulative probability, which is used to obtain the new $\lambda$ states:

\begin{equation}
\Phi = \int_{\lambda =0}^{\lambda =1} \Pi^{'}(\Lambda_{i}) d \Lambda = \dfrac{i}{K}
\label{eqn:cumfun}
\end{equation}
where, $K$ is the total number of $\lambda$ states.

\section{Gibbs Ensemble Monte Carlo (GEMC)}\label{gemc}

The Gibbs Ensemble Method \cite{papa1987} is used to study phase coexistence with simultaneous Monte Carlo (MC) simulations of two boxes,  representing a two phase system, with periodic conditions. The boxes exchange molecules, energy and volume between them. Equilibrium is obtained through MC steps that consist of translation and rotation moves, volume exchange moves and randomly exchanges of molecules between the boxes. For the phase equilibrium of multi-component systems,the GEMC simulations should be carried out at the NPT (constant number of particles, pressure and temperature) ensemble to obey the requirement of an additional degree of freedom for mixtures. Meanwhile, the simulation of single component systems is carried out at constant number of particles, temperature and volume (NVT) since the two phase region would be a line for this system at constant pressure and temperature. The constant volume GEMC ensemble is rigorously equivalent to the canonical ensemble in the thermodynamic limit as demonstrated by \citeonline{frenkel}. The partition function of the GEMC-NVT ensemble is obtained considering that the particles in both boxes are subjected to the same intermolecular interactions. Also, the boxes’ volumes and number of particles ($N_{1}$,$N_{2}$,$V_{1}$ and $V_{2}$) can vary while the total volume ($V$) and total number of particles ($N$) remain constant ($N = N_{1} + N_{2}$,$V = V_{1} + V_{2}$):

\begin{equation}
\begin{aligned}
Q(NVT) {} \equiv & \sum_{N_{1}}^{N} \dfrac{1}{V \Lambda ^{3N} N_{1}!(N-N_{1})!} \int_{0}^{V} dV_{1} V_{1}^{N_{1}} V_{2}^{N_{2}} \int dx_{1}^{N_{1}} \exp[-\beta U(x_{1}^{N_{1}})] \\
& \int dx_{2}^{N_{2}} \exp[-\beta U(x_{2}^{N_{2}})]
\end{aligned}
\label{eqn:gepart}
\end{equation}

In order to define the acceptance rules for the MC moves, it is necessary to know the probability of finding the configuration with $N_{1}$ particles in box 1 with volume $V_{1}$ and positions $x_{1}^{N_{1}}$ and $x_{2}^{N_{2}}$. This probability is given by:

\begin{equation}
\pi(x_{1}^{N_{1}},x_{2}^{N_{2}},N_{1},N_{2},V_{1},V_{2}) \propto \dfrac{V_{1}^{N_{1}}V_{2}^{N_{2}}}{N_{1}!N_{2}!} \exp[-\beta U(x_{1}^{N_{1}}) -\beta U(x_{2}^{N_{2}})]
\label{eqn:geprob}
\end{equation}

The acceptance criterion for the translation and rotation moves of configuration A	to B is similar to the conventional NVT MC ensembles and is equal to:

\begin{equation}
acc_{A \rightarrow B} = \min(1,\exp[-\beta U(x_{A}^{N_{1}}) -\beta U(x_{B}^{N_{1}})])
\label{eqn:drprob}
\end{equation} 

The exchange volume moves happen by exchanging an amount $\Delta V$ between the boxes to achieve pressure equilibrium. $\Delta V$ can be chosen from a uniform distribution based on the maximum variation of volume defined ($\delta V_{max}$) with probability $1/\delta V_{max}$. The acceptance rule for these moves is: 

\begin{equation}
acc_{A \rightarrow B} = \min \left(1, \left(\dfrac{V_{1}^{B}}{V_{1}^{A}} \right)^{N_{1}=1} \left( \dfrac{V_{2}^{B}}{V_{2}^{A}} \right)^{N_{2}+1} \exp[-\beta U(x_{A}^{N}) -\beta U(x_{B}^{N})] \right)
\label{vprob}
\end{equation}

Particle exchange moves are carried out to obtain the equality of chemical potential between the boxes. One particle from one box is removed and then added to a random location in the other box. The criteria to accept or reject this type of move is:

\begin{equation}
acc_{A \rightarrow B} = \min \left( 1, \dfrac{N_{1}V_{2}}{N_{2}V_{1}}  \exp[-\beta U(x_{A}^{N}) -\beta U(x_{B}^{N})] \right)
\label{moleprob}
\end{equation}

This method has been widely used to calculate phase equilibrium, but it under performs for the region near the critical point due to large density fluctuations. The GEMC also has a poor performance for dense systems since the particle exchange moves have a low acceptance ratio for these dense systems.  





