\chapter{Introduction} % Main chapter title

\label{Chapter1} % Change X to a consecutive number; for referencing this chapter elsewhere, use \ref{ChapterX}
%Specifically, in some
%situations, free energy calculations appear to be capable
%of achieving RMS errors in the 1-2 kcal/mol range with
%current force fields, even in prospective applications.

Solvation free energy calculations with molecular dynamics (MD) have a variety of applications ranging from drug design in the pharmaceutical industry to development of separation technologies in the chemical industry. Solvation free energy is, more specifically, the Gibbs free energy difference between the solute alone in the gas phase and the solute interacting with the solvent. Through the study of this solvation phenomenon, it is possible to obtain information about the behavior of the solvent in different chemical environments and the influence of the solute's molecular geometry. It is also possible to calculate other important properties with the solvation free energy, namely the activity coefficient at infinite dilution, Henry constant, partition coefficients and solubility. 

The solvation phenomenon is intrinsically complex. There are many competing forces interfering in the behavior of the solute-solvent interaction and free energy simulations are susceptible to sampling problems for high energy regions. With the intention of improving free energy calculations, simulation methodologies such as the expanded ensemble \cite{lyubartsev}, thermodynamic integration \cite{kirkwood1935}, free energy perturbation \cite{zwanzig1954,bennet1976,mbar} and umbrella sampling \cite{TORRIE1977187} have been developed in order to obtain accurate estimations for the energy differences. 

Another influencing factor in the output of these calculations are the force fields chosen to describe the solvent and solute molecules. Force fields have different levels of description (quantum mechanics, atomistic, coarse grained). In the coarse grained description, molecules are grouped in pseudo atoms or beads. Coarse grained models generally reproduce free energy differences since the effects of reducing degrees of freedom in the entropy are counterbalanced by the reduction of enthalpic terms \cite{kmiecik2016}. Additionally, the success of a coarse grained force field is important to increase the scale of solvation free energies calculations and reveal deficiencies in the description of small molecules by these models \cite{mobley2007,shirts2013}. That's the reason we, at this study, try to provide information about free energy calculation on the expanded ensemble with the SAFT-$\gamma$ Mie coarse grained force field. This force field uses the Mie Potential \cite{MIE} and has a more straightforward method of obtaining its parameters than other models. It was initially parameterized with pure component equilibrium and interfacial tension data. This strategy has provided satisfactory results for prediction of phase equilibrium of aromatic compounds, alkanes, light gases and water \cite{herdes2015,muller2017,lobanova2015} , thermodynamic properties of carbon dioxide and methane \cite{cassiano1}, multiphase equilibrium of mixtures of water, carbon dioxide and n-alkanes \cite{lobanova2016} and water/oil interfacial tension \cite{herdes2017}. 

The solvents and solutes chosen in this study range from standard sets used as benchmark in solvation free energy calculations to ring substances used as a model to asphaltenes. Asphaltenes are complicated to characterize by determining their composition on a molecular basis, but it is generally accepted that they can be characterized as a fraction of crude oil soluble in toluene and insoluble in n-alkenes (pentane, hexane, heptane) \cite{SJOBLOM2003399}. They've motivated many studies interested in developing models for their structure due to all the problems they can cause during their transportation and refining \cite{SJOBLOM20151}. This work’s strategy to test the efficiency of the SAFT-$\gamma$ Mie force field in describing the solvation phenomenon was to use molecules that appear in asphaltenes models and that have similarities in terms of solubilities with asphaltenes (phenanthrene, anthracene, pyrene) for the solutes. Meanwhile, for the solvents, we chose compounds that are used to characterize asphaltenes (toluene, hexane). The anti solvent/solvent effect of carbon dioxide was also tested due to its influence in asphaltene precipitation during the oil processing \cite{SOROUSH2014405}. Asphaltenes are described by their solubility, hence simulation scale can be increased and studies with more complete models can be carried out if this coarse grained model can correctly describe solvation free energy differences of simple molecules mimicking asphaltenes.

