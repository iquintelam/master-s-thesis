\chapter{Introduction} % Main chapter title

\label{Chapter1} % Change X to a consecutive number; for referencing this chapter elsewhere, use \ref{ChapterX}
Specifically, in some
situations, free energy calculations appear to be capable
of achieving RMS errors in the 1-2 kcal/mol range with
current force fields, even in prospective applications.

The most immediate application of these techniques is to guide synthesis for lead optimization, but applications to scaffold
hopping and in other areas also appear possible.

At the same time, it is clear that not all situations are
so favorable, so it is worth asking what level of accuracy
is actually needed


The parameterisation of a force field is a non-trivial and a cumbersome task. Some of
the main requirements on a force field include the accuracy, transferability, and robustness. Unfortunately, there is no generic force field that can reproduce all properties with a
unique parameter set.

MD(molecular dynamics)
Solvation free energy calculations with molecular simulations have a variety of applications for drug design in the pharmaceutical industry  and for development of separation technologies in the chemical industry. These calculations can provide information about the behavior of the solvent in different chemical environments and the influence of the solute's molecular geometry. Besides that, these calculations also can be used to obtain infinite dilution activity coefficients, Henry's law constants and solubility. Another reason for the study of the solvation phenomena is its intrinsic complexity. There are many competing forces interfering in the behavior of the solute-solvent and free energy simulations are susceptible to sampling problems. With the intention of improving free energy simulations, post processing methods \cite{mbar,bareva,dexp,gdel} have been developed and evaluated and a variety of work have been published about free energy calculations in the last decades. Recent work \cite{mobley2014,mobley2017} made available a big database of hydration free energy of small molecules using the GAFF force field. A comparison of polar and nonpolar contributions to these hydration free energy indicated the significance of each the terms  \cite{izairi2017}. Garrido \textit{et al.} \citep{garrido,garrido2011} calculated the free energy of solvation of large alkanes in 1-octanol and water with three different force fields (TraPPE, Gromos, OPLS-AA/TraPPE) and the solvation free energy of propane and benzene in non aqueous solvents like n-hexadecane, n-hexane, ethyl benzene, acetone  with the force fields TraPPE-UA and TraPPE-AA. Roy \textit{et al.} \citep{roy2017} addressed the choice of the Lennard Jones parameters for predicting solvation free energy in 1-octanol. Gon\c{c}alves and Stassen \citep{goncalves} calculated the free energy of solvation in the solvents tetrachloride, chloroform and benzene with GROMOS force field. Using the GAFF and the polarizable AMOEBA force fields, Mohamed \textit{et al.} \citep{mohamed2016} evaluated the solvation free energy of small molecules in toluene, chloroform and acetonitrile and obtained a mean unsigned error of 1.22 $kcal/mol$ for the AMOEBA and 0.66 $kcal/mol$ for the GAFF. Though these variety of data using the intramolecular Lennard-Jones potential, we are not aware of works using the Mie Potential\cite{MIE} in free energy calculations. We, at this study, try to provide information about theses calculations with the SAFT-$\gamma$ Mie coarse grained force field. The sets of solvents and solutes chosen in this study range from the standard sets used as benchmark in solvation free energy calculation to ring substances used as a model to asphaltenes. The latter is used to observe the phenomenon that carbon dioxide has on the solvation of phenanthrene in toluene.The partition coefficients, that are a measure of the partitioning of one solute in two solvents, have also been calculated at the temperature (T) according to the following equation:

The SAFT-$\gamma$ Mie coarse grained force field has a more straightforward method of obtaining its parameters than other models. It was initially parametrized with pure component equilibrium and superficial tension data. This strategy have provided satisfactory results for the prediction of phase equilibrium of aromatic compounds, alkanes, light gases and water \cite{herdes2015,muller2017,lobanova2015} , thermodynamic properties of carbon dioxide and methane \cite{cassiano}, multiphase equilibrium of mixtures of water, carbon dioxide and n-alkanes \cite{lobanova2016} and water/oil interfacial tension \cite{herdes2017}. Assessing the success of the The SAFT-$\gamma$ Mie model in solvation free energy calculations can be useful for this force field development. The output of these calculations are highly dependent on the the force field and they can also reveal deficiencies in the description of small molecules \cite{mobley2007,shirts2013}. Other important reason for testing a coarse grained force field is that they can generally reproduce free energy difference since the effects of the degrees of freedom reduction  in the entropy are counterbalanced by reduced enthalpic terms \cite{kmiecik2016}. Hence, knowing if more coarse grained approaches have a similar performance to the all atoms force fields can help increase the scale of solvation free energy calculations. 