\chapter{Introduction} % Main chapter title

\label{Chapter1} % Change X to a consecutive number; for referencing this chapter elsewhere, use \ref{ChapterX}
%Specifically, in some
%situations, free energy calculations appear to be capable
%of achieving RMS errors in the 1-2 kcal/mol range with
%current force fields, even in prospective applications.

Solvation free energy differences calculations with molecular dynamics (MD) have a variety of applications ranging from drug design in the pharmaceutical industry  to the development of separation technologies in the chemical industry. The solvation free energy difference is, more specifically, the Gibbs free energy difference between the solute alone in the gas phase and the solute interacting with the solvent. Through the study of this solvation phenomena, it is possible to obtain  information about the behavior of the solvent in different chemical environments and the influence of the solute's molecular geometry. It is also possible to calculate other important properties like the activity coefficient at infinite dilution, the Henry constant, partition coefficients and solubility with the solvation free energy. 

The solvation phenomena is intrinsic complex, there are many competing forces interfering in the behavior of the solute-solvent interaction and free energy simulations are susceptible to sampling problems for high energies regions. With the intention of improving free energy calculations, simulations methodologies like the expanded ensemble \cite{lyubartsev}, thermodynamic integration \cite{kirkwood1935}, free energy perturbation \cite{zwanzig1954,bennet1976,mbar} and umbrella sampling \cite{TORRIE1977187} have been developed in order obtain accurate estimations for the energies differences. 

Other influencing factor in the output of these calculations are the force fields chosen to describe the solvent and solute molecules, the free energies differences  can reveal deficiencies in the description of small molecules \cite{mobley2007,shirts2013}. The force fields have different levels of description (quantum mechanics, atomistic, coarse grained). In the coarse grained description, the molecules are grouped in pseudo atoms or beads. The coarse grained models generally reproduce free energy difference since the effects of the degrees of freedom reduction  in the entropy are counterbalanced by reduced enthalpic terms \cite{kmiecik2016}. Hence, the success of these kind of models are important to  increase the scale of solvation free energies calculations. That's the reason I, at this study, try to provide information about free energy calculation on the expanded ensemble with the SAFT-$\gamma$ Mie coarse grained force field. The SAFT-$\gamma$ Mie coarse grained force field uses the Mie Potential \cite{MIE} and has a more straightforward method of obtaining its parameters than other models. It was initially parametrized with pure component equilibrium and superficial tension data. This strategy have provided satisfactory results for the prediction of phase equilibrium of aromatic compounds, alkanes, light gases and water \cite{herdes2015,muller2017,lobanova2015} , thermodynamic properties of carbon dioxide and methane \cite{cassiano1}, multiphase equilibrium of mixtures of water, carbon dioxide and n-alkanes \cite{lobanova2016} and water/oil interfacial tension \cite{herdes2017}.  

The sets of solvents and solutes chosen in this study range from the standard sets used as benchmark in solvation free energy calculations to ring substances used as a model to asphaltenes. Asphaltenes are complicated to characterize by determining its composition on a molecular basis, but its is generally accepted that they can be characterized as a fraction of crude oil soluble in toluene and insoluble in n-alkenes (pentane, hexane, heptane) \cite{SJOBLOM2003399}. They've attracted a  lot of studies to develop models for its structure for the last ten years  due all the problems they can cause during its transportation and refining \cite{SJOBLOM20151}. The strategy for this work was to use molecules that appears in the asphaltenes models and that have similarities in terms of solubilities with it (phenanthrene, anthracene, pyrene) and solvents that characterize alphaltenes (tolune, hexane) to test the efficiency of the SAFT-$\gamma$ Mie force field in describing the solvation phenomena. The anti solvent/solvent effect of carbon dioxide was also tested due to the $CO_{2}$ influence in the asphaltene precipitation during the oil processing. The asphaltenes are described by its solubility, hence if this coarse grained model can correctly describe solvation free energy differences of simple molecules mimicking them, the simulations scale can be increased and studies with more complete asphaltenes models can be carried out.

