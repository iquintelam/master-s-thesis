\chapter{Introduction} % Main chapter title
\label{Chapter1} % 
\pagestyle{simple}
\onehalfspacing
Solvation free energy calculations with molecular dynamics (MD) have a variety of applications ranging from drug design in the pharmaceutical industry to the development of separation technologies in the chemical industry. {Solvation free energy is, more specifically, the difference in free energy related to the process of transferring a solute from an ideal gas phase into a liquid solution \cite{shirts2013}}. Through the study of the solvation phenomenon, it is possible to obtain information about the behavior of the solvent in different thermodynamic conditions and the influence of the solute's molecular geometry. It is also possible to calculate other important properties with the solvation free energy, namely the activity coefficient at infinite dilution, Henry constant, and partition coefficients.{ Additionally, solvation free energy calculations can be part of the methodology of calculating solubility from molecular dynamics}. 

The solvation free energy calculations described above are intrinsically complex due to the many competing forces interfering in the behavior of the solute-solvent interaction. In addition, free energy simulations are susceptible to sampling problems in low energy regions, and simulation results need to be correctly post-processed in order to yield free energy differences with small uncertainties. {Another influencing factor in the output of these calculations is the choice of force field used to model the solvent and solute molecules. Force field is the name given to the group of parameters and equations used to represent the potential energy function of a system in molecular simulations. They have different levels of description, such as quantum mechanics, atomistic, and coarse-grained. The quantum mechanics approach describes the motion of electrons and requires for the solution of the Schr\"{o}dinger equation during the simulation. In the atomistic description, only the atomic motions are represented,  and this is done by solving Newton's equations of motion. Finally, in the coarse-grained description, atoms are grouped into pseudo-atoms or beads, and the equations of motion are solved for them. 
	
These coarse-grained models are generally able to reproduce experimental free energy differences since the effects of reducing degrees of freedom in the entropy are counterbalanced by the reduction of enthalpic terms \cite{kmiecik2016}. This fact makes these models a viable option to decrease the computational time of solvation free energy calculations. Additionally, deficiencies in the description of small molecules by coarse-grained models can be revealed by free energy calculations \cite{mobley2007,shirts2013}. Hence, we, in this study, assess the performance and shortcomings of the SAFT-$\gamma$ Mie coarse-grained force field  \cite{avendano2011} with free energy calculations of a variety of solute-solvent pairs. We choose this coarse-grained force field because it uses, unlike the majority of the force fields, the Mie potential \cite{MIE} and because its method of obtaining parameters is more straightforward than those of other models. It was initially parameterized with pure component equilibrium and interfacial tension data \cite{avendano2011}, and this strategy has provided satisfactory results. Examples include the prediction of phase equilibrium of aromatic compounds \cite{muller2017}, alkanes, light gases \cite{herdes2015}, and water \cite{lobanova2015}, thermodynamic properties of carbon dioxide and methane \cite{cassiano1}, multiphase equilibrium of mixtures of water, carbon dioxide, and n-alkanes \cite{lobanova2016}, and water/oil interfacial tension \cite{herdes2017}.} 


We selected the solvents and solutes in our free energy calculations with the intention of testing the force field with standard sets used as a benchmark in solvation free energy calculations and with polycyclic aromatic substances used as models to asphaltenes. Asphaltenes are complicated to characterize by determining their composition on a molecular basis, but the literature broadly accepts that they can be described as a fraction of crude oil soluble in toluene and insoluble in n-alkanes (pentane, hexane, heptane) \cite{SJOBLOM2003399}. They have motivated many studies with interest in developing models for their structure and behavior due to all the problems they can cause during their transportation and refining such as precipitation during the oil processing \cite{SJOBLOM20151}. This precipitation issue is a recurrent problem due to the growing market of the production of crude oil in deep waters, whose conditions are favorable to precipitation \cite{AIC:AIC10243}. As an example, asphaltene precipitation due to pressure drop can clog oil production equipment and cause a growth in the cost of production \cite{doi:10.1021/ef010047l}. All these factors make the understanding of the behavior of asphaltenes in different chemical and physical environments relevant to the oil industry. 

As said in the previous paragraph, asphaltene characterization still faces some issues. Hence, we choose to use polycyclic aromatic hydrocarbons (PAHs), which have well-defined characteristics, to initially test the efficiency of the SAFT-$\gamma$ Mie force field in describing the solvation phenomenon. PAHs are a group of organic compounds that have fused rings, carbon and hydrogen in their structure \cite{RAVINDRA20082895}. The ones utilized in this work were phenanthrene, anthracene, and pyrene since they share similarities with asphaltenes regarding their solubility. In this context,  we selected compounds that are used to characterize asphaltenes (toluene, hexane) as solvents in our free energy calculations. We also tested the anti-solvent/solvent effect of carbon dioxide due to its influence in asphaltene precipitation during the oil processing \cite{SOROUSH2014405}. With these calculations of solvation free energies with the SAFT-$\gamma$ Mie model, we intend to improve this force field and provide satisfactory solvation free energy estimates of PAHs with a coarse-grained model. The success of the description of small asphaltene-like compounds by this force field can then open up the possibility of obtaining satisfactory results for more complex asphaltene models with a force field with a low computational cost.

