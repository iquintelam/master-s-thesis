\chapter{Methodology} % Main chapter title

\label{Chapter4} % Change X to a consecutive number; for referencing this chapter elsewhere, use \ref{ChapterX}

\section{Phenanthrene Parameterization}

The two parameterization strategies for ring molecules described in section \ref{parsaft} were followed for phenanthrene. For both of them, only vapor pressure data \cite{pvphen} was used due to the unavailability of saturated liquid density and the attractive parameter was set to six to avoid correlation with repulsive parameter. The parameterization with the ring equation of \citeonline{muller2017} was made with the number of segments fixed in 5 since this level of coarse graining was also used for the similar molecule anthracene in the original article:
\begin{figure}[th]
\centering
\includegraphics[width=0.35\linewidth]{Figures/fen5}
\caption{Geometry for $m_{s}=5$}
\label{fig:fen5}
\end{figure}

The minimization was done using the Particle Swarm Optimization (PSO) method with the following objective function:
\begin{equation}
\min\limits_{\sigma,\epsilon,\lambda_{r}} F_{obj}(\sigma,\epsilon,\lambda_{r})= \sum_{i=1}^{N_{p}} \left(\frac{P_{v}^{SAFT}(T_{i},\sigma,\epsilon,\lambda_{r})-P_{v}^{exp}(T_{i})}{P_{v}^{exp}(T_{i})} \right)^2
\label{eqn:fobjm}
\end{equation}

The $P_{v}^{exp}$ is the experimental vapor pressure and $P_{v}^{SAFT}$ is the vapor pressure obtained with SAFT-VR Mie EoS. The method used to calculate the bubble point was the one proposed by \citeonline{smithbook}. The parameters $\sigma$, $\epsilon $ and $\lambda _{r}$ resulted from this minimization are the final force field parameters to be used in molecular simulations. The parameterization with the \citeonline{lafitte2012} ring equation requires was done with $m_{s}=3$ so every bead would represent one ring. The first part of the estimation followed the same procedure of the one described above. As explained in section \ref{parsaft}, the \citeonline{lafitte2012} equation requires the estimation of the correction factors $c_{\sigma}$ and $c_{\epsilon}$ (Eqs. \eqref{eqn:csigma} and \eqref{eqn:ceps}). The PSO method was also used and the objective function is the one in Eq. \eqref{eqn:fobjla}. The vapor pressures and saturated liquid densities from molecular simulations was obtained with the GEMC method (section \ref{gemc}).
    
%Essas simulações no GEMC-NVT foram feitas inserindo-se
%aleatoriamente 400 moléculas na caixa líquida e 100 moléculas na caixa vapor. As densidades
%iniciais das caixas foram escolhidas ajustando-as às densidades obtidas com a EdE SAFT-VR
%Mie, para evitar que todas as moléculas migrassem pra uma única caixa ao longo da
%simulação. A simulação consistiu em, no mínimo, 10000 ciclos de equilibração e 100000
%ciclos de produção, sendo que cada ciclo de MC corresponde a 1000 tentativas de rotação,
%1000 de translação, 100 de inserção, 100 de exclusão e 10 de variação de volume. A distância
%de corte usada foi igual a quatro vezes o valor do diâmetro do segmento estimado e as
%interações de van der Waals foram calculadas através do potencial Mie com correção de longa
%distância (tail correction). As propriedades densidade de vapor ( vap ρ ), densidade de líquido
%( liq ρ ) e pressão ( sim Pv ) foram amostradas a cada 100 ciclos de MC e essas amostragens foram
%divididas em cinco blocos para os cálculos da média e do desvio padrão. Os resultados
%
%obtidos nessas simulações foram usados para estimar coeficientes de correção para os
%parâmetros conformacionais do campo de força (
%σ c e c
%), que são relacionados aos
%parâmetros provenientes da EdE SAFT-VR Mie através de parâmetros oriundos de ajuste,
\section{Solvation Free energy Calculations}

The molecular dynamic simulations to estimate the free energy differences with the SAFT-$\gamma$ Mie force field were performed using the Lammps package \cite{lammps}. The motion's equations were integrated with the velocity-Verlet algorithm \cite{verlet} with time step of 1 fs. The molecules were treated as rigid bodies, as required by the coarse grained model. The thermostat and the barostat were the Nose/Hoover with chains with a damping factors of 100 and 1000 fs respectively. The SAFT-$\gamma$ Mie model doesn't consider electrostatics interactions explicitly, hence there was no shifting of forces or long range corrections and the potential cutoff was equal to 20 $\dot{A}$ with a neighbor skin of 2 $\dot{A}$. The initial configuration of the  solvated systems were generated with the Packmol package \cite{packmol}. For the binary mixtures, one molecule of solute and a varying number of solvent molecules- 700 molecules for toluene and octanol, 1024 for hexane, 3000 for water - were randomly added to a cubic box. The simulations to study solvation free energy of phenanthrene in a mixture of toluene and carbon dioxide were done with different fractions of carbon dioxide. The  system consisted of one molecule of phenanthrene for all the fractions and 123 molecules of $CO_{2}$ and 618 molecules of toluene for $w_{CO_{2}} = 0.087$; 166 molecules of $CO_{2}$; 589 molecules of toluene for $w_{CO_{2}} = 0.119$; 232 molecules of $CO_{2}$ and 545 molecules of toluene for $w_{CO_{2}} = 0.169$ and 380 molecules of $CO_{2}$ and 446 molecules of toluene for $w_{CO_{2}} = 0.289$.

All simulations were carried out maintaining the temperature and pressure constant at 298 K and 1 bar, except the ones containing carbon dioxide. These were performed at 298 K and at the pressure of the liquid phase equilibrium correspondent to the $CO_{2}$ fraction \cite{co2toliq}. Primarily, the initial box was equilibrated at the NPT ensemble for 2 ns and then the resulting configurations were used on the alchemical simulations. These were carried out with the Lammps user package for alchemical simulations with the Mie Potential developed by our group, available at https://github.com/atoms-ufrj/USER-ALCHEMICAL. In the simulations, the sampling of a new state was tried at every 10 MD steps. In order to define the optimal values of $\lambda$ and $\eta$ related to each state, short trials simulations, lasting around 10 ns, were carried out. In the first simulation, the group of $\lambda$ for all the pairs solvent-solute was: (0.0,0.15,0.2,0.25,0.3,0.4,0.45,0.5,0.55,0.7,0.9,1.0) and the $\eta s$ were set to zero or were given the values of the ones found for similar mixtures. The subsequent groups of $\eta$ were estimated  with the flat histogram approach (Eq. \eqref{eqn:weight}) using the solvation free energy values stemming from the previous simulations. The results with the new weights were then utilized to optimize the group of $\lambda s$ by minimizing the number of round trips as described in section \ref{ee}. The $\eta s$ corespondent to the newest group of $\lambda s$ were interpolated from the free energy differences. With the final values of the $\eta$ and $\lambda $ defined for each mixture, larger simulations with a time of 20 ns were performed. 

The post processing method used to calculate the free energies was the Multisate Bennet Acceptance Ratio (MBAR) described in section \ref{mbar}. The software alchemical-analysis \cite{klimovich} were used to obtain the $\Delta G_{solv}$ with MBAR and to assess the results quality.

%72 molecules of $CO_{2}$ and 652 molecules of toluene for $x_{CO_{2}} = 0.1$;

