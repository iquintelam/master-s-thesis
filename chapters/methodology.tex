\chapter{Methodology} % Main chapter title

\label{Chapter4} % Change X to a consecutive number; for referencing this chapter elsewhere, use \ref{ChapterX}

\section{Phenanthrene Parameterization}

\section{Solvation Free energy Calculations}

The molecular dynamic simulations to estimate the free energy differences with the SAFT-$\gamma$ Mie force field were perfomed using the Lammps package \cite{lammps}. The motion's equations were integrated with the velocity-Verlet algorithm \cite{verlet} and a time step of 1 fs. The molecules were treated as rigid bodies since the coarse grained model doesn't have this level of detailment. The thermostat and the barostat were the Nose/Hoover with chains with a damping factors of 100 and 1000 fs respectively. The SAFT-$\gamma$ Mie model doesn't consider electrostatics interactions explicitly, hence there was no shifting of forces or long range corrections and the potential cutoff was equal to 20 $\dot{A}$ with a neighbor skin of 2 $\dot{A}$. The initial configuration of the  solvated systems were generated with the Packmol package \cite{packmol}. For the binary mixtures, one molecule of solute and a varying number of solvent molecules- 700 molecules for toluene and octanol, 1024 for hexane, 3000 for water - were randomly added to a cubic box. The simulations to study solvation free energy of phenanthrene in a mixture of toluene and carbon dioxide were done with different fractions of carbon dioxide. The system consisted of one molecule of phenanthrene for all the fractions and 123 molecules of $CO_{2}$ and 618 molecules of toluene for $w_{CO_{2}} = 0.087$; 166 molecules of $CO_{2}$; 589 molecules of toluene for $w_{CO_{2}} = 0.119$; 232 molecules of $CO_{2}$ and 545 molecules of toluene for $w_{CO_{2}} = 0.169$ and 380 molecules of $CO_{2}$ and 446 molecules of toluene for $w_{CO_{2}} = 0.289$.

All simulations were carried out maintaining the temperature and pressure constant at 298 K and 1 bar, except the ones containing carbon dioxide. These were performed at 298 K and at the pressure of the liquid phase equilibrium correspondent to the $CO_{2}$ fraction \cite{co2toliq}. Primaraly, the initial box was equilibrated at the NPT ensemble for 2 ns and then the resulting configurations were used on the alchemical simulations. These were carried out with the Lammps user package for alchemical simulations with the Mie Potential developed by our group, available at https://github.com/atoms-ufrj/USER-ALCHEMICAL. The first expanded ensemble simulations were short trials, lasting around 10 ns, to obtain the values of the weights and the optimal values of lambda. The first group of $\lambda$ for all the pairs solvent-solute was: (0.0,0.15,0.2,0.25,0.3,0.4,0.45,0.5,0.55,0.7,0.9,1.0) and the first weights were set to zero or were given the values of the weights found for similar mixtures. The subsequent groups of $\lambda$ were optimized mimizating the number of round trips .With the values of the weights and the $\lambda $ defined for each mixture, larger simulations with a time of 20 ns were performed. Lastly, we applied the MBAR method and the tools of the software alchemical-analysis \cite{klimovich} on the equilibrated data to effectively calculate and analyze the free energy of solvation.

%72 molecules of $CO_{2}$ and 652 molecules of toluene for $x_{CO_{2}} = 0.1$;

