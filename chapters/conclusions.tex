\chapter{Conclusions} % Main chapter title

\label{Chapter6} % Change X to a consecutive number; for referencing this chapter elsewhere, use \ref{ChapterX}
The work done on this dissertation focused on the calculation of solvation free energy differences of aromatics solutes that can mimic the asphaltene's properties with the SAFT-$\gamma$ Mie force field. By doing that, we enriched the information about these calculations since the data available on the literature generally uses all atoms force fields based on the Lennard Jones Potential and solvation is studied on water for the majority of the data. The success of a coarse grained force filed in these calculations can increase the scale of free energy simulations and decrease the simulation time. The parametrization of SAFT-$\gamma$ Mie force field is more straightforward when compared to other force field since its parameters are obtained with the SAF-VR Mie EoS. Following this strategy, the phenanthrene parameters were obtained with two different ring equations and geometries. The ring equation proposed by \citeonline{muller2017} provided the more adequate set of parameters with respect to theoretical rigor.    

-nao usar a agua
-testar outros sistemas com agua para o mesmo kij
-testar modelos mais complexos de asfaltenos com esse campo de forca
