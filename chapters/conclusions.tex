\chapter{Conclusions} % Main chapter title

\label{Chapter6} % Change X to a consecutive number; for referencing this chapter elsewhere, use \ref{ChapterX}

This dissertation consisted in  the study of solvation free
energy calculations of aromatic solutes that can mimic asphaltenes with the SAFT-$\gamma$ 
Mie force field. By doing that, we provided information about these calculations since solvation free energy studies are mostly done using  water as solvent and all atom force fields based on the Lennard Jones Potential. The parametrization of the SAFT-$\gamma$  Mie force field is more straightforward
when compared to other force fields since its parameters are obtained through the SAF-VR
Mie EoS. Following this strategy, the phenanthrene parameters were obtained using two
different ring equations and geometries. The ring equation proposed by Müller and Mejía
(2017) provided the more adequate set of parameters.
The potential energy data for every intermediate state were obtained with simulations at the expanded ensemble. 

Solvation free energy differences were then
estimated with these data using the MBAR method. The results for solvation free energy differences with non aqueous solvents had absolute deviations to the experimental
data of less than 2.0 kcal/mol, except for the pair 1-octanol+anthracene. The geometry
effect on the free energy curves was also observed- larger molecules had steeper curves
and larger absolute deviations. The influence of carbon dioxide on the solvation free
energy of phenanthrene in toluene was found to be minimum. The $\Delta G_{solv}$ decreased slightly until the mass fraction of $CO_{2}$ was equal to 0.119 and, after this point, solvation free
energies increased. 

Hydration free energy differences calculations with the SAFT-$\gamma$ Mie model
required the use of relatively larger values of $k_{ij}$ in order to obtain satisfactory results.
The parameter was estimated with the output from molecular dynamics data as
the strategy of using the SAFT-VR Mie EoS also didn’t provide good results. This
necessity of one additional parameter happens probably due to the lack of an association term
on the EoS that the model is based on. The results with $k_{ij}$ estimated with MD output
were great, the absolute deviations to the experimental data found were smaller than
the ones for the GAFF force field.

Generally, the SAFT-$\gamma$ Mie force field proved to be a good model to represent the solvation
phenomenon. It correctly described solvation free energy differences of solutes
mimicking asphaltenes in hexane, toluene, 1-octanol and water. The requirement of
binary interaction parameter estimated with MD output for hydration free energies
increases the simulation time, which is already larger for this water model due to its
coarse graining level. Nevertheless, the SAFT-$\gamma$ Mie force field for water used doesn’t predict freezing at room temperature as other force fields, which is essential for hydration
free energy calculations.
This dissertation had success in using a coarse grained force field to perform
free energy calculations. Based on this work, we have some ideas for future development. We intend to test the binary interaction parameter transferability to calculations with other
aromatic solutes in water. Additionally, we want to use the SAFT-$\gamma$ Mie force field to model more
complex asphaltenes models and, consequently, increase the scale of these simulations. The final step to expand this work would be to develop new methodologies to use solvation free energies to effectively calculate solubility


