\chapter{Conclusions} % Main chapter title

\label{Chapter6} % Change X to a consecutive number; for referencing this chapter elsewhere, use \ref{ChapterX}
The work done on this dissertation focused on the calculation of solvation free energy differences of aromatics solutes that can mimic the asphaltene's properties with the SAFT-$\gamma$ Mie force field. By doing that, we enriched the information about these calculations since the data available on the literature generally uses all atoms force fields based on the Lennard Jones Potential and solvation is studied on water for the majority of the data. The parametrization of SAFT-$\gamma$ Mie force field is more straightforward when compared to other force field since its parameters are obtained with the SAF-VR Mie EoS. Following this strategy, the phenanthrene parameters were obtained with two different ring equations and geometries. The ring equation proposed by \citeonline{muller2017} provided the more adequate set of parameters with respect to theoretical rigor. 

The potential energy data for every intermediated state was obtained with simulations at the expanded ensemble. The solvation free energy differences was then estimated with this data using the MBAR method. The results for the solvation free energy differences  with non aqueous solvents had absolute deviations to the experimental data of less of 2.0 kcal/mol, unless for the pair 1-octanol+anthracene. The geometry effect on the free energy curves was also observed, larger molecules had steeper curves and larger absolute deviations. The influence of carbon dioxide on the solvation free energy of phenanthrene in toluene found was minimum. The $\Delta G_{solv}$ decreased slightly until the mass fraction of $CO_{2}$ was equal to 0.119 and, after this point, the solvation free energies increased.   

The hydration free energy differences calculations with the SAFT-$\gamma$ Mie model required the use of relatively larger values of $k_{ij}$ in order to obtain satisfactory results. The parameter was estimated with the output form molecular dynamics data because the  strategy of using the SAFT-VR Mie EoS also didn't provide good results. This necessity of one additional parameter is probably due to the lack of an association term on the EoS the models is based on. The results with $k_{ij}$ estimated with MD output were great, the absolute deviations to the experimental data found were smaller than the ones for the GAFF force field.   

Generally, the SAFT-$\gamma$ Mie proved to be a good model to represent the solvation phenomenon. It correctly described the solvation free energy differences of solutes mimicking asphaltenes in hexane, toluene, 1-octanol and water. The requirement of binary interaction parameter estimated with MD  output for the hydration free energies increases the simulation time, which is already larger for water for this model due to its coarse graining level. Nevertheless, SAFT-$\gamma$ Mie force field for water used doesn't predict freezing at room temperature as other force fields, which is essential for hydration free energy calculations.

This masther's thesis had success in using a coarse grained force filed in to perform free energy calculations and, consequently, in increasing the scale of these calculations. Based on this work, we have some ideas for future development. Test the binary interaction parameter transferability to calculations with other aromatic solutes in water. Additionally, use the SAFT-$\gamma$ Mie force filed to model more complex asphaltenes models and develop new methodologies to use the solvation free energies to effectively calculate solubility. 
 

