\chapter{Conclusions} % Main chapter title

\label{Chapter6} 

This dissertation consisted of the study of solvation free energy calculations of aromatic solutes in different solvents with the SAFT-$\gamma$ Mie coarse-grained force field. Solvation free energy studies are mostly done using water as a solvent and with all-atom force fields based on the Lennard-Jones potential. Therefore, with this study, we were able to provide data about the capability of a coarse-grained force field based on the Mie Potential in calculating solvation free energy differences. Additionally, the solvation free energy estimations carried out here can help improve the SAFT-$\gamma$  Mie force field since these calculations are helpful in identifying errors in the modeling process. The SAFT-$\gamma$ Mie uses the SAFT-VR Mie EoS in its parameterization, which results in a more straightforward method of obtaining parameters. Following this strategy, the phenanthrene parameters, which were not available in this force field database, were obtained using vapor-liquid equilibrium data and two different ring equations and geometries. However, only the parameters estimated with the ring equation proposed by \citeonline{muller2017} were utilized in the solvation free energy simulations since this equation did not require molecular simulation data in its parameterization.

To perform our expanded ensemble simulations, we optimized the coupling parameters and their respective simulation weights. The resulting potential energies from the expanded ensemble simulations were then served as input to estimate solvation free energy differences with the MBAR method. The results for solvation free energy differences with non-aqueous solvents had absolute deviations from the experimental
data of less than 2.0 kcal/mol, except for the pair 1-octanol+anthracene. We also observed the geometry effect on the free energy curves - larger molecules had steeper curves and more substantial absolute deviations. The influence of carbon dioxide on the solvation free energy of phenanthrene in toluene was found to be negligible according to the SAFT-$\gamma$ Mie force field. 

Hydration free energy differences calculations with the SAFT-$\gamma$ Mie model required the use of relatively large values of $k_{ij}$ to obtain satisfactory results. We chose to estimate the parameter with the output from molecular dynamics data since the strategy of using the SAFT-VR Mie EoS provided high absolute deviations from the experimental data. This necessity of one additional parameter probably happens due to the lack of a term to account for the hydrogen bond in the EoS that this force field is based and due to the problems associated with the coarse-graining of water molecules. The results with $k_{ij}$ estimated with MD output were satisfactory, the absolute deviations from the experimental data found were smaller than the ones for the GAFF and ELBA force field. We also used the solvation free energies to calculate partition coefficients in water/1-octanol and water/hexane. The absolute deviations from the experimental data obtained were similar to the ones found for all-atom force fields (GROMOS, TraPPE and OPLS-AA/TraPPE) and other coarse-grained force filed (ELBA).

Overall, the SAFT-$\gamma$ Mie force field proved to be an excellent model to represent the solvation phenomenon of non-aqueous solvents. It correctly described solvation free energy differences of solutes mimicking asphaltenes in hexane, toluene, 1-octanol. However, the calculation of hydration free energies required the use of a binary interaction parameter estimated with MD output, what increased the simulation time significantly. This fact evidenced flaws in the methodology used by the SAFT-$\gamma$ force field and made us question the feasibility of this model for hydration free energy calculations. Nevertheless, the SAFT-$\gamma$ Mie force field for water used here does not predict freezing at room temperature as other force fields, which is essential for our hydration free energy calculations. Hence, it would be relevant to test if the binary interaction parameter for our aromatic solutes estimated here can be used in hydration free energy calculations of other aromatic solutes. Also based on this dissertation, we have some ideas for future development. We intend to use the SAFT-$\gamma$ Mie force field to model bigger asphaltenes models and, consequently, increase the scale of the simulations we performed. Additionally, we want to develop new methodologies to calculate solubility using solvation free energies.


