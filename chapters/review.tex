% Chapter Template

\chapter{Literature Review} % Main chapter title

\label{Chapter1} % Change X to a consecutive number; for referencing this chapter elsewhere, use \ref{ChapterX}

%----------------------------------------------------------------------------------------
%	SECTION 1
%----------------------------------------------------------------------------------------

\section{Coarse Grained Force Fields}

Molecular simulations can be carried out at different levels of descriptions. The detailed atomistic level or \textit{ab initio}level is described by the laws of quantum mechanics. The system consists of a set of subatomic particulars in which Schrodinger's equation is solved for all of them. The next level is the atomistic description. It considers that the system is made up of atoms following the laws of statistical mechanics.  Force fields at this level are based on pair potentials with Coulombic charged sites, which account for the molecular interactions. The contributions due to to intramolecular interactions like bond-stretching, angle-bending and torsion are also usually accounted by these kind of force fields. When the scale of the simulations needs to be increased and the atomistic simulations become too computationally expensive, the coarse-grained (CG) description is more suited. It considers that the system is made up of pseudo atoms or beads that contain multiple atoms. 

There is a obvious loss of information in grouping atoms, hence it is necessary to assure that the process of eliminating unnecessary or unimportant information ('coarse graining') doesn't affect the system's physical behavior. The coarse grained force fields based on this description are developed by mapping the atomistic model to define the pseudo atoms. This mapping is normally done by grouping similar funcional groups. The level of coarse-graining also needs to be defined, up to 6 heavy atoms (non-hydrogen atoms) per bead in order to not loose much detail and maintain isotropic representations of the beads \cite{shinoda2007,martini2007,hadley2012}. The CG force field can be parametrized following two different approaches: bottoms up and top down. The bottoms up approach uses information of a more detailed scale such as the \textit{ab initio} description or the atomistic description to obtain the information necessary to the parametrization. This method depends highly of the quality of the detailed model to succeed. Meanwhile, the top down methodology obtains the parameters from one larger scales. This information at larger scales could be experimentally observed data like thermodynamic properties or native-structure based properties. 

%http://aip.scitation.org/doi/pdf/10.1063/1.4818908


The main advantage of
coarse-graining lies in the immense speed up of the simulation

Obviously, coarse graining comes at the cost of loosing electronic and atomistic details.
Therefore it is crucial to identify the unimportant details and to preserve feature that are
essential for the description of the phenomenon on interest. In particular, CG mapping
is critical for the accuracy, transferability, robustness, and computational efficiency of the
CG model.



An early example of coarse-gaining is the seminal
work of Levitt and Warshel in 1975 [90], where the authors studied the folding of small
proteins by obtaining atomistic potentials from the QM trajectories using MD simulations. 



The pair potential is based on molecule specific
parameters, such as the ranges of the repulsive and attractive interactions, the size of
the bead and the energy parameter which characterises the strength of the attractive interaction.

At short distances, the potential is
invariably repulsive and tends to infinity, whereas at larger separations, the potential is
attractive and tends to zero so that the energy remains finite. The total potential energy
is usually assumed as a sum of both, the repulsive and the attractive, contributions. Some
of the commonly used pair potentials are illustrated in Figure 2.1 and briefly described
below.

sing the versatile exponents has been shown to provide
a significant improvement of the vapour pressure and the second-derivative thermodynamic
properties of real fluids, such as speed of sound, heat capacity, and compressibility \cite{avendano2011,lafitte2013,lafitte2006}.

In conventional statistical mechanics, it is common to replace time
averages with ensemble averages. This hypothesis, currently referred to as the ergodic
hypothesis, states that the time and the ensemble averages of two systems with the same tate variables, e.g., N, V , and T are identical in infinite time and thermodynamic limits.

By contrast, commonly used
bottom-up approaches often make use of temperature dependent parameters that have to
be re-derived at various state points. The parameters obtained with our SAFT-gamma top-down
methodology can be used as a direct input in molecular simulation. Unlike intermolecular
interactions can also be obtained by using appropriate experimental data for the mixtures,
but often combining rules are employed for some of the parameters as described in the
next section.

The parameterisation of a force field is a non-trivial and a cumbersome task. Some of
the main requirements on a force field include the accuracy, transferability, and robustness. Unfortunately, there is no generic force field that can reproduce all properties with a
unique parameter set.



mostly the intermolecular interactions and the torsional terms. In this work, our main focus lies on the
dispersion interactions. The dispersion interactions are normally described by the intermolecular potentials presented in Section 2.1.2.

The LJ potential function forms the basis for a large number of widely
used molecular force fields, such as OPLS [73], TraPPE [74], NERD [75] force fields, and
the TIP [76] and SPC [77] models of water, to mention but a few. The popular coarsegrained MARTINI force field [78] is also based on the LJ representation.


