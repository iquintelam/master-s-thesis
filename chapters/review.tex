% Chapter Template

\chapter{Literature Review} % Main chapter title

\label{Chapter1} % Change X to a consecutive number; for referencing this chapter elsewhere, use \ref{ChapterX}

%----------------------------------------------------------------------------------------
%	SECTION 1
%----------------------------------------------------------------------------------------

\section{Coarse Grained Force Fields}

Molecular simulations can be carried out at different levels of descriptions. The detailed atomistic level or \textit{ab initio}level is described by the laws of quantum mechanics. The system consists of a set of subatomic particulars in which Schrodinger's equation is solved for all of them. The next level is the atomistic description. It considers that the system is made up of atoms following the laws of statistical mechanics.  Force fields at this level are based on pair potentials with Coulombic charged sites, which account for the molecular interactions. The contributions due to to intramolecular interactions like bond-stretching, angle-bending and torsion are also usually accounted by these kind of force fields. When the scale of the simulations needs to be increased and the atomistic simulations become too computationally expensive, the coarse-grained (CG) description is more suited. It considers that the system is made up of pseudo atoms or beads that contain multiple atoms. 

There is a obvious loss of information in grouping atoms, hence it is necessary to assure that the process of eliminating unnecessary or unimportant information ('coarse graining') doesn't affect the system's physical behavior. The coarse grained force fields are developed by mapping the atomistic model to define the pseudo atoms with the intetion of assuring that the model has accuracy, transferability, robustness, and computational efficiency. This mapping is normally done by grouping similar funcional groups. The level of coarse-graining also needs to be defined, up to 6 heavy atoms (non-hydrogen atoms) per bead in order to not loose much detail and maintain isotropic representations of the beads \cite{shinoda2007,martini2007,hadley2012}. The CG force field can be parametrized following two different approaches: bottoms up and top down. The bottoms up approach uses information of a more detailed scale such as the \textit{ab initio} description or the atomistic description to obtain the information necessary to the parametrization. This method depends highly of the quality of the detailed model to succeed. Meanwhile, the top down methodology obtains the parameters from one larger scales. This information at larger scales could be experimentally observed data like thermodynamic properties or native-structure based properties. 

One of the first applications of coarse grained models is the study of protein folding \cite{levitt1975,levitt1976}. These earlier protein CG models were based on the structure of the molecule and they contributed for the knowledge of the physicochemical forces associated with protein folding and protein interactions \cite{koga2001}.  More recent models focused on retaining the protein's chemical specificity. The Bereau and Deresmo model \cite{bereau2009} has a up to four-bead representation and was used in studies of protein folding and aggregation. However, this model still needs tuning to improve stability of proteins \cite{bereau2010}. The OPEP (Optimized Potential for Efficient Protein Structure Prediction) model \cite{opep2014,opep2015} has up to six-bead representation. It was used to investigate a variety of phenomena, ranging from protein folding to \textit{ab initio} peptide structure prediction \cite{opep2011,opep2009,opep20092}. Other CG protein models used in the literature are the Scorpion (solvated coarse-grained protein interaction)  \cite{scorpion2013}, the UNRES (united residue) \cite{unres2014} and the MARTINI model \cite{martini2013}. The later one is the most popular model for the CG modeling of membrane proteins \cite{martini20132}. The MARTINI model is also extensively used as CG model for water. This model represents four water molecules as one bead using a shifted Lennard Jones potential for the non bonded interactions. Though its extensive use, the MARTINI water model doesn't properly represent properties as interfacial tension and compressibility \cite{shinoda2010} and can freeze at room temperature \cite{winger2009,martini2007}, what makes necessary the use of anti-freeze agents during the simulations. This behavior can be explained by the high level of coarse graining (4:1), the lack of explicit charges and the use of the 12-6 potential. With the idea of improving the MARTINI model, \citeonline{chiu2010} used the Morse Potential, which is softer than the LJ potential. Meanwhile, \citeonline{shinoda2007} used different forms of the Mie potential. They concluded that a 12-4 Mie Potential was ideal for the all water cross interactions and  a 9-6 Mie Potential was suited for all the solute-solute interactions. 

Outside of the Martini framework, \citeonline{shinoda2010} studied different levels of coarse-graining for water ranging for one to 4 molecules per bead using different Mie and Morse potentials. Other works also assessed the use of Soft-core potentials to study aqueous solutions of surfactants \cite{shinoda2007}, ionic liquids \cite{bhargava2009}, lipids \cite{shinoda20102} and membranes \cite{pantano2009}. Other CG force field for water based on the Mie Potential is the SAFT-$\gamma$ Mie \cite{lobanova2015}. In this model, the water molecule can be represented by two different one isotropic bead interacting via a 8-6 Mie Potential models. The CGW1-vle model was parametrized using saturated-liquid density and vapor pressure data, and should be used for simulations of aqueous systems' fluid-phase equilibria at high temperatures and pressures. This model still suffers from premature freezing with a triple point at 343 K. The other model, CGW1-ift, was parametrized using saturated-liquid density and vapor-liquid interfacial tension, hence it is best suited for interfacial properties calculations. Both models have temperature-dependent size and energy parameters and performed well for these properties over the entire temperature range of the liquid. The SAFT-$\gamma$ Mie force field have also been applied to other compounds with satisfactory results. \citeonline{muller2017} parametrized the force field for aromatic compounds and tested it with simulations of fluid phase equilibrium. \citeonline{herdes2015} carried out simulations of alkanes and light gases with this force fields. Binary and ternary mixtures of water, carbon-dioxide and water \cite{lobanova2016}, thermodynamic and transport properties of carbon dioxide and methane \cite{cassiano1,cassiano2} and water/oil interfacial tension \cite{herdes2017} were also studied with this force field.   

\section{Solvation Free Energy}


