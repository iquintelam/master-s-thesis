% Chapter Template

\chapter{Literature Review} % Main chapter title

\label{Chapter2} % Change X to a consecutive number; for referencing this chapter elsewhere, use \ref{ChapterX}
\section{Molecular Simulations of Molecules Mimicking Asphaltenes}
Asphaltenes consist of polyfunctional molecules, and they are defined by their solubility: insoluble in n-alkanes (pentane, hexane and heptane) and solube in tolune. Due to uncertainties related to its structures, a lot work has been done to develop model compounds that have well defined structure and can represent an average asphaltene. The two category of models presented in the literature are the archipelago and continental models. In the archipelago, asphaltenes consist of polyaromatic parts linked together by aliphatic or naphthemic moieties and, in the continental, they consist of a single
polyaromatic ring with linked aliphatic or naphthenic chains \cite{doi:10.1021/ef900975e,doi:10.1080/0892702031000148762}. Studies with molecular simulation uses this models or compounds, such as like toluene and pyrene, that have similar solubility properties. The model's structure as chemical bonding  can cause high energies during the simulation and, consequently, low relative occurrence probabilities \cite{doi:10.1021/ef200507c} .   

\citeonline{ERVIK2016576}  obtained correct interfacial orientation of asphaltenes using coarse grained molecular dynamics simulations of the interface, with an accurate model for the asphaltene molecules. Using a coarse grained force field also, \citeonline{doi:10.1021/ef502209j} carried molecular simulations with a continental asphaltene model. The results reproduced experimental data if the strong aggregation
of asphaltene molecules in n-heptane and high solubility in toluene. Meanwhile, \citeonline{doi:10.1021/ef5020428} performed a molecular
dynamics study with the force field GROMOS 45a3 force field to identify the structural features of different asphaltene molecules (C5 Pe and
anionic C5 Pe). The archipelago model was employed in the work of \citeonline{doi:10.1021/ef301610q} to investigate interfacial behavior of asphaltene molecules at the oil - water interface using molecular dynamics simulations with the OPLS-AA force field. They found that  asphaltenes are preferably distributed in the oil phase in the case of pure toluene and at the oil - water interface
in the case of pure heptane. A molecular
oscillation behavior of asphaltene molecules at the oil - water interface was also discovered in the shape of a nanoscale aggregate.

A pyrelene based model was used in the work of \citeonline{doi:10.1021/jp3010184,doi:10.1021/jp407363p} to study molecular association and interaction as well as the adsorption properties of the pyrelene model at the water/toluene or heptane interface. Molecular dynamics simulations were also used to the study the nanoaggregation of four types of model asphaltene molecules in binary mixtures of toluene and water \cite{doi:10.1021/ef9004576}. This study observed that  in thin films of toluene trapped between two aqueous phases,
both interface-bound and core-bound asphaltenes have similar diffusion behavior. \citeonline{doi:10.1021/acs.energyfuels.6b02161} reported molecular dynamics simulations of four model asphaltenes. They aver that there is no formation of nanoaggregates, and the distribution of clusters is continuous for mixtures of asphaltene in heptane. Generally the works on molecular simulation with asphaltenes models showed a difference in
packing tendencies according to the model \cite{doi:10.1080/10298436.2011.575141}. For a thorough review of studies of different models of asphaltenes, the reader is referred elsewhere \cite{doi:10.1080/10298436.2011.575141}. 


\section{Coarse Grained Force Fields}


Molecular simulations can be carried out at different levels of description. The detailed atomistic level or \textit{ab initio} level is described by the laws of quantum mechanics. The system consists of a set of subatomic particles in which Schrodinger's equation is solved for all of them. The next level is the atomistic description. It considers that the system is made up of atoms following the laws of statistical mechanics.  Force fields at this level are based on pair potentials with Coulombic charged sites. Contributions due to intramolecular interactions such as bond-stretching, angle-bending and torsion are also usually accounted by these kinds of force fields. When the simulation scale needs to be increased and the atomistic simulations become too computationally expensive, the coarse-grained (CG) description is more suited. It considers that the system is made up of pseudo atoms or beads that contain multiple atoms or even an entire molecule. 

There is an obvious loss of information in grouping atoms, hence it is necessary to assure that the process of eliminating unnecessary or unimportant information ('coarse graining') doesn't affect the system's physical behavior. Ideally, coarse grained models need to have accuracy, transferability, robustness, and computational efficiency. In order to achieve these goals, coarse grained force fields are usually developed by mapping the atomistic model in order to define the pseudo atoms, which are generally formed by similar functional groups. The level of coarse graining also needs to be defined, up to 6 heavy atoms (non-hydrogen atoms) per bead in order to not lose important details and maintain isotropic representations of the beads \cite{shinoda2007,martini2007,hadley2012}. CG force fields can be parametrized following two different approaches, bottoms up and top down, to link the simulations on the coarse grained scale to a more detailed scale according to the schematic representation in \figref{fig:multiscale}. The bottoms up approach uses information of a more detailed scale such as the \textit{ab initio} description or the atomistic description to obtain information necessary to the parametrization. This method highly depends on the detailed model quality to succeed. Meanwhile, the top down methodology obtains parameters from larger scales, namely experimental thermodynamic properties or native-structure based properties. 

\begin{figure}{H}
	\centering
	\includegraphics[width=0.8\linewidth]{Figures/multiscale}
	\caption{Schematic representation for the two different approaches of coarse graining.Taken from \citeonline{tatyana}}
	\label{fig:multiscale}
\end{figure}
\FloatBarrier

One of the first applications of coarse grained models is the study of protein folding \cite{levitt1975,levitt1976}. These earlier protein CG models were based on molecule structure, and they contributed for the knowledge of physicochemical forces associated with protein folding and protein interactions \cite{koga2001}.  More recent, models focused on retaining the protein's chemical specificity. The Bereau and Deresmo model \cite{bereau2009} has up to four-bead representation and was used in studies of protein folding and aggregation. However, this model still needs tuning to improve protein stability \cite{bereau2010}. The OPEP (Optimized Potential for Efficient Protein Structure Prediction) model \cite{opep2014,opep2015} has up to six-bead representation. It was used to investigate a variety of phenomenon, ranging from protein folding to \textit{ab initio} peptide structure prediction \cite{opep2011,opep2009,opep20092}. Another CG protein models used in the literature are the Scorpion (solvated coarse-grained protein interaction)  \cite{scorpion2013}, the UNRES (united residue) \cite{unres2014} and the MARTINI model \cite{martini2013}. The later one is the most popular model for CG modeling of membrane proteins \cite{martini20132}. The MARTINI model is also extensively used as CG model for water. This model represents four water molecules as one bead using a shifted Lennard Jones potential for non bonded interactions. Though its extensive use, the MARTINI water model doesn't properly represent properties as interfacial tension and compressibility \cite{shinoda2010} and can freeze at room temperature \cite{winger2009,martini2007}, what requires the use of anti-freeze agents during the simulations. This behavior can be explained by the high level of coarse graining (4:1), the lack of explicit charges and the use of a 12-6 potential. \citeonline{chiu2010} used the Morse Potential, which is softer than the LJ potential, to improve the MARTINI model. Meanwhile, \citeonline{shinoda2007} used different forms of the Mie potential. They concluded that a 12-4 Mie Potential was ideal for water cross interactions and  a 9-6 Mie Potential was suited for solute-solute interactions. 

Outside of the Martini framework, \citeonline{shinoda2010} studied different levels of coarse-graining for water ranging for one to 4 molecules per bead using different Mie and Morse potentials. Works also assessed the use of Soft-core potentials to study aqueous solutions of surfactants \cite{shinoda2007}, ionic liquids \cite{bhargava2009}, lipids \cite{shinoda20102} and membranes \cite{pantano2009}. Another CG force field for water based on the Mie Potential is the SAFT-$\gamma$ Mie \cite{lobanova2015}. In this strategy, there are two different models: the CGW1-vle and the CGW1-ift. Both of them represent the water molecule as one bead and  the Mie Potential has a repulsive and attractive parameter equal to eight and six, respectively. The CGW1-vle model was parameterized using saturated-liquid density and vapor pressure data, and should be used for simulations of aqueous systems' fluid-phase equilibrium at high temperatures and pressures. This model still suffers from premature freezing with a triple point at 343 K. The other model, CGW1-ift, was parameterized using saturated-liquid density and vapor-liquid interfacial tension, hence it is best suited for interfacial properties calculations. Both models have temperature-dependent size and energy parameters and performed well for these properties over the entire liquid temperature range. The SAFT-$\gamma$ Mie force field have also been applied to other compounds with satisfactory results. \citeonline{muller2017} parameterized the force field for aromatic compounds and tested it with simulations of fluid phase equilibrium. \citeonline{herdes2015} carried out simulations of alkanes and light gases. Binary and ternary mixtures of water, carbon-dioxide and water \cite{lobanova2016}, thermodynamic and transport properties of carbon dioxide and methane \cite{cassiano1,cassiano2} and water/oil interfacial tension \cite{herdes2017} were also studied with this force field.  



%----------------------------------------------------------------------------------------
%	SECTION 2
%----------------------------------------------------------------------------------------
\section{Solvation Free Energies}

Solvation free energy calculations with molecular dynamics have a variety of applications due the amount of information it can provide about the behavior of the solvent in different chemical environments and the influence of the solute's molecular geometry. Free energy calculations have an inherent complexity that attracted their study in order to  improve free energy simulations and post processing methods \cite{mbar,bareva,dexp,gdel} in the last decades.

Recent work \cite{PMID:24928188,mobley2017} made available a big database of hydration free energy of small molecules using the GAFF force field. \citeonline{Beckstein2014} also calculated the hydration free energies for 52 compounds with OPLS-AA force field. They obtained an overall root mean square deviation of the prediction from the experimental data of 1.75 kcal/mol and concluded that the precision of the results are mainly limited by the reproducibility of the Lennard-Jones contribution towards the solvation free energy. A comparison of polar and nonpolar contributions to these hydration free energy indicated the significance of each the terms  \cite{izairi2017}. \citeonline{garrido,garrido2011} calculated the free energy of solvation of large alkanes in 1-octanol and water with three different force fields (TraPPE, Gromos, OPLS-AA/TraPPE) and the solvation free energy of propane and benzene in non aqueous solvents like n-hexadecane, n-hexane, ethyl benzene and acetone  with the force fields TraPPE-UA and TraPPE-AA. \citeonline{roy2017} addressed the choice of the Lennard Jones parameters for predicting solvation free energy in 1-octanol. \citeonline{goncalves} calculated the free energy of solvation in the solvents tetrachloride, chloroform and benzene with GROMOS force field. Using the GAFF and the polarizable AMOEBA force fields, \citeonline{mohamed2016} evaluated the solvation free energy of small molecules in toluene, chloroform and acetonitrile, and obtained a mean unsigned error of 1.22 kcal/mol for AMOEBA and 0.66 kcal/mol for GAFF. In order to define the role of solvent water in the docking structure determination, \citeonline{MATUBAYASI201745} developped a method to compute the solvation free energy of proteins while using OPLS-AA force field for the
solutes and TIP3P for water.

Though these variety of data using the intramolecular Lennard-Jones potential, we are not aware of works using the Mie Potential\cite{MIE} in free energy calculations. We, at this study, try to provide information about theses calculations with the SAFT-$\gamma$ Mie coarse grained force field.  The output of these calculations are highly dependent on the force field and deficiencies in the description of small molecules by the models can be revealed with these calculations \cite{mobley2007,shirts2013}. Other important reason for testing a coarse grained force field is that they can generally reproduce free energy difference since the effects of the degrees of freedom reduction  in the entropy are counterbalanced by reduced enthalpic terms \cite{kmiecik2016}. Hence, knowing if more coarse grained approaches have a similar performance to the all atoms force fields can help increase the scale of solvation free energy calculations. 