%% abtex2-modelo-trabalho-academico.tex, v-1.9.6 laurocesar
%% Copyright 2012-2016 by abnTeX2 group at http://www.abntex.net.br/ 
%%
%% This work may be distributed and/or modified under the
%% conditions of the LaTeX Project Public License, either version 1.3
%% of this license or (at your option) any later version.
%% The latest version of this license is in
%%   http://www.latex-project.org/lppl.txt
%% and version 1.3 or later is part of all distributions of LaTeX
%% version 2005/12/01 or later.
%%
%% This work has the LPPL maintenance status `maintained'.
%% 
%% The Current Maintainer of this work is the abnTeX2 team, led
%% by Lauro César Araujo. Further information are available on 
%% http://www.abntex.net.br/
%%
%% This work consists of the files abntex2-modelo-trabalho-academico.tex,
%% abntex2-modelo-include-comandos and abntex2-modelo-references.bib
%%

% ------------------------------------------------------------------------
% ------------------------------------------------------------------------
% abnTeX2: Modelo de Trabalho Academico (tese de doutorado, dissertacao de
% mestrado e trabalhos monograficos em geral) em conformidade com 
% ABNT NBR 14724:2011: Informacao e documentacao - Trabalhos academicos -
% Apresentacao
% ------------------------------------------------------------------------
% ------------------------------------------------------------------------

\documentclass[
	% -- opções da classe memoir --
	12pt,				% tamanho da fonte
	openany,			% capítulos começam em pág ímpar (insere página vazia caso preciso)
	oneside,			% para impressão em recto e verso. Oposto a oneside
	a4paper,			% tamanho do papel. 
	% -- opções da classe abntex2 --
	%chapter=TITLE,		% títulos de capítulos convertidos em letras maiúsculas
	%section=TITLE,		% títulos de seções convertidos em letras maiúsculas
	%subsection=TITLE,	% títulos de subseções convertidos em letras maiúsculas
	%subsubsection=TITLE,% títulos de subsubseções convertidos em letras maiúsculas
	% -- opções do pacote babel --
	english,			% idioma adicional para hifenização
	brazil				% o último idioma é o principal do documento
	]{abntex2}

% ---
% Pacotes básicos 
% ---
%\usepackage{lmodern}			% Usa a fonte Latin Modern			
\usepackage[T1]{fontenc}		% Selecao de codigos de fonte.
\usepackage[utf8]{inputenc}		% Codificacao do documento (conversão automática dos acentos)
\DeclareUnicodeCharacter{2010}{-}
\usepackage{lastpage}			% Usado pela Ficha catalográfica
\usepackage{indentfirst}		% Indenta o primeiro parágrafo de cada seção.
\usepackage{color}				% Controle das cores
\usepackage{graphicx}			% Inclusão de gráficos
\usepackage{microtype} 			% para melhorias de justificação
% ---
\usepackage{palatino} % Use the Palatino font by default
\usepackage{amsmath}
\usepackage{nicefrac}
\usepackage{nameref}
\usepackage{placeins}
\usepackage{float}
\usepackage{pdfpages}
\DisemulatePackage{setspace}
\usepackage{setspace} % espacamento entre linhas
\let\added\undefined      % neutralize \added command
\let\deleted\undefined    % neutralize \deleted command
\usepackage[markup=bfit]{changes}
\definechangesauthor{iq}
\setremarkmarkup{(#2)}
% padrao 1.5 de espacamento entre linhas



%\usepackage[portuguese]{babel}
% ---
% Pacotes adicionais, usados apenas no âmbito do Modelo Canônico do abnteX2
% ---
% ---
\usepackage{chngcntr}
\counterwithin{figure}{section}
\usepackage[justification=centering]{caption}
% ---
% Pacotes de citações
% ---
%\usepackage[brazilian,hyperpageref]{backref}	 % Paginas com as citações na bibl
%\citeoption{abnt-substyle=COPPE}
\usepackage[alf,abnt-etal-list=0,abnt-etal-text=it,abnt-doi=doi]{abntex2cite}	% Citações padrão ABNT
% --- 
% CONFIGURAÇÕES DE PACOTES
% --- 

% ---
% Configurações do pacote backref
% Usado sem a opção hyperpageref de backref
%\renewcommand{\backrefpagesname}{Citado na(s) página(s):~}
% Texto padrão antes do número das páginas
%\renewcommand{\backref}{}
% Define os textos da citação
%\renewcommand*{\backrefalt}[4]{
%	\ifcase #1 %
%		Nenhuma citação no texto.%
%	\or
%		Citado na página #2.%
%	\else
%		Citado #1 vezes nas páginas #2.%
%	\fi}%
%% ---

% ---
% Informações de dados para CAPA e FOLHA DE ROSTO
% ---
\titulo{Solvation Free Energy Calculations of Molecules Mimicking Asphaltenes Using The SAFT-\boldmath$\gamma$ Mie Force Field}
\autor{Isabela Quintela Matos}
\local{Rio de Janeiro}
\data{2018}
\orientador{Charlles Rubber de Almeida Abreu}
\coorientador{Papa Matar Ndiaye}
\instituicao{%
  Universidade Federal do Rio de Janeiro
  \par
  Escola de Química 
  \par
  Engenharia de Processos Químicos e Bioquímicos Graduate Program}
\tipotrabalho{Dissertation (Master)}
% O preambulo deve conter o tipo do trabalho, o objetivo, 
% o nome da instituição e a área de concentração 
\preambulo{Masther's thesis presented to Engenharia de Processos Químicos e Bioquímicos graduate program, Escola de Química,
Universidade Federal do Rio de Janeiro, as
required for obtaining a Master's degree
in Chemical Engineering.}
% ---


% ---
% Configurações de aparência do PDF final

% alterando o aspecto da cor azul
\definecolor{blue}{RGB}{41,5,195}

% informações do PDF
\makeatletter
\hypersetup{
     	%pagebackref=true,
		pdftitle={\@title}, 
		pdfauthor={\@author},
    	pdfsubject={\imprimirpreambulo},
	    pdfcreator={LaTeX with abnTeX2},
		pdfkeywords={abnt}{latex}{abntex}{abntex2}{trabalho acadêmico}, 
		colorlinks=true,       		% false: boxed links; true: colored links
    	linkcolor=blue,          	% color of internal links
    	citecolor=blue,        		% color of links to bibliography
    	filecolor=magenta,      		% color of file links
		urlcolor=blue,
		bookmarksdepth=4
}
\makeatother
% --- 
\newcommand{\figref}[2][{}]{\hyperref[#2]{\figurename~\ref{#2}#1}} 

% --- Use it as \figref{label} for figures that contain only one plot, and use it as \figref[A]{label} for those figures for which you want the reference to be Figure 1A.
% Espaçamentos entre linhas e parágrafos 
% --- 

% O tamanho do parágrafo é dado por:
\setlength{\parindent}{1.3cm}

% Controle do espaçamento entre um parágrafo e outro:
\setlength{\parskip}{0.2cm}  % tente também \onelineskip

% ---
% compila o indice
% ---
\makeindex
% ---
\makeatletter
\renewcommand*\l@figure{\@dottedtocline{1}{1em}{3.2em}}
\makeatother
% ----
% Início do documento
% ----
\begin{document}

% Seleciona o idioma do documento (conforme pacotes do babel)
\selectlanguage{english}
%\selectlanguage{brazil}

% Retira espaço extra obsoleto entre as frases.
\frenchspacing 

% ----------------------------------------------------------
% ELEMENTOS PRÉ-TEXTUAIS
% ----------------------------------------------------------
% \pretextual

% ---
% Capa
% ---
\imprimircapa
% ---

% ---
% Folha de rosto
% (o * indica que haverá a ficha bibliográfica)
% ---
\imprimirfolhaderosto*
% ---

% ---
% Inserir a ficha bibliografica
% ---

% Isto é um exemplo de Ficha Catalográfica, ou ``Dados internacionais de
% catalogação-na-publicação''. Você pode utilizar este modelo como referência. 
% Porém, provavelmente a biblioteca da sua universidade lhe fornecerá um PDF
% com a ficha catalográfica definitiva após a defesa do trabalho. Quando estiver
% com o documento, salve-o como PDF no diretório do seu projeto e substitua todo
% o conteúdo de implementação deste arquivo pelo comando abaixo:
%
% \begin{fichacatalografica}
%     \includepdf{fig_ficha_catalografica.pdf}
% \end{fichacatalografica}

\begin{fichacatalografica}
	\sffamily
	\vspace*{\fill}					% Posição vertical
	\begin{center}					% Minipage Centralizado
	\fbox{\begin{minipage}[c][8cm]{13.5cm}		% Largura
	\small
	\imprimirautor
	%Sobrenome, Nome do autor
	
	\hspace{0.5cm} \imprimirtitulo  / \imprimirautor. --
	\imprimirlocal, \imprimirdata-
	
	\hspace{0.5cm} \pageref{LastPage} \\
	
	\hspace{0.5cm} \imprimirorientadorRotulo~\imprimirorientador\\
	
	\hspace{0.5cm}
	\parbox[t]{\textwidth}{\imprimirtipotrabalho~--~\imprimirinstituicao,
	\imprimirdata.}\\
	
	\hspace{0.5cm}
		1. Solvation free energy.  
		2. Asphaltenes.
		3. SAFT-$\gamma$ Mie force field.
%		I. Orientador.
%		II. Universidade xxx.
%		III. Faculdade de xxx.
%		IV. Título 			
	\end{minipage}}
	\end{center}
\end{fichacatalografica}
% ---

% ---
% Inserir errata
% ---
%\begin{errata}
%Elemento opcional da \citeonline[4.2.1.2]{NBR14724:2011}. Exemplo:
%
%\vspace{\onelineskip}
%
%FERRIGNO, C. R. A. \textbf{Tratamento de neoplasias ósseas apendiculares com
%reimplantação de enxerto ósseo autólogo autoclavado associado ao plasma
%rico em plaquetas}: estudo crítico na cirurgia de preservação de membro em
%cães. 2011. 128 f. Tese (Livre-Docência) - Faculdade de Medicina Veterinária e
%Zootecnia, Universidade de São Paulo, São Paulo, 2011.
%
%\begin{table}[htb]
%\center
%\footnotesize
%\begin{tabular}{|p{1.4cm}|p{1cm}|p{3cm}|p{3cm}|}
%  \hline
%   \textbf{Folha} & \textbf{Linha}  & \textbf{Onde se lê}  & \textbf{Leia-se}  \\
%    \hline
%    1 & 10 & auto-conclavo & autoconclavo\\
%   \hline
%\end{tabular}
%\end{table}
%
%\end{errata}
% ---

% ---
% Inserir folha de aprovação
% ---

% Isto é um exemplo de Folha de aprovação, elemento obrigatório da NBR
% 14724/2011 (seção 4.2.1.3). Você pode utilizar este modelo até a aprovação
% do trabalho. Após isso, substitua todo o conteúdo deste arquivo por uma
% imagem da página assinada pela banca com o comando abaixo:
%
% \includepdf{folhadeaprovacao_final.pdf}
%
\begin{folhadeaprovacao}

  \begin{center}
    {\ABNTEXchapterfont\large\imprimirautor}

    \vspace*{\fill}\vspace*{\fill}
    \begin{center}
      \ABNTEXchapterfont\bfseries\Large\imprimirtitulo
    \end{center}
    \vspace*{\fill}
    
    \hspace{.45\textwidth}
    \begin{minipage}{.5\textwidth}
        \imprimirpreambulo
    \end{minipage}%
    \vspace*{\fill}
   \end{center}
        
   Approved dissertation. \imprimirlocal, \today:

   \assinatura{\textbf{Charlles Rubber de Almeida Abreu, D.Sc. -EPQB/UFRJ} \\ Supervisor} 
   \assinatura{\textbf{Frederico Wanderley Tavares, D.Sc. - EPQB/UFRJ} \\ Guest 1}
   \assinatura{\textbf{Bruno Araujo Cautiero Horta, D.Sc. - PGQU/UFRJ} \\ Guest 2}
   \assinatura{\textbf{Luciano Tavares da Costa, D.Sc. - PPGQ/UFF} \\ Guest 3}
      
   \begin{center}
    \vspace*{0.5cm}
    {\large\imprimirlocal}
    \par
    {\large\imprimirdata}
    \vspace*{1cm}
  \end{center}
  
\end{folhadeaprovacao}
% ---

% ---
% Dedicatória
% ---
%\begin{dedicatoria}
%   \vspace*{\fill}
%   \centering
%   \noindent
%   \textit{ Este trabalho é dedicado às crianças adultas que,\\
%   quando pequenas, sonharam em se tornar cientistas.} \vspace*{\fill}
%\end{dedicatoria}
% ---

% ---
% Agradecimentos
% ---
\begin{agradecimentos}
I would like to give some words of thanks to my parents for the support during the movie from Aracaju to Rio de Janeiro, and for supporting my decisions. I also would like to thank my advisers. Charlles Rubber de Almeida Abreu for the elucidative meetings and for taking the time to explain in such clearly way the concepts of molecular simulations and free energy calculations. To my co adviser Papa Matar Ndiaye, I would like to thank him for presenting the ATOMS group to me and for accepting my participation in his project about asphaltenes. I also would like to thank all the ATOMS group members, specially Ana Jorgelina Silveira  for helping me with any theoretical and software doubts I had. Finally, I would like to thank COPPETEC foundation for the financial support. 
\end{agradecimentos}
% ---

% ---
% Epígrafe
% ---
%\begin{epigrafe}
%    \vspace*{\fill}
%	\begin{flushright}
%		\textit{``Não vos amoldeis às estruturas deste mundo, \\
%		mas transformai-vos pela renovação da mente, \\
%		a fim de distinguir qual é a vontade de Deus: \\
%		o que é bom, o que Lhe é agradável, o que é perfeito.\\
%		(Bíblia Sagrada, Romanos 12, 2)}
%	\end{flushright}
%\end{epigrafe}
% ---

% ---
% RESUMOS
% ---

% resumo em português
\setlength{\absparsep}{18pt} % ajusta o espaçamento dos parágrafos do resumo
%\begin{resumo}
% Segundo a \citeonline[3.1-3.2]{NBR6028:2003}, o resumo deve ressaltar o
% objetivo, o método, os resultados e as conclusões do documento. A ordem e a extensão
% destes itens dependem do tipo de resumo (informativo ou indicativo) e do
% tratamento que cada item recebe no documento original. O resumo deve ser
% precedido da referência do documento, com exceção do resumo inserido no
% próprio documento. (\ldots) As palavras-chave devem figurar logo abaixo do
% resumo, antecedidas da expressão Palavras-chave:, separadas entre si por
% ponto e finalizadas também por ponto.
%
% \textbf{Palavras-chave}: latex. abntex. editoração de texto.
%\end{resumo}

% resumo em inglês
\begin{resumo}[Abstract]
 \begin{otherlanguage*}{english}
  We, at this work, studied the solvation free energy differences of molecules mimicking asphaltenes in different solvents with the SAFT-$\gamma$ Mie force field. We obtained solvation free energy differences by carrying out molecular dynamics simulations at the expanded ensemble. The output from these simulations was then used to estimate the differences with the MBAR method. The results with solvents other than water had low absolute deviations to the experimental data. Meanwhile, the hydration free energy calculations required a binary interaction parameter estimated with output data from molecular dynamics in order to obtain accurate free energy differences. These results indicated some problems on the SAFT-$\gamma$ Mie model for water, but, generally, proved that this coarse-grained model could represent the free energy differences of the studied sets of solute-solvent.

   \vspace{\onelineskip}
 
   \noindent 
   \textbf{Keywords}: solvation free energy. asphaltenes. SAFT-$\gamma$ Mie force field.
 \end{otherlanguage*}
\end{resumo}

% ---
% inserir lista de ilustrações
% ---
\pdfbookmark[0]{\listfigurename}{lof}
\listoffigures*
\cleardoublepage
% ---

% ---
% inserir lista de tabelas
% ---
\pdfbookmark[0]{\listtablename}{lot}
\listoftables*
\cleardoublepage
% ---

% ---
% inserir lista de símbolos
% ---
\begin{simbolos}
  \item[$T$] Temperature
  \item[$P$] Pressure
  \item[$V$] Volume
  \item[$t$] Time
  \item[$p$] Momentum
  \item[$r$] Coordinates
  \item[$U,u$] Potential Energy
  \item[$m$] mass
  \item[$v$] velocity
  \item[$P$] Pressure
  \item[$ \mathcal{H} $] Hamiltonian
  \item[$q$] Generalized Coordinates
  \item[$K $] 	Kinetic Energy 
  \item[$N$] Number of atoms/molecules
  \item[$h$] Planck Constant
  \item[$S$] Entropy
  \item[$\kappa_{b}$] Boltzmann Constant
  \item[$\beta$] $1/\kappa_{b}T$
  \item[$\mu$] Chemical Potential
  \item[$A$] Helmholtz Free Energy
  \item[$G$] Gibbs Free Energy
  \item[$f$] Free Energy
  \item[$F$] Forces
  \item[$\epsilon$] Depth of the potential well
  \item[$\sigma$] Distance correspondent to a zero intermolecular potential
  \item[$\lambda _{r}$] Repulsive exponent
  \item[$\lambda _{a}$] Attractive exponent
  \item[$x_{i}$] Molar fraction 
  \item[$w_{i}$] Weight fraction 
  \item[$\rho$] Density
  \item[$\lambda$] Coupling Parameter 
  \item[$\eta$] Arbitrary Weight 
  \item[$k_{ij}$] Binary Interaction Parameter
\end{simbolos}
% ---

% ---
% inserir o sumario
% ---
\pdfbookmark[0]{\contentsname}{toc}
\tableofcontents*
\cleardoublepage
% ---



% ----------------------------------------------------------
% ELEMENTOS TEXTUAIS
% ----------------------------------------------------------



% ----------------------------------------------------------
% Introdução (exemplo de capítulo sem numeração, mas presente no Sumário)
% ----------------------------------------------------------
\clearpage
\setcounter{page}{1}
\chapter{Introduction} % Main chapter title
\label{Chapter1} % 
\pagestyle{simple}
\onehalfspacing
Solvation free energy calculations with molecular dynamics (MD) have a variety of applications ranging from drug design in the pharmaceutical industry to development of separation technologies in the chemical industry. Solvation free energy is, more specifically, the Gibbs free energy difference between the solute alone in the gas phase and the solute interacting with the solvent. Through the study of the solvation phenomenon, it is possible to obtain information about the behavior of the solvent in different chemical environments and the influence of the solute's molecular geometry. It is also possible to calculate other important properties with the solvation free energy, namely the activity coefficient at infinite dilution, Henry constant, partition coefficients and solubility. 

The solvation phenomenon is intrinsically complex. There are many competing forces interfering in the behavior of the solute-solvent interaction and free energy simulations are susceptible to sampling problems in high energy regions. With the intention of improving free energy calculations, simulation methodologies such as the expanded ensemble \cite{lyubartsev}, thermodynamic integration \cite{kirkwood1935}, free energy perturbation \cite{zwanzig1954,bennet1976,mbar} and umbrella sampling \cite{TORRIE1977187} have been developed in order to obtain accurate estimations for the energy differences. 

Another influencing factor in the output of these calculations are the force fields chosen to describe the solvent and solute molecules. Force fields have different levels of description (quantum mechanics, atomistic, coarse grained). In the coarse grained description, molecules are grouped in pseudo atoms or beads. Coarse grained models generally reproduce free energy differences since the effects of reducing degrees of freedom in the entropy are counterbalanced by the reduction of enthalpic terms \cite{kmiecik2016}. Additionally, the success of a coarse grained force field is important to increase the scale of solvation free energy calculations and reveal deficiencies in the description of small molecules by these models \cite{mobley2007,shirts2013}. That's the reason we, at this study, try to provide information about free energy calculation on the expanded ensemble with the SAFT-$\gamma$ Mie coarse grained force field. This force field uses the Mie Potential \cite{MIE} and has a more straightforward method of obtaining its parameters than other models. It was initially parameterized with pure component equilibrium and interfacial tension data. This strategy has provided satisfactory results for prediction of phase equilibrium of aromatic compounds, alkanes, light gases and water \cite{herdes2015,muller2017,lobanova2015} , thermodynamic properties of carbon dioxide and methane \cite{cassiano1}, multiphase equilibrium of mixtures of water, carbon dioxide and n-alkanes \cite{lobanova2016} and water/oil interfacial tension \cite{herdes2017}. 

The solvents and solutes chosen in this study range from standard sets used as benchmark in solvation free energy calculations to ring substances used as a model to asphaltenes. Asphaltenes are complicated to characterize by determining their composition on a molecular basis, but it is generally accepted that they can be characterized as a fraction of crude oil soluble in toluene and insoluble in n-alkenes (pentane, hexane, heptane) \cite{SJOBLOM2003399}. They've motivated many studies interested in developing models for their structure due to all the problems they can cause during their transportation and refining \cite{SJOBLOM20151}. This work’s strategy to test the efficiency of the SAFT-$\gamma$ Mie force field in describing the solvation phenomenon was to use molecules that appear in asphaltenes models and that have similarities in terms of solubilities with asphaltenes (phenanthrene, anthracene, pyrene) for the solutes. Meanwhile, for the solvents, we chose compounds that are used to characterize asphaltenes (toluene, hexane). The anti solvent/solvent effect of carbon dioxide was also tested due to its influence in asphaltene precipitation during the oil processing \cite{SOROUSH2014405}. Asphaltenes are described by their solubility, hence simulation scale can be increased and studies with more complete models can be carried out if this coarse grained model can correctly describe solvation free energy differences of simple molecules mimicking asphaltenes.


% Chapter Template

\chapter{Literature Review} % Main chapter title

\label{Chapter1} % Change X to a consecutive number; for referencing this chapter elsewhere, use \ref{ChapterX}

%----------------------------------------------------------------------------------------
%	SECTION 1
%----------------------------------------------------------------------------------------

\section{Coarse Grained Force Fields}

Molecular simulations can be carried out at different levels of descriptions. The detailed atomistic level or \textit{ab initio}level is described by the laws of quantum mechanics. The system consists of a set of subatomic particulars in which Schrodinger's equation is solved for all of them. The next level is the atomistic description. It considers that the system is made up of atoms following the laws of statistical mechanics.  Force fields at this level are based on pair potentials with Coulombic charged sites, which account for the molecular interactions. The contributions due to to intramolecular interactions like bond-stretching, angle-bending and torsion are also usually accounted by these kind of force fields. When the scale of the simulations needs to be increased and the atomistic simulations become too computationally expensive, the coarse-grained (CG) description is more suited. It considers that the system is made up of pseudo atoms or beads that contain multiple atoms. 

There is a obvious loss of information in grouping atoms, hence it is necessary to assure that the process of eliminating unnecessary or unimportant information ('coarse graining') doesn't affect the system's physical behavior. The coarse grained force fields based on this description are developed by mapping the atomistic model to define the pseudo atoms. This mapping is normally done by grouping similar funcional groups. The level of coarse-graining also needs to be defined, up to 6 heavy atoms (non-hydrogen atoms) per bead in order to not loose much detail and maintain isotropic representations of the beads \cite{shinoda2007,martini2007,hadley2012}. The CG force field can be parametrized following two different approaches: bottoms up and top down. The bottoms up approach uses information of a more detailed scale such as the \textit{ab initio} description or the atomistic description to obtain the information necessary to the parametrization. This method depends highly of the quality of the detailed model to succeed. Meanwhile, the top down methodology obtains the parameters from one larger scales. This information at larger scales could be experimentally observed data like thermodynamic properties or native-structure based properties. 

%http://aip.scitation.org/doi/pdf/10.1063/1.4818908


The main advantage of
coarse-graining lies in the immense speed up of the simulation

Obviously, coarse graining comes at the cost of loosing electronic and atomistic details.
Therefore it is crucial to identify the unimportant details and to preserve feature that are
essential for the description of the phenomenon on interest. In particular, CG mapping
is critical for the accuracy, transferability, robustness, and computational efficiency of the
CG model.



An early example of coarse-gaining is the seminal
work of Levitt and Warshel in 1975 [90], where the authors studied the folding of small
proteins by obtaining atomistic potentials from the QM trajectories using MD simulations. 



The pair potential is based on molecule specific
parameters, such as the ranges of the repulsive and attractive interactions, the size of
the bead and the energy parameter which characterises the strength of the attractive interaction.

At short distances, the potential is
invariably repulsive and tends to infinity, whereas at larger separations, the potential is
attractive and tends to zero so that the energy remains finite. The total potential energy
is usually assumed as a sum of both, the repulsive and the attractive, contributions. Some
of the commonly used pair potentials are illustrated in Figure 2.1 and briefly described
below.

sing the versatile exponents has been shown to provide
a significant improvement of the vapour pressure and the second-derivative thermodynamic
properties of real fluids, such as speed of sound, heat capacity, and compressibility \cite{avendano2011,lafitte2013,lafitte2006}.

In conventional statistical mechanics, it is common to replace time
averages with ensemble averages. This hypothesis, currently referred to as the ergodic
hypothesis, states that the time and the ensemble averages of two systems with the same tate variables, e.g., N, V , and T are identical in infinite time and thermodynamic limits.

By contrast, commonly used
bottom-up approaches often make use of temperature dependent parameters that have to
be re-derived at various state points. The parameters obtained with our SAFT-gamma top-down
methodology can be used as a direct input in molecular simulation. Unlike intermolecular
interactions can also be obtained by using appropriate experimental data for the mixtures,
but often combining rules are employed for some of the parameters as described in the
next section.

The parameterisation of a force field is a non-trivial and a cumbersome task. Some of
the main requirements on a force field include the accuracy, transferability, and robustness. Unfortunately, there is no generic force field that can reproduce all properties with a
unique parameter set.



mostly the intermolecular interactions and the torsional terms. In this work, our main focus lies on the
dispersion interactions. The dispersion interactions are normally described by the intermolecular potentials presented in Section 2.1.2.

The LJ potential function forms the basis for a large number of widely
used molecular force fields, such as OPLS [73], TraPPE [74], NERD [75] force fields, and
the TIP [76] and SPC [77] models of water, to mention but a few. The popular coarsegrained MARTINI force field [78] is also based on the LJ representation.


 
\chapter{Fundamentals of the Computational Methods} % Main chapter title

\label{Chapter3} % Change X to a consecutive number; for referencing this chapter elsewhere, use \ref{ChapterX}


\section{SAFT-$\gamma$ Mie Force Field}

\subsection{SAFT-VR Mie EoS}

The SAFT-VR Mie equation of state \cite{lafitte2013} is the basis for the SAFT-$\gamma$ Mie coarse grained force field \cite{avendano2011}. This EoS was initially developed to describe chain molecule formed from fused Mie segments using the Mie attractive and repulsive potential. The Mie potential is a type of generalized Lennard-Jones potential that can be used to describe explicitly repulsive interactions of different hardness/softness and attractive interactions of different ranges, and is given by:
\begin{equation}
U_{Mie}(r) = \epsilon\frac{\lambda_r}{\lambda_r - \lambda_a} \left(\frac{\lambda_r}{\lambda_a} \right)^{\left( \frac{\lambda_a}{\lambda_r - \lambda_a} \right)}
\left[ \left(\frac{\sigma}{r} \right)^{\lambda_r} - \left(\frac{\sigma}{r} \right)^{\lambda_a} \right]
\label{eqn:miepotential}
\end{equation}
where $\epsilon$ is the potential well depth, $\sigma$ is the segment diameter, r is the distance between the spherical segments, $\lambda_r$ is the repulsive exponent and $\lambda_a$ is the attractive exponent. This equation uses the \citeonline{bh1976} high perturbation expansion of the Helmholtz free energy up to third order and an improved expression for the  radial distribution function (RDF) of Mie monomers at contact to obtain a equation able to give an accurate theoretical description of the vapor-liquid equilibria and second derivative properties \cite{lafitte2013}. For a non-associating fluid, the Helmholtz free energy is:
\begin{equation}
\frac{A}{N\kappa_{b}T} = a = a^{IDEAL} + a^{MONO} + a^{CHAIN}
\label{eqn:miehelm}
\end{equation}

\subsubsection{Ideal Contribution}

The ideal contribution for a mixture is given by:
\begin{equation}
a^{IDEAL} = \sum_{i=1}^{N_{c}} x_{i}\ln{(\rho_{i}{\Lambda_{i}}^3)} -1
\label{eqn:aideal}
\end{equation}
where $x_{i}=N_{i}/N$ is the molar fraction of component i, $\rho_{i}=N_{i}/V$ is the number density, $N_{i}$ is the number of molecules of each component and $\Lambda_{i}^3$ is de Broglie wavelength. 

\subsubsection{Monomer Contribution}

The monomer contribution describes the interactions between Mie segments and can be expressed for a mixture as:
\begin{equation}
a^{MONO} = \left(\sum_{i=1}^{N_{c}} x_{i}m_{s,i} \right)a^{M}
\label{eqn:amonomer}
\end{equation}

In the equation above, $m_{s,i}$ is the number of spherical segments making up the molecule i and $a^{M}$  is the monomer dimensionless Helmholtz free energy and it is expressed as a third order perturbation expansion in the inverse temperature \cite{bh1976}:
\begin{equation}
a^{M} = a^{HS}+\beta{a_{1}}+\beta{a_{2}}^2+\beta{a_{3}}^3 
\label{eqn:aM}
\end{equation}
where $\beta=\kappa_{b}T$ and $a^{HS}$ is the hard-sphere dimensionless Helmholtz free energy for a mixture :
\begin{equation}
a^{HS} = \frac{6}{\pi\rho_{s}}\left[\left(\frac{\zeta^3_2}{\zeta^2_3}-\zeta_0 \right)\ln(1-\zeta_3)+\frac{3\zeta_{1}\zeta_{2}}{1-\zeta_3}+ \frac{\zeta^3_2}{\zeta_{3}(1-\zeta_3)^2}\right]
\label{eqn:hs}
\end{equation}

The variable $\rho_{s}=\rho\sum_{i}^{N_c} x_{i}m{s,i}$ is the total number density of spherical segments and $\zeta_l$ are the moments of the number density:
\begin{equation}
\zeta_l = \frac{\pi\rho_s}{6}\left(\sum_{i=1}^{N_c} x_{s,i}d^l_{ii} \right), l = 0,1,2,3
\label{eqn:zetal}
\end{equation}
where $x_{s,i}$ is the mole fraction of the segments and is related through the mole fraction of component i ($x_i$) by:
\begin{equation}
x_{s,i} = \frac{m_{s,i}x_i}{\sum_{k=1}^{N_c} m_{s,k}x_{k} }
\label{eqn:xsi}
\end{equation}


The effective hard-sphere diameter $d_{ii}$ for the segments is:
\begin{equation}
d_{ii} =\int_{0}^{\sigma_{ii}} ( 1 - \exp(-\beta U^{Mie}_{ii}(r)) ) dr
\label{eqn:diameter}
\end{equation}


The integral in Eq. \eqref{eqn:diameter} is normally obtained by means of Gauss-Legendre with a 5-point quadrature \cite{papa2014}. The detailing of the terms of Eq. \eqref{eqn:amonomer} can be found in \citeonline{lafitte2013}.

\subsubsection{Chain Contribution}
The chain formation of $m_{s}$ tangentially bonded Mie segments contribution is based on the first-order pertubation theory (TPT1)  \cite{papa2014} and can be expressed as:
\begin{equation}
a^{CHAIN} =-\sum_{i=1}^{N_{c}} x_{i}(m_{s,i} - 1)\ln(g_{ii}^{Mie}(\sigma_{ii}))
\label{eqn:achain}
\end{equation}


The $g_{ij}^{Mie}(\sigma_{ij})$ term correspond to the value of the radial distribution function (RDF) of the hypothetical Mie system evaluated at the effective diameter and can be obtained with the perturbation expansion:
\begin{equation}
\begin{aligned}
g_{ij}^{Mie}(\sigma_{ij}) =g_{d,ij}^{HS}(\sigma_{ij})\exp[\beta\epsilon g_{1,ij}(\sigma_{ij})/g_{d,ij}^{HS}(\sigma_{ij}) + (\beta\epsilon)^{2} g_{2,ij}(\sigma_{ij})/g_{d,ij}^{HS}(\sigma_{ij})]
\end{aligned}
\label{eqn:gmie}
\end{equation}


The other terms in the equations above are explicitly exposed in the original article \cite{lafitte2013}. 

\subsubsection{Ring Contribution}
There are two forms for the Helmholtz free energy for rings formed from $m_{s}$ tangentially bonded segments in the literature. The first one  \cite{lafitte2012} considered that the difference between a chain and a ring molecule is that the latter one has one more bond that is connecting the first segment to the last. With this assumption, the Eq. \eqref{eqn:achain} can be adapted to rings by:
\begin{equation}
a^{RING} =-\sum_{i=1}^{N_{c}} x_{i}m_{s,i}\ln(g_{ii}^{Mie}(\sigma_{ii}))
\label{eqn:aringlafitte}
\end{equation}

According to \citeonline{lafitte2012}, Eq. \eqref{eqn:aringlafitte} needs an additional parametrization with molecular simulation data so the EoS can  be used in molecular simulations, but this procedure is not the necessary for chain molecules. Recently \citeonline{muller2017} tried to correct this inconsistency by means of developing the ring free energy based on the work of \citeonline{muller1993} who obtained rigorous expressions for hard fluids with molecular geometries of rings of $m_s=3$. The final expression developed for the ring dimensionless Helmholtz free energy is:
\begin{equation}
a^{RING} =-\sum_{i=1}^{N_{c}} x_{i}(m_{s,i}-1+\chi_{i}\eta_{i})\ln(g_{ii}^{Mie}(\sigma_{ii}))
\label{eqn:aringmuller}
\end{equation}
$\eta_{i}=m_{s,i}\rho_{i}\sigma_{ii}^{3}/6$ is the packing fraction and $\chi_{i}$ is a parameter which depends on $m_{s,i}$ and on the geometry of the ring of each component i. For a value of $\chi=0$ Eq. \eqref{eqn:aringmuller} is equal to Eq. \eqref{eqn:achain}. Meanwhile, the system corresponds to a hard sphere system of triangles when $\chi=1.3827$. \citeonline{muller2017} also calculated values of $\zeta$ for $m_{s}=3,m_{s}=4,m_{s}=5,m_{s}=7$ with pseudo-experimental data from molecular dynamics (MD) for a defined pure fluid. The values of $\chi$ for each geometry estimated can be seen in the \figref{ringqsi}.
\begin{figure}[th]
\centering
\includegraphics[scale=0.8]{Figures/mullergeo.jpg}
\caption{Values for parameter $\chi$ according to the ring geometry \cite{muller2017}}
\label{ringqsi}
\end{figure}

\subsubsection{Combining rules for the intermolecular potential parameters}
\citeonline{lafitte2013} also suggested mixing rules for the potential parameters based on Lorentz-Berthelot combining rules \cite{rowlinson}:
\begin{equation}
\sigma_{ij} =\frac{\sigma{ii}+\sigma{jj}}{2}
\label{eqn:sigmamix}
\end{equation}
\begin{equation}
\lambda_{k,ij} -3 =\sqrt{(\lambda_{k,ii}-3)(\lambda_{k,jj}-3)} , k=r,a
\label{eqn:lambdamix}
\end{equation}
\begin{equation}
\epsilon_{ij} =(1-k_{ij})\frac{\sqrt{\sigma_{ii}^{3}\sigma_{jj}^{3}}}{\sigma_{ij}^{3}}\sqrt{\epsilon_{ii}\epsilon_{jj}}
\label{eqn:epsmix}
\end{equation}

The $k_{ij}$ is a binary interaction parameter to correct the deviations of the Lorentz-Berthelot rule for chemically distinct compounds. This parameter can also be fitted to experimental data or pseudo experimental data.

\subsection{Parameter Estimation for the SAFT-$\gamma$ Mie Force Field}

The SAFT-$\gamma$ Mie Force Field uses a coarse graining top down methodology in its parameterization. This methodology aims to obtain the intermolecular parameters from macroscopic experimental data like fluid-phase equilibrium or superficial tension data. The idea is that the force field's  parameters estimated with the the SAFT-VR Mie EoS can be used on molecular simulations since both the equation of state and the force field use the same explicit intermolecular potential model (Mie potential). This correspondence between models has already been seem for a variety of fluids in which this force field was parameterized and  this success in the representation of the properties of real fluids can be imputed to the degrees of freedom of Mie Potential \cite{herdes2015}. This flexibility also provides an exploration of a very large parameter space without using a iterative simulation scheme \cite{avendano2011}. 

Each substance has initially five parameters to be estimated ($m_s$,$\sigma$,$\epsilon$,$\lambda_{r}$ and $\lambda_{a}$) according to Eq. \eqref{eqn:miepotential}. The number of segments are usually fixed in an integer value so it can be used in the coarse grained simulations. The attractive parameter can also be fixed since there is a high correlation between the attractive and repulsilve parameter. Usually, the parameter is fixed in the London value of 6, which is expected to be a good representation of the dispersion scale of most simple fluids that don't have strong polar interactions \cite{ramrattan2015,herdes2015}. There are two strategies to obtain the parameters of each substance: one is by fitting the Saft-Vr Mie EoS to experimental data as vapor pressure and liquid density and the other is using correspondent state parametrization. The first one, generally, minimizes the following unweighted least-squares objective function:

\begin{equation}
\begin{aligned}
\min\limits_{\sigma,\epsilon,\lambda_{r}} F_{obj}(\sigma,\epsilon,\lambda_{r})= \sum_{i=1}^{N_{p}} \left(\frac{P_{v}^{SAFT}(T_{i},\sigma,\epsilon,\lambda_{r})-P_{v}^{exp}(T_{i})}{P_{v}^{exp}(T_{i})} \right)^2 +\\
 \sum_{i=1}^{N_{p}} \left(\frac{\rho_{l}^{SAFT}(T_{i},\sigma,\epsilon,\lambda_{r})-\rho_{l}^{exp}(T_{i})}{\rho_{l}^{exp}(T_{i})} \right)^2
\end{aligned}
\label{eqn:fobj}
\end{equation}
where $N_{p}$ is the number of experimental points, $P_{v}$ is the vapor pressure and $\rho_{l}$ is the saturated liquid density. The minimized properties can also change and other possible properties as superficial tension and speed of sound can also be taken into account. These multiple parameters make it necessary the use of a wide range of experimental data since multiple solutions can be found for the fit. So one need to be careful in deciding the level of coarse graining (i.e. the parameter $m_{s}$) and subsequent parameter space that will not result in some physical inconsistencies like a fluid with premature freezing.

\citeonline{lafitte2012} suggested that the two corrections factors ($c_{\sigma}$ and $c_{\epsilon}$) should be estimated with simulation data when using Eq. \eqref{eqn:aringlafitte} for the ring contribution. They are related to the EoS parameters by scaled parameters:

\begin{equation}
\sigma^{scaled} = c_{\sigma}\sigma^{SAFT}
\label{eqn:csigma}
\end{equation}
\begin{equation}
\epsilon^{scaled} = c_{\epsilon}\epsilon^{SAFT}
\label{eqn:ceps}
\end{equation}

According to \citeonline{lafitte2012}, these corrections are necessary because the approximations employed in the EoS theory generate discrepancies between molecular simulations and the EoS results for ring molecules modeled with Eq. \eqref{eqn:aringlafitte}. The objective function for this second estimation is given by:

\begin{equation}
\begin{split}
\min\limits_{c_{\sigma},c_{\epsilon}} F_{obj}(c_{\sigma},c_{\epsilon})= \sum_{i=1}^{N_{p}} \left(\frac{P_{v}^{sim}(T_{i},\sigma^{SAFT},\epsilon^{SAFT})-P_{v}^{SAFT}(T_{i},\sigma^{scaled},\epsilon^{scaled})}{P_{v}^{sim}(T_{i},\sigma^{SAFT},\epsilon^{SAFT})} \right)^2 + \\
 \sum_{i=1}^{N_{p}} \left(\frac{\rho_{liq}^{sim}(T_{i},\sigma^{SAFT},\epsilon^{SAFT})-\rho_{liq}^{SAFT}(T_{i},\sigma^{scaled},\epsilon^{scaled})}{\rho_{liq}^{sim}(T_{i},\sigma^{SAFT},\epsilon^{SAFT})} \right)^2
\end{split}
\label{eqn:fobjla}
\end{equation}

The repulsive parameter is maintained in the value found on the minimization of Eq. \eqref{eqn:fobj}, so the refined values for the force field are:

\begin{equation}
\sigma^{sim} = \sigma^{SAFT}/c_{\sigma}
\label{eqn:simsigma}
\end{equation}

\begin{equation}
\epsilon^{scaled} = \epsilon^{SAFT}/c_{\epsilon}
\label{eqn:simeps}
\end{equation}

It is interesting to point out that this new parametrization is not necessary when using Eq. \eqref{eqn:aringmuller} as the ring contribution. The other method to obtain the force field parameters is the correspondent state parametrization for the EoS SAFT-VR Mie \cite{mejia2014}. This method considers that the unweighted volume average of the attractive contribution to the Mie intermolecular potential,$a_{1}$, can be given a mean field approximation:

\begin{equation}
a_{1} = 2\pi\rho\sigma^{3}\epsilon\alpha
\label{eqn:a1corres}
\end{equation}

The van der Waals constant, $\alpha$, considering $ \lambda_{a} = 6$ is related by the Mie exponents by:

\begin{equation}
\alpha = \frac{1}{\epsilon\sigma^{3	}} \int_{\sigma}^{\infty} \phi(r)r^{2}dr = \frac{\lambda_{r}}{3(\lambda_{r}-3)}\left(\frac{\lambda_r}{6}\right)^{6/(\lambda_{r} - 6)}  
\label{eqn:alpha}
\end{equation}

The parametrization in this method starts by using the experimental acentric factor, $\omega$, for each molecule with a fixed value of $ m_{s}$ to obtain the value of the repulsive exponent with the following Padé series:

\begin{equation}
\lambda_{r} = \frac{\sum_{i=0} a_{i}\omega^{i}}{1+\sum_{i=1} b_{i}\omega^{i}}   
\label{eqn:lambdaco}
\end{equation}

$a_{i}$ and $b_{i}$ are dependent parameters of the number of segments and a table with its values is presented in the original paper \cite{mejia2014}. Substituting $\lambda_{r}$ into Eq. \eqref{eqn:alpha}, the van der Waals constant can be found. The reduced critical potential $T_{c}^{*}$ can also be related to $\alpha$ by a Padé series: 

\begin{equation}
T_{c}^{*} = \frac{\sum_{i=0} c_{i}\alpha^{i}}{1+\sum_{i=1} d_{i}\alpha^{i}}   
\label{eqn:tc}
\end{equation}

The values of $c_{i}$ and $d_{i}$ are also available in the original paper. The reduced temperature of the equation above is used in conjunction with the experimental critical temperature, $ T_{c}$, to find the energy parameter with the relation below:

\begin{equation}
T_{c}^{*} = \frac{\kappa_{b}T_{c}}{\epsilon}   
\label{eqn:epscorre}
\end{equation}

The diameter parameter, however, is not obtained with the critical properties, but with the reduced liquid density,$\rho_{T_{r}=0.7}$, at the reduced temperature ,$T_{r}$, of $0.7$. This density is also obtained with a Padé series using parameters obtained by \citeonline{mejia2014}:

\begin{equation}
\rho_{T_{r}=0.7}^{*} = \frac{\sum_{i=0} j_{i}\alpha^{i}}{1+\sum_{i=1} k_{i}\alpha^{i}} 
\label{eqn:denscorre}
\end{equation}

The relation among the equation above, $\sigma$ and the experimental density is given by:

\begin{equation}
\rho_{T_{r}=0.7}^{*} = \rho_{T_{r}=0.7}\sigma^{3}N_{av}   
\label{eqn:sigmacorre}
\end{equation}
where $N_{av}$ is The Avogadro number. This correspondent state method has the advantage of only requiring critical data, that it is available for a great range of fluids, and one liquid density point. In addition to that, there is an available online parameter database obtained with this strategy \cite{ervik2016}.     

The binary interaction parameter $k_{ij}$ of Eq. \eqref{eqn:epsmix} is necessary to adjust the mixture behaviour of chemically distinct components. Normally, it is estimated minimizing the difference between experimental binary vapor liquid equilibrium or superficial tension data and the SAFT-VR Mie EoS output data \cite{muller2017,lobanova2016}. The objective function is similar to: 

\begin{equation}
\begin{aligned}
\min\limits_{k_{ij}} F_{obj}(k_{ij})= \sum_{k=1}^{N_{p}} \left(\frac{P_{v}^{SAFT}(T_{k},x,k_{ij})-P_{v}^{exp}(T_{k},x)}{P_{v}^{exp}(T_{k},x)} \right)^2 +\\
 \sum_{k=1}^{N_{p}} \left(\frac{\rho_{l}^{SAFT}(T_{k},x,k_{ij})-\rho_{l}^{exp}(T_{i})}{\rho_{l}^{exp}(T_{i})} \right)^2
\end{aligned}
\label{eqn:fobjmix}
\end{equation}

However, \citeonline{ervik20162} used molecular simulation results to fit the parameter to the superficial tension data of the mixture water-toluene. The strategy followed by them was to do simulations in three values of $k_{ij}$ and then refine the parameter value until a value in good agreement with the experimental data was found. 

%----------------------------------------------------------------------------------------
%	SECTION 2
%----------------------------------------------------------------------------------------


\section{Expanded Ensemble Method}

Instead of doing various simulations in different values of $\lambda$, expanded ensemble simulations \cite{lyubartsev} were developed to allow a non-Boltzmann sampling scheme of different states in only one simulation. The statistical expanded ensemble, $Z^{EE}$ can be defined as a sum of sub ensembles $Z_{i}$ in different values of $\lambda$:

\begin{equation}
Z^{EE} = \sum_{i=1}^{N} Z_{i}(\lambda_{i}) exp(\eta_{i})
\label{eqn:ee}
\end{equation}   
where N is the number of alchemical states and $\eta_{i}$ is the arbitrary weight of the subs ensemble $Z_{i}$ at each state. In the application of this method for solvation energy calculations with molecular dynamics, $\lambda$ corresponds to the coupling parameter of the soft-core potential (Eq. \ref{eq:softcore}) and the expanded ensemble is sampled in the MD simulations by performing a arbitrary number of steps followed by a state transition. \citeonline{chodera2011} proved that the sampling of the expanded ensemble are similar to the Gibbs sampling method \cite{geman1984,liu2002}. Following the Gibbs method, the sampling of the configuration space x for one state $\lambda_{k}$ during the MD steps is done using the conditional distribution:

\begin{equation}
\pi(x|\lambda_{k}) = \dfrac{\exp[-\beta u(x,\lambda_{k})]}{\int dx \exp [- \beta u(x,\lambda_{k})]}
\label{eqn:rhoee1}
\end{equation} 

Meanwhile, the state transition in the MD simulation uses the following conditional distribution:

\begin{equation}
\pi(\lambda_{k}|x) = \dfrac{\exp[-\beta u(x,\lambda_{k}) + \eta_{k}]}{ \sum_{k=1}^{K} \exp [- \beta u(x,\lambda_{k})+ \eta_{k}]}
\label{eqn:rhoee2}
\end{equation} 
where $u(x,\lambda_{k}) = U(x,\lambda_{k}) + PV(x,\lambda_{k})$ is the reduced potential function for the NPT ensemble. There are a variety of proposal or acceptance schemes to do the expanded sampling using the Eq. \eqref{eqn:rhoee2}, but \citeonline{chodera2011} suggested that the independence sampling \cite{liu2002} is the best strategy to increase the number of uncorrelated configurations. The implementation suggested by them updates the state index from $i$ to $j$ by first generating a uniform random number $R$ on the interval $[0,1)$ and then selecting the smallest new value of $j$ that satisfies  the relation bellow:

\begin{equation}
R < \sum_{i=1}^{j} \pi(\lambda_{i}|x) 
\label{eqn:relee2}
\end{equation} 

The sampling strategy above depends on the weights above in order to assure a adequate sampling of the states. If there isn't a sufficient number of states sampled, the expanded ensemble becomes deficient in obtaining input data to estimate free energy differences with the methods exposed in Chapter 2. The weights can be calculated following the flat-histogram approach \cite{bernd1992,bernd1993,dayal2004}. This strategy aims to obtain adequate sampling by sampling all the states in a equal number of ways, i. e. the ratio of the probability of sampling state ($\pi_{i}$) to the probability of sampling state $j$ ($\pi_{j}$) is equal to one. Given that $\pi_{i}$ has the following equation:

\begin{equation}
\pi_{i} = \dfrac{Z_{i}(\lambda_{i}) exp(\eta_{i})}{Z^{EE}} 
\label{eqn:wei1}
\end{equation} 
and using Eqs. \ref{eq:dif} and \ref {eq:partiso}, the following relation can be obtained for $\pi_{i}/\pi_{j}=1$:

\begin{equation}
(\eta_{i} - \eta_{j}) = \beta(G_i-G_j)
\label{eqn:weight}
\end{equation}

The Eq. \eqref{eqn:weight} is solved iteratively with trial simulations. For the first simulation, the values of $\eta$ are chosen or set to zero and the histogram of the states visited is obtained. With this histogram is possible to estimate the free energy differences and, since the weights are related to the free energies by Eq. \eqref{weight}), the next values of $\eta$ can be calculated  from the previous result. This iteration goes on until a uniform distribution is secured. The weights of the uniform distribution can then  be used in a longer simulation to obtain the final solvation free energy differences.

The choice of the $\lambda$ set correspondent to overlapping alchemical states are crucial to acquire accurate energy differences. One of the existing methods to obtain the optimal stage of the $\lambda$ domain is the one developed by \citeonline{escobedo2017} with basis in the one developed by \cite{trebst2004}. In this method, the $\lambda s$ are optimized with the minimization of the number of round trips per CPU time between the lowest $\lambda$ ($=0$) and highest $\lambda$ ($=1$). This is done by maximizing the steady-state stream $\phi$ of the simulation which "walks" among the values of $\lambda$. This stream is estimated form Fick's diffusion type of law:

\begin{equation}
\phi = D(\Lambda) \Pi (\Lambda) \dfrac{dx(\Lambda)}{d \Lambda}
\label{eqn:stream}
\end{equation}

In the equation above, $\Lambda$ is the actual continue value of the coupling parameter. This continue function of $\lambda s$ may be obtained by interpolating them linearly. $D(\Lambda)$ is the diffusivity at the state $\Lambda$ and $x(\Lambda)$ is the fraction of times that the trial simulation at state $\Lambda_{i}$ has most recently visited the state $\lambda=1$ as opposed to state $\lambda=0$. The derivative ${dx(\Lambda)}/{d \Lambda}$ can be approximated with the central finite differences. Finally, $\Pi (\Lambda)$ represents the probability of visiting $\Lambda$. 

\begin{equation}
\Pi (\Lambda) = \dfrac{C^{'} \bar{\Pi} (\lambda)}{\Lambda_{i+1} - \Lambda_{i}}
\label{eqn:plambda}
\end{equation}

The $C^{'} $ term in the equation above represents a constant and $\bar{\Pi} (\lambda)$ is the arithmetic average of visiting the $\Lambda$ states:

\begin{equation}
\bar{\Pi} (\lambda) = \dfrac{\pi_{i+1} - \pi_{i}}{2}
\label{eqn:barplambda}
\end{equation}

The $\phi$ is maximum when the the probability $\Pi^{'}(\Lambda_{i})$ is proportional to $1/\sqrt{D(\Lambda)}$. With that information is possible to estimate the diffusivity using one trial simulation with the equation bellow:

\begin{equation}
D(\Lambda) = \dfrac{\Lambda_{i+1} - \Lambda_{i}}{\bar{\Pi} (\lambda) {dx(\Lambda)}/{d \Lambda}}
\label{eqn:diff}
\end{equation}

Hence, it is possible to calculate $\bar{\Pi} $ and, consequently, the cumulative probability, which is used to calculate the new $\lambda$ states:

\begin{equation}
\Phi = \int_{\lambda =0}^{\lambda =1} \Pi^{'}(\Lambda_{i}) d \Lambda = \dfrac{i}{K}
\label{eqn:cumfun}
\end{equation}
where, $K$ is the total number of $\lambda$ states.

\section{Gibbs Ensemble Monte Carlo}





%Ramrattan et al. [57] have noted that the value of the repulsive exponent
%has a direct relation to the fluid range, i.e. the ratio between the
%critical and triple point of a fluid; and that this metric is a valuable tool to bracket the possible parameter space. F
%For the attractive
%exponent used here, “hard” repulsive exponents, e.g. values larger
%than 12 reduce the fluid range and after a value of 43 the
%fluid phase is no longer stable being suppressed by the presence
%of the solid phase [57]. The upshot of this is that hard potentials
%might exhibit premature freezing as compared to the experimental
%results.




\chapter{Methodology} % Main chapter title

\label{Chapter4} % Change X to a consecutive number; for referencing this chapter elsewhere, use \ref{ChapterX}

\section{Phenanthrene Parameterization}\label{parame}

The two parameterization strategies for ring molecules described in section \ref{parsaft} were implemented for phenanthrene. For both of them, only vapor pressure data \cite{pvphen} was used due to the unavailability of saturated liquid density. The attractive parameter $\lambda _{a}$ was set to six to avoid correlation with the repulsive parameter. The parameterization with the ring equation of \citeonline{muller2017} was done with the number of segments fixed in 5 since this level of coarse graining was also used for the similar molecule anthracene:
\begin{figure}[th]
	\centering
	\includegraphics[width=0.25\linewidth]{Figures/fen5}
	\caption{Geometry for $m_{s}=5$}
	\label{fig:fen5}
\end{figure}

The minimization was done using the Particle Swarm Optimization (PSO) method with the following objective function:
\begin{equation}
\min\limits_{\sigma,\epsilon,\lambda_{r}} F_{obj}(\sigma,\epsilon,\lambda_{r})= \sum_{i=1}^{N_{p}} \left(\frac{P_{v}^{SAFT}(T_{i},\sigma,\epsilon,\lambda_{r})-P_{v}^{exp}(T_{i})}{P_{v}^{exp}(T_{i})} \right)^2
\label{eqn:fobjm}
\end{equation}

$P_{v}^{exp}$ is the experimental vapor pressure and $P_{v}^{SAFT}$ is the vapor pressure obtained with SAFT-VR Mie EoS. We used the routine proposed by  \citeonline{smithbook} routine to calculate the bubble point with the EoS. The resulting  parameters $\sigma$, $\epsilon $ and $\lambda _{r}$ of Eq. \eqref{eqn:fobjm} are the final force field parameters used in molecular simulations. 

The parameterization with the \citeonline{lafitte2012} ring equation was done with $m_{s}=3$ so every bead would represent one ring:

\begin{figure}[th]
	\centering
	\includegraphics[width=0.15\linewidth]{Figures/fe3}
	\caption{Geometry for $m_{s}=3$}
	\label{fig:fen3}
\end{figure}

The first part of the estimation followed the same procedure described above for the \citeonline{muller2017} equation. As explained in section \ref{parsaft}, the \citeonline{lafitte2012} equation requires the estimation of correction factors $c_{\sigma}$ and $c_{\epsilon}$ (Eqs. \eqref{eqn:csigma} and \eqref{eqn:ceps}). The PSO method was used with Eq. \eqref{eqn:fobjla}. In this equation, the vapor pressures and saturated liquid densities from molecular simulations were obtained using the Gibbs Ensemble Monte Carlo method on the NVT ensemble  (section \ref{gemc}).

The boxes for the GEMC-NVT simulations were generated inserting 400 molecules on the liquid box and 100 on the vapor one using the Packmol package \cite{packmol}. Initial densities of each box were equal to the saturated densities found with the SAFT-VR Mie Eos in order to avoid the migration of all molecules to a single phase throughout the simulation.  Simulations had equilibration and production times of $10^{4}$ and $5 10^{4}$ MC cycles respectively. Each MC cycle corresponded to $10^3$ rotation trials, $10^3$ translation trials, $10^2$ molecule insertion trials, $10^2$ molecule deletion trials and 10 volume exchange trials. The cut off distance was equal to $20 \hat{A}$ without long range interactions. The saturated vapor density ($\rho_{vap}$), the saturated liquid density ($\rho_{liq}$) and the vapor pressure ($P_{v}$) were sampled at each 100 MC cycles and this data were divided in five blocks to calculate the averages and standard deviations. With the correction factors found after the estimations, we calculated the $\sigma$ and $\epsilon$ parameters with Eqs. \eqref{eqn:simsigma} and \eqref{eqn:simeps}.


\section{Solvation Free Energy Calculations}\label{solvme}

Molecular dynamic simulations to estimate the free energy differences with the SAFT-$\gamma$ Mie force field were performed using the Lammps package \cite{lammps}. The motion equations were integrated with the velocity-Verlet algorithm \cite{verlet} with a time step of 1 fs. As required by the coarse grained model,  molecules were treated as rigid bodies. The thermostat and the barostat were the Nose/Hoover with chains with a damping factors of 100 and 1000 fs respectively. Electrostatics interactions are not explicitly accounted by the SAFT-$\gamma$ Mie force field, hence there was no shifting of forces or long range corrections. The potential cutoff was equal to 20 $\dot{A}$ \cite{muller2017} with a neighbor skin of 2 $\dot{A}$. The initial configurations of the  solvated systems were generated with the Packmol package \cite{packmol}. For the binary mixtures, one molecule of solute and a varying number of solvent molecules- 700 molecules for toluene and octanol, 1024 for hexane, 3000 for water - were randomly added to a cubic box. The simulations to study solvation free energy of phenanthrene in a mixture of toluene and carbon dioxide were done with different fractions of carbon dioxide. The  system consisted of one molecule of phenanthrene for all the fractions and 123 molecules of $CO_{2}$ and 618 molecules of toluene for $w_{CO_{2}} = 0.087$; 166 molecules of $CO_{2}$; 589 molecules of toluene for $w_{CO_{2}} = 0.119$; 232 molecules of $CO_{2}$ and 545 molecules of toluene for $w_{CO_{2}} = 0.169$ and 380 molecules of $CO_{2}$ and 446 molecules of toluene for $w_{CO_{2}} = 0.289$.

All simulations were carried out maintaining the temperature and pressure constant at 298 K and 1 bar, except the ones containing carbon dioxide. These had the temperature of 298 K and the pressure of the liquid phase equilibrium correspondent to the $CO_{2}$ fraction \cite{co2toliq}. For all the simulations, the initial box was equilibrated at the NPT ensemble for 2 ns and then the resulting configurations were used as the initial configuration of the expanded ensemble simulations. These were carried out with the Lammps user package for expanded ensemble simulations with the Mie Potential developed by our group, available at https://github.com/atoms-ufrj/USER-ALCHEMICAL. The sampling of a new state was tried at every 10 MD steps. In order to define the optimal values of $\lambda$ and $\eta$ related to each state, short trials simulations, lasting around 10 ns, were carried out. In the first simulation, the group of $\lambda$ for all the pairs solvent-solute was: (0.0,0.15,0.2,0.25,0.3,0.4,0.45,0.5,0.55,0.7,0.9,1.0) and the $\eta s$ were set to zero or were given the values of the ones found for similar mixtures. The subsequent groups of $\eta$ were estimated  with the flat histogram approach (Eq. \eqref{eqn:weight}). We then used the new weights to optimize the group of $\lambda s$ by minimizing the number of round trips, as described in section \ref{ee}. The $\eta s$ correspondent to the newest group of $\lambda s$ were interpolated from the free energy differences. With the final values of $\eta$ and $\lambda $ defined for each mixture, larger simulations with a time of 20 ns were carried out. 

Since the force field considers that the beads don't have charges, there is no coulombic interaction and Eq. \eqref{eq:freesolv} becomes equal to $\Delta G_{3 \rightarrow 4} $. The post processing method used to calculate free energy differences was the Multisate Bennet Acceptance Ratio (MBAR) described in section \ref{mbar}. The software alchemical-analysis \cite{klimovich} was used to obtain the $\Delta G_{solv}$ with MBAR and to assess the results quality. We only estimated the binary interaction  parameter of Eq. \eqref{eqn:epsmix} for pairs with water as a solvent. The estimation was done by performing trial  expanded ensemble simulations in three values of $k_{ij}$. With the $\Delta G_{solv}$ obtained with these simulations, we did a linear fit to obtain the refined value of the parameter. We usede this strategy because the estimation with SAFT VR Mie EoS gave poor results for the free energy.


\chapter{Results and Discussion} % Main chapter title

\label{Chapter5} % Change X to a consecutive number; for referencing this chapter elsewhere, use \ref{ChapterX}

\begin{table*}[h]
\center
  \caption{Calculated and experimental values for the Gibbs energy of solvation (kcal/mol) of solutes in non aqueous solvents}
  \label{tbl:solv1}
  \begin{tabular}{lllll}
    \hline
      Solvent & Solute & $\Delta G_{solv}^{exp}$ & $\Delta G_{solv}^{Mie}$ & RMSE \\
    \hline
    hexane    & benzene      & -3.96  & -3.90  $\pm$ 0.02 &  \\
    hexane    & phenanthrene & -10.01 & -9.16  $\pm$ 0.02 & 0.85 \\
    hexane    & pyrene       & -11.53 & -10.82 $\pm$ 0.02 & 0.71 \\
    toluene   & pyrene       & -12.86 & -11.74 $\pm$ 0.02 & 1.12\\
    toluene   & anthracene   & -11.31 & -10.03 $\pm$ 0.02 & 1.28\\
    1-octanol & propane      & -1.32  & -1.34  $\pm$ 0.02 & 0.02 \\
    1-octanol & phenanthrene & -10.22 & -8.34  $\pm$ 0.04 & - \\
    1-octanol & anthracene   & -10.22 & -8.34  $\pm$ 0.04 & - \\
    \hline
  \end{tabular}

\end{table*}

\begin{table}[h]
\center
  \caption{Calculated values for the Gibbs energy of solvation (kcal/mol) of phenanthrene in toluene+$CO_{2}$}
  \label{tbl:solv3}
  \begin{tabular}{lll}
    \hline
      $w_{CO_{2}}$ & $\Delta G_{solv}^{Mie}$ & $\Delta G_{solv}^{Mie}$\\
    \hline
     & $k_{ij}=0$& $k_{ij} \neq 0$\\
    \hline
    0.0    & -10.55 $\pm$ 0.03  & \\
    0.050  & -10.55 $\pm$ 0.03  & \\
    0.087  & -10.73 $\pm$ 0.02  & \\
    0.119  & -10.73 $\pm$ 0.02  & \\
    0.169  & -10.67 $\pm$ 0.02  & \\
    0.289  & -10.77 $\pm$ 0.02  & \\
    0.471  & -8.958 $\pm$ 0.02  & \\
    \hline
  \end{tabular}

\end{table}

\subsection{Hydration free energies}


\begin{table}[h]
\center
  \caption{Binary interaction parameters employed}
  \label{tbl:kij}
  \begin{tabular}{ll}
    \hline
      Pair & $k_{ij}$ \\
    \hline
    water  + propane      & 0.067  \\
    water  + benzene      & 0.162 \\  
    water  + toluene      & 0.152 \\
    water  + phenanthrene & 0.148  \\
    \hline
  \end{tabular}

\end{table}

\begin{table*}[h]
\center
  \caption{Calculated and experimental values for the Gibbs energy of solvation (kcal/mol) of solutes in water}
  \label{tbl:solv2}
  \begin{tabular}{lllll}
    \hline
     Solute      & $\Delta G_{solv}^{exp}$ & $\Delta G_{solv}^{Mie}$ & RMSE &$\Delta G_{solv}^{GAFF}$ \\
    \hline
    propane      &  2.00 $\pm$ 0.20 & 1.98 $\pm$ 0.02& 0.03 &2.50 $\pm$0.02 \\
    benzene      & -0.90 $\pm$ 0.20 & 4.51 $\pm$     &      &-0.81$\pm$0.02 \\  
    toluene      & -0.90 $\pm$ 0.20 & 4.34 $\pm$     &      &-0.79$\pm$0.03\\
    phenanthrene & -3.88 $\pm$ 0.60 & 3.63 $\pm$ 0.03& 0.25 &-5.26$\pm$0.03 \\
    \hline
  \end{tabular}

\end{table*}
  
\chapter{Conclusions} % Main chapter title

\label{Chapter6} % Change X to a consecutive number; for referencing this chapter elsewhere, use \ref{ChapterX}

This dissertation consisted of the study of solvation free
energy calculations of aromatic solutes that can mimic asphaltenes with the SAFT-$\gamma$ 
Mie force field. By doing that, we provided information about these estimates since solvation free energy studies are mostly done using water as a solvent and all-atom force fields based on the Lennard Jones Potential.Additionally, the free energy estimations can help improve the SAFT-$\gamma$  Mie force field since these calculations are helpful in identifying errors in the modeling process. The parametrization of the SAFT-$\gamma$  Mie force field is more straightforward
than other force fields since its parameters are obtained through the SAF-VR
Mie EoS. Following this strategy, the phenanthrene parameters were obtained using two
different ring equations and geometries and phase equilibrium data. However, only the parameters estimated with the ring equation proposed by \citeonline{muller2017} were utilized in the solvation free energy simulations since this equation provided an adequate set of parameters.


To obtain accurate solvation free energies, we carefully selected and optimized the coupling parameter and their respective simulation weights used in our Expanded Ensemble simulations. The resulting potential energies from these simulations were then served as input to estimate solvation free energy differences with the MBAR method. The results for solvation free energy differences with non-aqueous solvents had absolute deviations to the experimental
data of less than 2.0 kcal/mol, except for the pair 1-octanol+anthracene. We also observed the geometry
effect on the free energy curves - larger molecules had steeper curves
and more substantial absolute deviations. The influence of carbon dioxide on the solvation free
energy of phenanthrene in toluene was found to be minimum. The $\Delta G_{solv}$ decreased slightly until the mass fraction of $CO_{2}$ was equal to 0.119 and, after this point, solvation free
energies increased. 

Hydration free energy differences calculations with the SAFT-$\gamma$ Mie model
required the use of relatively larger values of $k_{ij}$ to obtain satisfactory results.
We chose to estimate the parameter with the output from molecular dynamics data since
the strategy of using the SAFT-VR Mie EoS also didn’t provide good results. This
necessity of one additional parameter happens probably due to the lack of an association term
on the EoS that the model is based. The results with $k_{ij}$ estimated with MD output
were great, the absolute deviations to the experimental data found were smaller than
the ones for the GAFF and ELBA force field. We also used the solvation free energies to calculate partition coefficients in water/1-octanol, water/hexane, and toluene/hexane, and the results were satisfactory when compared to the experimental data.

Overall, the SAFT-$\gamma$ Mie force field proved to be an excellent model to represent the solvation
phenomenon. It correctly described solvation free energy differences of solutes
mimicking asphaltenes in hexane, toluene, 1-octanol, and water. The requirement of
binary interaction parameter estimated with MD output for hydration free energies
increases the simulation time, which is already more significant for this water model due to its
coarse-graining level. Nevertheless, the SAFT-$\gamma$ Mie force field for water used doesn’t predict freezing at room temperature as other force fields, which is essential for hydration
free energy calculations.
This dissertation had success in using a coarse-grained force field to perform
free energy calculations. Based on this work, we have some ideas for future development. We intend to test the binary interaction parameter transferability to calculations with other
aromatic solutes in water. Additionally, we want to use the SAFT-$\gamma$ Mie force field to model more
complex asphaltenes models and, consequently, increase the scale of these simulations. The final step to expand this work would be to develop new methodologies to use solvation free energies to calculate solubility efficiently


 


%ideias para revisão da bibliografia:
%tipos de modelo coarse grained
%metodologias de energia livre
%\chapter*[Introdução]{Introdução}
%\addcontentsline{toc}{chapter}{Introdução}
%% ----------------------------------------------------------
%
%Este documento e seu código-fonte são exemplos de referência de uso da classe
%\textsf{abntex2} e do pacote \textsf{abntex2cite}. O documento 
%exemplifica a elaboração de trabalho acadêmico (tese, dissertação e outros do
%gênero) produzido conforme a ABNT NBR 14724:2011 \emph{Informação e documentação
%- Trabalhos acadêmicos - Apresentação}.
%
%A expressão ``Modelo Canônico'' é utilizada para indicar que \abnTeX\ não é
%modelo específico de nenhuma universidade ou instituição, mas que implementa tão
%somente os requisitos das normas da ABNT. Uma lista completa das normas
%observadas pelo \abnTeX\ é apresentada em \citeonline{abntex2classe}.
%
%Sinta-se convidado a participar do projeto \abnTeX! Acesse o site do projeto em
%\url{http://www.abntex.net.br/}. Também fique livre para conhecer,
%estudar, alterar e redistribuir o trabalho do \abnTeX, desde que os arquivos
%modificados tenham seus nomes alterados e que os créditos sejam dados aos
%autores originais, nos termos da ``The \LaTeX\ Project Public
%License''\footnote{\url{http://www.latex-project.org/lppl.txt}}.
%
%Encorajamos que sejam realizadas customizações específicas deste exemplo para
%universidades e outras instituições --- como capas, folha de aprovação, etc.
%Porém, recomendamos que ao invés de se alterar diretamente os arquivos do
%\abnTeX, distribua-se arquivos com as respectivas customizações.
%Isso permite que futuras versões do \abnTeX~não se tornem automaticamente
%incompatíveis com as customizações promovidas. Consulte
%\citeonline{abntex2-wiki-como-customizar} par mais informações.
%
%Este documento deve ser utilizado como complemento dos manuais do \abnTeX\ 
%\cite{abntex2classe,abntex2cite,abntex2cite-alf} e da classe \textsf{memoir}
%\cite{memoir}. 
%
%Esperamos, sinceramente, que o \abnTeX\ aprimore a qualidade do trabalho que
%você produzirá, de modo que o principal esforço seja concentrado no principal:
%na contribuição científica.
%
%Equipe \abnTeX 
%
%Lauro César Araujo
%
%% ----------------------------------------------------------
%% PARTE
%% ----------------------------------------------------------
%\part{Preparação da pesquisa}
%% ----------------------------------------------------------
%
%% ---
%% Capitulo com exemplos de comandos inseridos de arquivo externo 
%% ---
%\include{abntex2-modelo-include-comandos}
%% ---
%
%% ----------------------------------------------------------
%% PARTE
%% ----------------------------------------------------------
%\part{Referenciais teóricos}
%% ----------------------------------------------------------
%
%% ---
%% Capitulo de revisão de literatura
%% ---
%\chapter{Lorem ipsum dolor sit amet}
%% ---
%
%% ---
%\section{Aliquam vestibulum fringilla lorem}
%% ---
%
%\lipsum[1]
%
%\lipsum[2-3]
%
%% ----------------------------------------------------------
%% PARTE
%% ----------------------------------------------------------
%\part{Resultados}
%% ----------------------------------------------------------
%
%% ---
%% primeiro capitulo de Resultados
%% ---
%\chapter{Lectus lobortis condimentum}
%% ---
%
%% ---
%\section{Vestibulum ante ipsum primis in faucibus orci luctus et ultrices
%posuere cubilia Curae}
%% ---
%
%\lipsum[21-22]
%
%% ---
%% segundo capitulo de Resultados
%% ---
%\chapter{Nam sed tellus sit amet lectus urna ullamcorper tristique interdum
%elementum}
%% ---
%
%% ---
%\section{Pellentesque sit amet pede ac sem eleifend consectetuer}
%% ---
%
%\lipsum[24]
%
%% ----------------------------------------------------------
%% Finaliza a parte no bookmark do PDF
%% para que se inicie o bookmark na raiz
%% e adiciona espaço de parte no Sumário
%% ----------------------------------------------------------
%\phantompart
%
%% ---
%% Conclusão
%% ---
%\chapter{Conclusão}
%% ---
%
%\lipsum[31-33]

% ----------------------------------------------------------
% ELEMENTOS PÓS-TEXTUAIS
% ----------------------------------------------------------
\postextual
% ----------------------------------------------------------

% ----------------------------------------------------------
% Referências bibliográficas
% ----------------------------------------------------------
\bibliography{thesis}

% ----------------------------------------------------------
% Glossário
% ----------------------------------------------------------
%
% Consulte o manual da classe abntex2 para orientações sobre o glossário.
%
%\glossary

% ----------------------------------------------------------
% Apêndices
% ----------------------------------------------------------

% ---
% Inicia os apêndices
% ---
\begin{apendicesenv}

% Imprime uma página indicando o início dos apêndices
%\partapendices

% ----------------------------------------------------------
\chapter{Optimized values of $\lambda$ and $\eta$}
% ----------------------------------------------------------
\begin{table*}[h]
	\centering
	\caption{Optimized values of $\lambda$ and $\eta$ for the solutes in hexane}
	\begin{tabular}{llll}
		\hline
        \multicolumn{2}{c}{pyrene}& \multicolumn{2}{c}{phenanthrene}\\
		\hline
		$\lambda$ & $\eta$  & $\lambda$ & $\eta$   \\ 
		\hline
   0.000	&	0.000	&	0.000	&	0.000	\\
   0.076	&	4.234	&	0.090	&	1.981	\\
   0.107	&	5.620	&	0.132	&	3.461	\\
   0.132	&	6.499	&	0.161	&	4.494	\\
   0.152	&	6.690	&	0.185	&	5.185	\\
   0.170	&	6.643	&	0.205	&	5.552	\\
   0.189	&	6.461	&	0.224	&	5.725	\\
   0.213	&	6.091	&	0.244	&	5.722	\\
   0.242	&	5.566	&	0.268	&	5.523	\\
   0.280	&	4.729	&	0.305	&	4.975	\\
   0.355	&	2.853	&	0.372	&	3.576	\\
   	0.483	&	-0.778	&	0.500	&	0.297	\\
   	0.678	&	-6.947	&	0.560	&	-1.390	\\
   	0.788	&	-10.631	&	0.722	&	-6.309	\\
   	1.000	&	-18.141	&	1.000	&	-15.448	\\
   	\hline
   \end{tabular}
\end{table*}

\begin{table*}[h]
	\centering
	\caption{Optimized values of $\lambda$ and $\eta$ for the solutes in 1-octanol}
	\begin{tabular}{llllll}
		\hline
		\multicolumn{2}{c}{propane}& \multicolumn{2}{c}{anthracene}& \multicolumn{2}{c}{phenanthrene}\\
		\hline
		$\lambda$ & $\eta$ & $\lambda$ & $\eta$  & $\lambda$ & $\eta$   \\ 
		\hline
0.000	&	0.000	&	0.000	&	0.000	&	0.000	&	0.000	\\
0.027	&	3.126	&	0.078	&	3.932	&	0.049	&	2.578	\\
0.050	&	5.109	&	0.111	&	6.178	&	0.091	&	5.663	\\
0.073	&	6.093	&	0.130	&	7.426	&	0.125	&	8.575	\\
0.095	&	6.570	&	0.143	&	8.201	&	0.144	&	10.069	\\
0.117	&	6.826	&	0.154	&	8.717	&	0.157	&	10.978	\\
0.142	&	6.956	&	0.164	&	9.085	&	0.169	&	11.599	\\
0.174	&	6.969	&	0.174	&	9.357	&	0.180	&	12.040	\\
0.215	&	6.847	&	0.184	&	9.556	&	0.192	&	12.340	\\
0.269	&	6.554	&	0.197	&	9.676	&	0.206	&	12.499	\\
0.337	&	6.050	&	0.214	&	9.681	&	0.225	&	12.478	\\
0.427	&	5.228	&	0.238	&	9.490	&	0.253	&	12.161	\\
0.545	&	3.955	&	0.274	&	8.958	&	0.298	&	11.280	\\
0.720	&	1.843	&	0.326	&	7.906	&	0.371	&	9.406	\\
1.000	&	-1.903	&	0.399	&	6.088	&	0.484	&	5.891	\\
&		&	0.515	&	2.777	&	0.664	&	-0.516	\\
&		&	0.695	&	-2.960	&	0.802	&	-5.908	\\
&		&	1.000	&	-13.768	&	1.000	&	-14.073	\\

		
		\hline
	\end{tabular}
\end{table*}

\begin{table*}[h]
	\centering
	\caption{Optimized values of $\lambda$ and $\eta$ for the solutes in toluene}
	\begin{tabular}{llllll}
		\hline
		\multicolumn{2}{c}{pyrene}& \multicolumn{2}{c}{anthracene}& \multicolumn{2}{c}{phenanthrene}\\
		\hline
		$\lambda$ & $\eta$ & $\lambda$ & $\eta$  & $\lambda$ & $\eta$   \\ 
		\hline
0.000	&	0.000	&	0.000	&	0.000	&	0.000	&	0.000	\\
0.090	&	2.563	&	0.119	&	0.218	&	0.136	&	0.726	\\
0.130	&	4.338	&	0.174	&	1.210	&	0.191	&	2.307	\\
0.154	&	5.439	&	0.209	&	2.052	&	0.223	&	3.430	\\
0.172	&	6.181	&	0.236	&	2.664	&	0.246	&	4.233	\\
0.188	&	6.670	&	0.261	&	3.122	&	0.264	&	4.780	\\
0.204	&	6.986	&	0.283	&	3.378	&	0.281	&	5.149	\\
0.222	&	7.121	&	0.306	&	3.449	&	0.299	&	5.354	\\
0.244	&	7.025	&	0.332	&	3.311	&	0.318	&	5.389	\\
0.278	&	6.520	&	0.360	&	2.936	&	0.340	&	5.222	\\
0.340	&	5.010	&	0.399	&	2.209	&	0.372	&	4.717	\\
0.462	&	1.247	&	0.466	&	0.567	&	0.425	&	3.440	\\
0.616	&	-4.283	&	0.564	&	-2.211	&	0.524	&	0.444	\\
0.788	&	-11.032	&	0.715	&	-6.983	&	0.701	&	-5.814	\\
1.000	&	-19.814	&	1.000	&	-16.923	&	1.000	&	-17.803	\\

		\hline
	\end{tabular}
\end{table*}

\begin{table*}[h]
	\centering
	\caption{Optimized values of $\lambda$ and $\eta$ for the phenanthrene in different mass fractions of $CO_{2}$ in toluene }
	\begin{tabular}{llllllll}
		\hline
		\multicolumn{2}{c}{0.087}& \multicolumn{2}{c}{0.119}& \multicolumn{2}{c}{0.169}& \multicolumn{2}{c}{0.289}\\
		\hline
		$\lambda$ & $\eta$ & $\lambda$ & $\eta$  & $\lambda$ & $\eta$  & $\lambda$ & $\eta$ \\ 
		\hline
0.000	&	0.000	&	0.000	&	0.000	&	0.000	&	0.000	&	0.000	&	0.000	\\
0.128	&	0.604	&	0.128	&	0.732	&	0.064	&	0.883	&	0.066	&	0.806	\\
0.184	&	2.067	&	0.186	&	2.223	&	0.108	&	0.764	&	0.111	&	0.760	\\
0.217	&	3.164	&	0.219	&	3.319	&	0.175	&	1.969	&	0.172	&	1.983	\\
0.240	&	3.940	&	0.244	&	4.098	&	0.214	&	3.156	&	0.204	&	2.967	\\
0.260	&	4.472	&	0.267	&	4.704	&	0.240	&	3.974	&	0.227	&	3.627	\\
0.277	&	4.823	&	0.289	&	5.031	&	0.258	&	4.457	&	0.245	&	4.082	\\
0.295	&	5.035	&	0.313	&	5.084	&	0.273	&	4.750	&	0.262	&	4.395	\\
0.318	&	5.059	&	0.339	&	4.950	&	0.287	&	4.921	&	0.279	&	4.583	\\
0.347	&	4.762	&	0.373	&	4.371	&	0.305	&	4.962	&	0.299	&	4.621	\\
0.397	&	3.753	&	0.425	&	3.055	&	0.326	&	4.885	&	0.325	&	4.423	\\
0.491	&	1.031	&	0.488	&	1.196	&	0.361	&	4.401	&	0.365	&	3.739	\\
0.670	&	-5.148	&	0.525	&	-0.027	&	0.419	&	2.990	&	0.428	&	2.198	\\
0.791	&	-9.713	&	0.730	&	-7.185	&	0.527	&	-0.299	&	0.530	&	-0.842	\\
1.000	&	-18.098	&	1.000	&	-17.769	&	0.697	&	-6.180	&	0.701	&	-6.763	\\
&		&		&		&	1.000	&	-17.998	&	1.000	&	-18.163	\\
		\hline
	\end{tabular}
\end{table*}

\begin{table*}[h]
	\centering
	\caption{Optimized values of $\lambda$ and $\eta$ for the solutes in water }
	\begin{tabular}{llllllll}
		\hline
		\multicolumn{2}{c}{propane}& \multicolumn{2}{c}{benzene}& \multicolumn{2}{c}{toluene}& \multicolumn{2}{c}{phenanthrene}\\
		\hline
		$\lambda$ & $\eta$ & $\lambda$ & $\eta$  & $\lambda$ & $\eta$  & $\lambda$ & $\eta$ \\ 
		\hline
0.000	&	0.000	&	0.000	&	0.000	&	0.000	&	0.000	&	0.000	&	0.000	\\
0.107	&	2.673	&	0.193	&	-0.295	&	0.177	&	0.182	&	0.142	&	-2.462	\\
0.157	&	4.703	&	0.279	&	1.468	&	0.262	&	2.432	&	0.256	&	0.597	\\
0.186	&	6.047	&	0.324	&	2.931	&	0.307	&	4.244	&	0.319	&	4.504	\\
0.210	&	7.148	&	0.357	&	4.168	&	0.336	&	5.552	&	0.358	&	7.762	\\
0.230	&	8.017	&	0.381	&	5.091	&	0.360	&	6.696	&	0.384	&	10.104	\\
0.250	&	8.883	&	0.405	&	5.891	&	0.380	&	7.558	&	0.407	&	12.185	\\
0.272	&	9.291	&	0.427	&	6.443	&	0.400	&	8.233	&	0.427	&	13.607	\\
0.294	&	9.700	&	0.449	&	6.770	&	0.422	&	8.678	&	0.446	&	14.490	\\
0.328	&	9.900	&	0.476	&	6.900	&	0.443	&	8.859	&	0.469	&	14.834	\\
0.381	&	9.930	&	0.506	&	6.805	&	0.473	&	8.810	&	0.494	&	14.667	\\
0.484	&	9.463	&	0.555	&	6.392	&	0.514	&	8.452	&	0.533	&	13.832	\\
0.623	&	8.195	&	0.653	&	5.109	&	0.606	&	7.148	&	0.620	&	11.069	\\
0.781	&	6.378	&	0.810	&	2.421	&	0.755	&	4.273	&	0.806	&	3.279	\\
1.000	&	3.333	&	1.000	&	-1.480	&	1.000	&	-1.547	&	1.000	&	-6.122	\\


		\hline
	\end{tabular}
\end{table*}
% ----------------------------------------------------------
\chapter{Overlapping Matrices}
% ----------------------------------------------------------

\begin{figure}[H]
	\centering
	\includegraphics[width=0.9\textwidth]{Figures/ohex_pyr}
	\caption{Overlapping matrix for hexane+pyrene.}
\end{figure}

\begin{figure}[H]
	\centering
	\includegraphics[width=0.9\textwidth]{Figures/ohex_phen}
	\caption{Overlapping matrix for hexane+phenanthrene.}
\end{figure}

\begin{figure}[H]
	\centering
	\includegraphics[width=0.9\textwidth]{Figures/ooct_prop}
	\caption{Overlapping matrix for 1-octanol+propane.}
\end{figure}

\begin{figure}[H]
	\centering
	\includegraphics[width=0.9\textwidth]{Figures/ooct_ant}
	\caption{Overlapping matrix for 1-octanol+anthracene.}
\end{figure}

\begin{figure}[H]
	\centering
	\includegraphics[width=0.9\textwidth]{Figures/ooct_phen}
	\caption{Overlapping matrix for 1-octanol+phenanthrene.}
\end{figure}

\begin{figure}[H]
	\centering
	\includegraphics[width=0.9\textwidth]{Figures/otol_pyr}
	\caption{Overlapping matrix for toluene+pyrene.}
\end{figure}

\begin{figure}[H]
	\centering
	\includegraphics[width=0.9\textwidth]{Figures/otol_antr}
	\caption{Overlapping matrix for toluene+anthracene.}
\end{figure}

\begin{figure}[H]
	\centering
	\includegraphics[width=0.9\textwidth]{Figures/otol_phen}
	\caption{Overlapping matrix for toluene+phenanthrene.}
\end{figure}

\begin{figure}[H]
	\centering
	\includegraphics[width=0.9\textwidth]{Figures/otolco2_1}
	\caption{Overlapping matrix for toluene+$CO_{2}$(0.087)+phenanthrene.}
\end{figure}
\begin{figure}[H]
	\centering
	\includegraphics[width=0.9\textwidth]{Figures/otolco2_2}
	\caption{Overlapping matrix for toluene+$CO_{2}$(0.119)+phenanthrene.}
\end{figure}

\begin{figure}[H]
	\centering
	\includegraphics[width=0.9\textwidth]{Figures/otolco2_3}
	\caption{Overlapping matrix for toluene+$CO_{2}$(0.169)+phenanthrene.}
\end{figure}

\begin{figure}[H]
	\centering
	\includegraphics[width=0.9\textwidth]{Figures/otolco2_4}
	\caption{Overlapping matrix for toluene+$CO_{2}$(0.289)+phenanthrene.}
\end{figure}

\begin{figure}[H]
	\centering
	\includegraphics[width=0.9\textwidth]{Figures/owat_prop}
	\caption{Overlapping matrix for water+propane.}
\end{figure}

\begin{figure}[H]
	\centering
	\includegraphics[width=0.9\textwidth]{Figures/owat_benz}
	\caption{Overlapping matrix for water+benzene.}
\end{figure}

\begin{figure}[H]
	\centering
	\includegraphics[width=0.9\textwidth]{Figures/owat_tol}
	\caption{Overlapping matrix for water+toluene.}
\end{figure}

\begin{figure}[H]
	\centering
	\includegraphics[width=0.9\textwidth]{Figures/owat_phen}
	\caption{Overlapping matrix for water+phenanthrene.}
\end{figure}

\chapter{Work Published in Scientific Conference}

\includepdf[pages={1-}]{C089.pdf}

\chapter{Paper for Publication in Scientific Journal}

\includepdf[pages={1-}]{article/achemso-demo}
\end{apendicesenv}
% ---


% ----------------------------------------------------------
% Anexos
% ----------------------------------------------------------

% ---
% Inicia os anexos
%% ---
%\begin{anexosenv}
%
%% Imprime uma página indicando o início dos anexos
%\partanexos
%
%% ---
%\chapter{Morbi ultrices rutrum lorem.}
%% ---
%
%% ---
%\chapter{Fusce facilisis lacinia dui}
%% ---
%
%\end{anexosenv}

%---------------------------------------------------------------------
% INDICE REMISSIVO
%---------------------------------------------------------------------
\phantompart
\printindex
%---------------------------------------------------------------------

\end{document}
